\subtitle{Topologie des deutschen Satzes}

\huberlintitlepage[22pt]

\section{Topologie des deutschen Satzes}

\iftoggle{teil1}{
\outline{

\begin{itemize}
\item Ziele
\item Wozu Syntax? / Phrasenstrukturgrammatiken
\item Formalismus
\item Valenz und Grammatikregeln
\item Komplementation
\item Semantik
\item Adjunktion und Spezifikation
\item Das Lexikon: Typen und Lexikonregeln
\item \blau{Topologie des deutschen Satzes}
\item Konstituentenreihenfolge
\item Nichtlokale Abhängigkeiten
\item Relativsätze
\item Lokalität
%\item Komplexe Prädikate: Der Verbalkomplex
\end{itemize}
}
} % \end{teil1}

\iftoggle{teil2}{
\outline{
\begin{itemize}
\item Wiederholung
      \begin{itemize}
\item Wozu Syntax? / Phrasenstrukturgrammatiken
\item Formalismus
\item Valenz und Grammatikregeln
\item Komplementation
\item Semantik
\item \blau{Topologie des deutschen Satzes}
\item Konstituentenreihenfolge
\item Nichtlokale Abhängigkeiten
\end{itemize}
\item Kongruenz
\item Kasus
\item Der Verbalkomplex
\item Kohärenz, Inkohärenz, Anhebung und Kontrolle
\item Passiv
\item Partikelverben
\item Morphologie
\end{itemize}
}
} % \end{teil2}

\frame{
\frametitle{Literaturhinweise}
%
\begin{itemize}
\item Literatur: \citew[Kapitel~8]{MuellerLehrbuch3}

\item Ausführlich: \citew{Woellstein2010a-u}

\end{itemize}

\vspace{1cm}

%% \rotbf{Achtung, wichtiger Hinweis: Diese Literaturangabe hier bedeutet,\\dass Sie die Literatur zum
%%   nächsten Mal lesen sollen!!!!}
}


\subsection{Verbstellungstypen, Satzklammern und topologische Felder}

\frame{
\frametitle{Topologie des deutschen Satzes}

\savespace
\begin{itemize}
\item Verbendstellung
      \ea
      Peter hat erzählt, \rot<5->{dass} er das Eis \braun<5->{\gruen<4>{gegessen} \blauit<-4>{hat}}.
      \z
\pause
\item Verberststellung
        \ea
      \rot<4->{\blauit<-3>{Hat}} Peter das Eis \braun<5->{\gruen<4>{gegessen}}?
      \z
\pause
\item Verbzweitstellung
      \ea
      Peter \rot<4->{\blauit<-3>{hat}} das Eis \braun<5->{\gruen<4>{gegessen}}.
      \z
\end{itemize}

\pause
\begin{itemize}[<+->]
\item verbale Elemente nur in (\mex{-2}) kontinuierlich
\item \rot<5->{linke} und \braun<5->{rechte} Satzklammer
\item Komplementierer (\emph{weil}, \emph{dass}, \emph{ob}) in der linken Satzklammer
\item Komplementierer und finites Verb sind komplementär verteilt
\item Bereiche vor, zwischen u.\ nach Klammern: Vorfeld, Mittelfeld, Nachfeld
\end{itemize}


}

\frame{

\resizebox{\textwidth}{!}{
\begin{tabular}{@{}lllll@{}}
\rowcolor{structure!15}Vorfeld & linke Klammer & Mittelfeld                             & rechte Klammer & Nachfeld                   \\ 
\\
\rowcolor{structure!10}Karl    & schläft.      &                                        &                &                            \\
                       Karl    & hat           &                                        & geschlafen.    &                            \\
\rowcolor{structure!10}Karl    & erkennt       & Maria.                                 &                &                            \\
                       Karl    & färbt         & den Mantel                             & um             & den Maria kennt.           \\
\rowcolor{structure!10}Karl    & hat           & Maria                                  & erkannt.       &                             \\
                       Karl    & hat           & Maria als sie aus dem Zug stieg sofort & erkannt.       &                             \\
\rowcolor{structure!10}Karl    & hat           & Maria sofort                           & erkannt        & als sie aus dem Zug stieg. \\
                       Karl    & hat           & Maria zu erkennen                      & behauptet.     &            \\
\rowcolor{structure!10}Karl    & hat           &                                        & behauptet      & Maria zu erkennen.         \\ 
\\
\rowcolor{structure!10}        & Schläft       & Karl?                                  &                &                            \\
                               & Schlaf!       &                                        &                &                             \\
\rowcolor{structure!10}        & Iß            & jetzt dein Eis                         & auf!           &                             \\
        & Hat           & er doch das ganze Eis alleine          & gegessen.            &      \\  \\
\rowcolor{structure!10}        & weil          & er das ganze Eis alleine               & gegessen hat   & ohne sich zu schämen.\\
        & weil          & er das ganze Eis alleine               & essen können will    & ohne gestört zu werden.    \\
%\rowcolor{structure!10}        & wer           & das ganze Eis alleine                  & gegessen hat.  &                             \\
\end{tabular}
}

}

\subsection{Zuordnung zu den Feldern}

\frame{

\begin{itemize}
\item mehrere Verben in der rechten Satzklammer: Verbalkomplex
\pause
%\iftoggle{teil1}{
\item manchmal wird auch von diskontinuierlichen Verbalkomplexen
      gesprochen (Initialstellung das Finitums)
\pause
\item auch prädikative Adjektive, Resultativprädikate:
      \eal
      \ex dass Karl seiner Frau treu ist.
      \ex dass Karl das Glas leer trinkt.
      \zl
\pause
%} % \end{teil1}
\item Felder nicht immer besetzt
      \ea
      \field{Der Mann}{VF} \field{gibt}{LS} \field{der Frau das Buch,}{MF} \field{die er kennt}{NF}.
      \z
\pause
\item Test: Rangprobe \citep[S.\,72]{Bech55a}
\eal
\ex[]{
Der Mann hat der Frau das Buch gegeben, die er kennt.
}
\ex[*]{
Der Mann hat der Frau das Buch, die er kennt, gegeben.
}
\zl
\end{itemize}
}

\subsection{Rekursion}

\frame{
\frametitle{Rekursives Auf"|tauchen der Felder}

\begin{itemize}
\item \citet*[S.\,82]{Reis80a}: Rekursion\\
      Vorfeld kann in Felder unterteilt sein:
\eal
\ex Die Möglichkeit, etwas zu verändern, ist damit \braun{verschüttet} \goldenrod{für lange lange Zeit}.
\ex {}[\braun{Verschüttet} \goldenrod{für lange lange Zeit}] ist damit die Möglichkeit, 
      etwas zu ver"-ändern.
\pause
\ex Wir haben schon seit langem \braun{gewußt}, \goldenrod{dass du kommst}.
\ex {}[\braun{Gewußt}, \goldenrod{dass du kommst},] haben wir schon seit langem.
\zl
\braun{rechte Satzklammer} und \goldenrod{Nachfeld} innerhalb das Vorfelds
\pause
\item im Mittelfeld beobachtbare Permutationen auch im Vorfeld

\eal
\ex Seiner Tochter ein Märchen erzählen wird er wohl müssen.
\ex Ein Märchen seiner Tochter erzählen wird er wohl müssen.
\zl
\end{itemize}

}

\subsection{Übungsaufgaben}


\frame{
\frametitle{Übungsaufgaben}

\begin{enumerate}
\item Bestimmen Sie Vorfeld, Mittelfeld und Nachfeld in den folgenden Sätzen:
\eal
\ex Karl ißt.
\ex Der Mann liebt eine Frau, den Peter kennt.
\ex Der Mann liebt eine Frau, die Peter kennt.
\ex Die Studenten behaupten, nur wegen der Hitze einzuschlafen.
\ex Die Studenten haben behauptet, nur wegen der Hitze einzuschlafen.
\zl
\end{enumerate}

\pause\pause\pause

}
