\section{Nichtlokale Abhängigkeiten}

\subtitle{Nichtlokale Abhängigkeiten}

\huberlintitlepage[22pt]


\iftoggle{teil1}{
\outline{

\begin{itemize}
%\item Ziele
\item Wozu Syntax? / Phrasenstrukturgrammatiken
\item Formalismus
\item Valenz und Grammatikregeln
\item Komplementation
\item Semantik
\item Adjunktion und Spezifikation
\item Das Lexikon: Typen und Lexikonregeln
\item Topologie des deutschen Satzes
\item Konstituentenreihenfolge
\item \blau{Nichtlokale Abhängigkeiten}
\item Relativsätze
\item Lokalität
%\item Komplexe Prädikate: Der Verbalkomplex
\end{itemize}
}
} % \end{teil1}

\iftoggle{teil2}{
\outline{

\begin{itemize}
\item Wiederholung
      \begin{itemize}
\item Wozu Syntax? / Phrasenstrukturgrammatiken
\item Formalismus
\item Valenz und Grammatikregeln
\item Komplementation
\item Topologie des deutschen Satzes
\item Konstituentenreihenfolge
\item \blau{Nichtlokale Abhängigkeiten}
\item Lokalität
\end{itemize}
\item Kongruenz
\item Kasus
\item Der Verbalkomplex
\item Kohärenz, Inkohärenz, Anhebung und Kontrolle
\item Passiv
\item Partikelverben
\item Morphologie
\end{itemize}
}
} % \end{teil2}

\frame{
\frametitle{Literaturhinweise}
%
\begin{itemize}
\item Literatur: \citew[Kapitel~10.1--10.2]{MuellerLehrbuch3}
\item Außerdem Handbuchartikel zu Fernabhängigkeiten: \citew{BC2021a}
\end{itemize}

\vspace{1cm}

%% \rotbf{Achtung, wichtiger Hinweis: Diese Literaturangabe hier bedeutet,\\dass Sie die Literatur zum
%%   nächsten Mal lesen sollen!!!!}
}


\subsection{Verschiedene Arten von Fernabhängigkeiten}

\frame[shrink]{
\frametitle{Das Deutsche als V2-Sprache}

Vorfeld kann mit einer Konstituente (Adjunkt, Subjekt o.\ Komplement) besetzt sein (\citealp{Erdmann1886a};
\citealp{Paul1919a}) $\to$ Verbzweitsprache

\resizebox{\linewidth}{!}{
\begin{tabular}[t]{@{}l@{~}ll@{}}
a. & Schläft Karl?                                        & Karl schläft.\\
\pause
b. & Kauft Karl diese Jacke?                              & Karl kauft diese Jacke.\\
\pause
   &                                                      & Diese Jacke kauft Karl.\\
\pause
c. & Kauft Karl morgen diese Jacke?                       & Morgen kauft Karl diese Jacke.\\
\pause
d. & Wird die Jacke von Karl gekauft?                     & Von Karl wird die Jacke gekauft.\\
\pause
e. & Ist Maria schön?                                     & Schön ist Maria.\\
\pause
f. & Muß man sich kämmen?                                 & Man muss sich kämmen.\\
\pause
   &                                                      & Sich kämmen muss man.\\
\pause
g. & Glaubt Karl, dass Maria ihn liebt?                    & Daß Maria ihn liebt, glaubt Karl.\\
\pause
h. & Lacht Karl, weil er den Trick kennt?   & Weil er den Trick kennt, lacht Karl.\\
\pause
i. & Schlaf jetzt endlich!                                & Jetzt schlaf endlich!\\
\end{tabular}
}

}


\frame[label=nla]{
\frametitle{Vorfeldbesetzung als nichtlokale Abhängigkeit}

\savespace\smallexamples
\begin{itemize}
\item Linearisierungsansätze: \\
      \citet*{NSW94a} (für Voranstellung von Idiomteilen)\\ 
      \citet[Kapitel~6.3]{Kathol95a} für einfache Voranstellungen
\pause
\item keine Lösung für alle Fälle:
\eal
\ex
{}[Um zwei Millionen Mark]$_i$ soll er versucht haben,\\{}[eine Versicherung \_$_i$ zu betrügen].\footnote{
         taz, 04.05.2001, S.\,20.
}
\pause
\ex
"`Wer$_i$, glaubt\iw{glauben} er, dass er \_$_i$ ist?"' erregte sich ein Politiker vom Nil.\footnote{
        Spiegel, 8/1999, S.\,18.
}
\pause
\ex Wen$_i$ glaubst du, dass ich \_$_i$ gesehen habe.\footnote{
    \citew[S.\,84]{Scherpenisse86a}.
    }
\zl
\pause
\item Zusammengehörigkeit wird durch Indizes gekennzeichnet.\\
\_$_i$ steht für die Lücke bzw.\ \blau{Spur} ({\it gap\/} bzw.\ {\it trace\/})\\
{}[\emph{um zwei Millionen Mark}]$_i$ ist \blau{Füller}

\end{itemize}

}

\iftoggle{teil1}{

\frame{
\frametitle{Andere Fernabhängigkeiten: Extraposition} 


\savespace
\begin{itemize}
\item {\it unbounded dependencies\/} vs.\ {\it long distance dependencies\/}
\pause
\item durch Satzgrenze beschränkt:
\eal
\ex Der Mann hat [der Frau \_$_i$] den Apfel gegeben,\\
    {}[die er am schönsten fand]$_i$.
\ex Der Mann hat \_$_i$ behauptet,\\
    {}[einer Frau den Apfel gegeben zu haben]$_i$.
\zl
\pause
\item aber wirklich nicht lokal:
\ea
Karl hat mir\\
{}[von [der Kopie [einer Fälschung [eines Bildes [einer \blau{Frau}]]]]] erzählt, \blau{die} schon lange tot ist.
\z
Zur Nichtlokalität der Extraposition siehe auch \citet{Mueller2004d}.
\pause
\item Zur Extraposition in HPSG: \citew{Keller95b,Bouma96,Mueller99a}.
\end{itemize}

}
} % \end{teil1}

\subsection{Vorfeldbesetzung}


\frame{

%\frametitle{Repräsentationen und Lexikonregeln: 
\frametitle{Überblick: Vorfeldbesetzung}

\vfill
\hfill%
\scalebox{0.7}{
%\small
\psset{xunit=1cm,yunit=5.4mm}
%
% node labels for moving elements will be typeset by the \tmove command
% here we have to provide invisible boxes to get the line drawing right.
\begin{pspicture}(-.1,0.8)(9.4,12)
%
%\psgrid
%
\only<2->{\pscurve[%showpoints=true,%
%arrows=<->](6.6,2.8)(6.7,2.4)(7.0,2.4)(7.4,3.2)(8.3,5.2)(6.8,7.3)(5.1,9.2)(3,10.2)(1.3,9.6)
linecolor=green,arrows=<->](4.6,4.8)(4.7,4.4)(5.0,4.4)(5.3,5.2)(6.8,7.3)(5.1,9.2)(3,10.2)(1.3,9.6)
}
%
\rput[B](1,1){\rnode{speccp}{\blau<1>{diesen Mann$_i$}}}
\rput[B](3,1){\rnode{cleer}{{kennt}}}
\rput[B](5,1){\rnode{jeder}{\blau<1>{[ \_ ]$_i$}}}
\rput[B](7,1){\rnode{ihnmf}{jeder}}
\rput[B](9,1){\rnode{kennt}{{[ \_ ]$_k$}}}
%
\rput[B](7,3){\rnode{np1}{{NP}}}
\rput[B](9,3){\rnode{v}{V}}
%
\rput[B](5,5){\rnode{np2}{\blau<1>{NP/NP}}}
\rput[B](8,5){\rnode{vs1}{V$'$}}
%
\rput[B](6.5,7){\rnode{vp}{VP/NP}}
%
%
\rput[B](3,5){\rnode{vlex}{V}}
%
\rput[B](3,7){\rnode{c}{V}}
%
\rput[B](1,9){\rnode{np3}{\blau<1>{NP}}}
\rput[B](4.75,9){\rnode{cs}{VP/NP}}
%
%
\rput[B](3.0625,11){\rnode{cp}{VP}}
%
%
%
%
\psset{angleA=-90,angleB=90,arm=0pt}
%
\ncdiag{v}{kennt}
\ncdiag{vs1}{np1}\ncdiag{vs1}{v}
\ncdiag{vs2}{np2}\ncdiag{vs2}{vs1}
\ncdiag{vp}{vs2}
%
\ncdiag{np3}{t1}
%
\ncdiag{i}{t2}
\ncdiag{is}{i}\ncdiag{is}{vp}
\ncdiag{vp}{np2}\ncdiag{ip}{is}
%
\alt<1>{\pstriangle[linecolor=blue](1,1.6)(2.0,7.2)}{\pstriangle(1,1.6)(2.0,7.2)}
\alt<1>{\ncdiag[linecolor=blue]{np2}{jeder}}{\ncdiag{np2}{jeder}}
\ncdiag{np1}{ihnmf}
\ncdiag{vp}{vs1}
\ncdiag{c}{vlex}
\ncdiag{vlex}{cleer}
\ncdiag{cs}{c}\ncdiag{cs}{vp}
\ncdiag{cp}{np3}
\ncdiag{cp}{cs}
%
\end{pspicture}
}
\hfill\hfill\mbox{}
\vfill
\begin{itemize}
\item Wie bei Verbbewegung: Spur an ursprünglicher "`normaler"' Position.
\pause
\item Weiterreichen der Information im Baum
\pause
\item Konstituentenbewegung ist nicht lokal, Verbbewegung ist lokal\\
      mit zwei verschiedenen Merkmalen modelliert ({\sc slash} vs.\ {\sc dsl})
\end{itemize}

%\handoutspace
}

\subsubsection{Eigenschaften der Analyse}

\frame{
\frametitle{Eigenschaften der Analyse}

\savespace

\begin{itemize}[<+->]
\item Perkolation nichtlokaler Information
\item Strukturteilung
\item Information ist gleichzeitig an jedem Knoten präsent.
\item Knoten in der Mitte einer Fernabhängigkeit können darauf zugreifen\\
      (\citet*{BMS2001a}: Irisch, Chamorro, Palauan, Isländisch,\\
      \hspaceThis{(}Kikuyu, Ewe, Thompson Salish, Moore, Französisch, Spanisch, Jiddisch)
\end{itemize}

}

\subsubsection{Datenstruktur: Unterteilung lokale/nichtlokale Information}

\frame{
\frametitle{Datenstruktur: Unterteilung lokale/nichtlokale Information}

\begin{itemize}
\item Unterteilung in Information, die lokal relevant ist ({\sc local})\\
      und solche, die in Fernabhängigkeiten eine Rolle spielt ({\sc nonlocal})

\bigskip
\ms[sign]{
phon   & list~of~phoneme strings\\
 loc & \ms[loc]{ cat  & \ms[cat]{ head   & head \\
                                                       subcat & list of synsem objects\\
                                                     } \\
                                      cont & cont\\
                                    }\\
                      nonloc & nonloc\\
}
\end{itemize}

}


\frame{
\frametitle{Datenstruktur für nichtlokale Information}

\begin{itemize}
\item \nonlocw ist weiter strukturiert:
\medskip

\ms[nonloc]{
 que & \type{list~of~npros} \\
 \visible<2->{rel & \type{list~of~indices}} \\
 \visible<3->{slash & \type{list~of~local~phrases}} \\ %\\
 %extra & \ms[list~of~local~phrases]{} \\
}
\medskip

\item {\sc que}: Liste von Indizes von Fragewörtern (Interrogativsätze)
\pause
\item {\sc rel}: Liste von Indizes von Relativpronomina (Relativsätze)
\pause
\item {\sc slash}: Liste von {\it local\/}"=Objekten (Vorfeldbesetzung, Relativsätze)
\pause
\item {\sc que} wird im folgenden weggelassen.
\end{itemize}

}

\subsubsection{Die Spur für das Akkusativobjekt}

\frame{
\frametitle{Spur für das Akkusativobjekt von \emph{kennen}}

\hspace{1em}\ms[word]{
 phon & \blau<2>{\phonliste{}} \\
 loc & \blau<4>{\ibox{1}} \blau<3>{\ms{ cat \ms{ head & \ms[noun]{
                                                     cas & acc\\
                                                     } \\
                                              subcat & \sliste{}\\
                                            } \\
                                   }}\\
               nonloc & \ms{ %que   & \sliste{} \\
%                               rel & \sliste{} \\
                                slash & \sliste{ \blau<4>{\ibox{1}} } \\
                                %extra & \liste{} \\
                             } \\
}

\pause

\begin{itemize}
\item Die Spur hat keinen phonologischen Beitrag.
\pause
\item Die Spur hat die lokalen Eigenschaften, die \emph{kennen} verlangt.
\pause
\item Diese werden auch in \slasch eingeführt.
\end{itemize}

}


\subsubsection{Die Perkolation nichtlokaler Information}

\frame{
\frametitlefit{Die Perkolation nichtlokaler Information (vereinfacht $\to$ falsche Verbstellung!)}


\centerline{%
\begin{forest}
sm edges
[{V[\begin{tabular}[t]{@{}l@{}}
    \subcat \eliste\alt<beamer|beamer:1>{]}{,}\\
    \visible<2->{\blau<3>{\slasch \eliste}]}\end{tabular}}, s sep+=2ex 
  [\blau<3>{NP\ibox{1}[\type{acc}]} [das Buch,roof]]
  [{V[\begin{tabular}[t]{@{}l@{}}
      \subcat \sliste{ }\alt<beamer|beamer:1>{]}{,}\\
      \visible<2->{\blau<2-3>{\slasch \sliste{ \ibox{1} }}]}\end{tabular}}
    [{V[\begin{tabular}[t]{@{}l@{}}
      \subcat \sliste{ \ibox{2} }\alt<beamer|beamer:1>{]}{,}\\
      \visible<2->{\blau<2>{\slasch \sliste{ \ibox{1} }}]}\end{tabular}}
      [{V[\begin{tabular}[t]{@{}l@{}}
      \subcat \sliste{ \ibox{2}, \blau<1>{\ibox{3}} }\alt<beamer|beamer:1>{]}{,}\\
      \visible<2->{\slasch \sliste{ }]}\end{tabular}}
        [kennt]]
      [{\blau<1>{\ibox{3} NP}[\begin{tabular}[t]{@{}l@{}}\type{acc}\alt<beamer|beamer:1>{]}{,}\\
                                      \visible<2->{\blau<2>{\slasch \sliste{ \ibox{1} }}]}\end{tabular}}
        [\trace]]]
    [{\ibox{2} NP[\type{nom}]}
      [jeder]]]]
\end{forest}}
\pause

%\alt<1>{]}{,}
}


\subsubsection{Das Kopf-Füller-Schema}

\frame{
\frametitle{Das Kopf"=Füller"=Schema}
\smallframe
%\savespace
\scalebox{0.9}{%
\begin{tabular}{@{}l@{}}
\type{head-filler-phrase} $\to$\\
\ms{ 
nonloc$|$slash &   \eliste\\
head-dtr      & \highlight<1>{\onems{ loc$|$cat \onems{ head \ms[verb]{vform & fin\\
                                                                 initial & {\rm +}\\
                                                                }\\
                                                  subcat \liste{}\\
                                               }\\
                             \visible<2->{\highlight<2>{nonloc$|$slash   \sliste{ \highlight<3>{\ibox{1}} }}}}}\\
\visible<3->{non-head-dtrs & \liste{ \onems{ \highlight<3>{loc \ibox{1}}\\
                                           \visible<4->{\highlight<4>{nonloc$|$slash \liste{}}} \\
                                 }}}\\
   }
\end{tabular}}

\begin{itemize}
\item Kopftochter ist ein finiter Satz mit Verb in Verberststellung ({\sc initial}+) 
\pause
      und\\ einem Element in {\sc slash}
\pause
\item {\sc local}"=Wert der Nicht-Kopftochter ist identisch mit Element in {\sc slash}
\pause
\item Aus Nicht-Kopftochter kann nichts extrahiert werden.
\end{itemize}
}

\frame{
\frametitle{Eigenschaften von Kopf"=Füller"=Strukturen}

\begin{itemize}[<+->]
\item Es werden keine Argumente gesättigt.\\
      {\it head-filler-phrase\/} ist Untertyp von {\it head-non-argument-phrase\/}.
\item Semantischer Beitrag kommt vom Verb (der Kopftochter).\\
      {\it head-filler-phrase\/} ist Untertyp von {\it head-non-adjunct-phrase\/}.
\end{itemize}

}



\iftoggle{teil1}{
\subsubsection{Typhierarchie für \type{sign}}



\frame{
\frametitle{Typhierarchie für \type{sign}}

\vfill
\oneline{
\begin{forest}
type hierarchy,
 for tree={
   calign=fixed angles,
   calign angle=50
 } 
[sign
  [word]
  [phrase
    [non-headed-phrase]
    [headed-phrase
      [head-non-adjunct-phrase%, for current and siblings={l sep*=4}, 
        [head-argument-phrase]
        [head-filler-phrase]]
      [head-non-argument-phrase
       [,identify=!r2212]
       [head-adjunct-phrase]]]]]
\end{forest}}
\vfill

}
} % \end{teil1}


\subsubsection{Die Extraktionsspur}

\frame{
\frametitle{Die Extraktionsspur}


\hspace{1em}\ms[word]{
 phon   & \phonliste{} \\
 loc    & \ibox{1} \\
 nonloc & \ms{ %que   & \sliste{} \\
 %                                               rel & \sliste{} \\
                                                slash & \sliste{ \ibox{1} } \\ 
                                                %extra & \sliste{} \\
                                              } \\ 
}
\begin{itemize}
\item Wie bei der Verbbewegung können wir abstrahieren.
\pause
\item Über den \localw müssen wir in der Spur nichts sagen,\\
      denn das Verb weiß ja, was es will,\\
      und stellt Anforderungen an den \localw seines Arguments.
\end{itemize}


}

\iftoggle{teil1}{

\subsubsection{Extraktion zusammen mit Verbbewegung}
\frame{
\frametitle{Extraktion zusammen mit Verbbewegung}

\centerline{
\scalebox{.85}{\begin{forest}
sm edges,
for level={3}{l+=\baselineskip}
[V\feattab{\subcat \eliste,\\
           \slasch \eliste}, s sep+=1em
  [{NP\ibox{1}[\type{acc}]}
     [das Buch, roof]]
  [V\feattab{\subcat \eliste,\\
             \slasch  \sliste{ \ibox{1} }}
     [{V[\subcat \sliste{ \ibox{2} }]}
       [{V[\subcat \sliste{ \ibox{3}, \ibox{4} }]},edge label={node[midway,right]{V1-LR}}
         [kennt]]]
     [\ibox{2} V\feattab{\subcat \eliste, \slasch \sliste{ \ibox{1} }}
         [\ibox{4} \feattab{\textsc{loc} \ibox{1},\\
                            \slasch \sliste{ \ibox{1} } }
         [\trace]]
       [V\feattab{\subcat \sliste{ \ibox{3} }, \slasch \sliste{ \ibox{1} }}
         [{\ibox{3} NP[\type{nom}]}
           [jeder]]
         [V\feattab{\subcat \sliste{ \ibox{3}, \ibox{4} }}
           [\trace]]]]]]
\end{forest}}
}
}

} % \end{teil1}

\iftoggle{teil1}{
\subsection{Probleme mit Spuren}

\frame{
\frametitle{Linguistische Probleme mit Spuren}

\begin{itemize}
\item Koordination
      \begin{itemize}
      \item {\sc cat}"=Werte und {\sc nonloc}"=Werte der Konjunkte werden unifiziert
      \item Mutter hat dieselben {\sc nonloc}"=Werte wie Töchter
      \item
        Across the Board"=Extraktion (ATB)
\ea
Bagels$_i$ [[I like \_$_i$] and [Alison hates \_$_i$]].
\z
\item aber nicht mit Spuren
\ea[*]{
Bagels$_i$ I like [\_$_i$ and \_$_i$].
}
\z
%      \item Selektion der Phonologie nicht möglich ({\it synsem}) $\to$ Merkmal stipulieren
      \end{itemize}
\end{itemize}

}
\frame{

\smallframe
\frametitle{Linguistische Probleme mit Spuren}
\begin{itemize}
\item Linearisierung (in Abhängigkeit von anderen Annahmen in der Grammatik)
      \ea
      Dem Mann$_i$ hilft eine Frau \_$_i$. ~~~vs.~~ Dem Mann$_i$ hilft \_$_i$ eine Frau.\\
      \z
\pause
\medskip
\item Restriktion auf Nicht"=Köpfe
      \eal
\ex[]{
{}[Der kluge Mann]$_i$ hat \_$_i$ geschlafen.
}
\ex[*]{
{}[Mann]$_i$ hat der kluge \_$_i$ geschlafen.
}
\zl
\end{itemize}

}

\frame{
\frametitle{Verarbeitungsprobleme bei der Annahme von Spuren}



\begin{itemize}
\item In Abhängigkeit vom Parser:\\
      Hypothesen für leere Elemente, die nie benutzt werden\\
      der \_ Mann
\end{itemize}


}

\subsection{Einführung nichtlokaler Abhängigkeiten}

\frame{
\frametitle{Einführung nichtlokaler Abhängigkeiten}



\begin{itemize}
\item Spur
\item Unäre Projektion
\item Lexikonregel
\item unterspezifizierte Lexikoneinträge und relationale Beschränkungen
\end{itemize}


}

\subsection{Grammatiktransformation}
\frame{
%
\frametitle{Grammatiktransformation}

\citet*{BHPS61a}:

\begin{tabular}{@{}l@{\hspace{1cm}}c@{\hspace{1cm}}l}
\baro{v}   $\to$ \mbox{v}, \mbox{np}        &                & \baro{v}   $\to$ \mbox{v}, \mbox{np}\\
\mbox{np}  $\to$ $\epsilon$                 & $\Rightarrow$  & \baro{v}   $\to$ \mbox{v}\\
\baro{v}   $\to$ \baro{v}, \mbox{adv}       &                & \baro{v}   $\to$ \baro{v}, \mbox{adv}\\
\mbox{adv} $\to$ $\epsilon$                 &                & \baro{v}   $\to$ \baro{v}\\
\end{tabular}

\pause
\bigskip

H[{\sc subcat} X]   $\to$ H[{\sc subcat} X $\oplus$ \sliste{Y}], Y\\
Y  $\to$ $\epsilon$

$\Rightarrow$


H[{\sc subcat} X]   $\to$ H[{\sc subcat} X $\oplus$ \sliste{Y}], Y\\
H[{\sc subcat} X]   $\to$ H[{\sc subcat} X $\oplus$ \sliste{Y}]

}

\subsection{Das \texorpdfstring{{\sc slash}}{SLASH}-Einführungsschema für Komplemente}

\frame{
\frametitle{{\sc slash}"=Einführungsschema für Komplemente}
\smallframe

%\resizebox{6cm}{!}{%
\type{head-comp-slash-phrase} $\to$\\
\onems{
   \visible<3->{loc$|$cat$|$subcat \highlight<3>{\ibox{1}}} \\
   \visible<4->{nonloc$|$slash \highlight<4>{\sliste{ \ibox{4} } $\oplus$ \ibox{5}}} \\
  head-dtr  \onems{
                           loc$|$cat$|$subcat \highlight<3>{\ibox{1}} $\oplus$ \sliste{ \highlight<2>{\onems{ loc \ibox{4} \\
                                                                                 nonloc$|$slash  \highlight<4>{\sliste{ \ibox{4} }} \\ %\\
                                                                               }}
                                                                       } \\
                           \visible<4->{nonloc$|$slash \highlight<4>{\ibox{5}}}\\
                   } \\
}
%}
%
\begin{itemize}[<+->]
\item Es gibt keine Nicht-Kopftochter. (Die würde durch Spur gefüllt)
\item Letztes Element der \subcatl der Kopftochter entspricht der Spur.
\item Restliche Argumente werden zur Mutter hochgegeben.
\item \slashw der Mutter ist \slasch der Kopftochter + \slasch der "`Spur"'.
\end{itemize}

}


\subsection{Lexikontransformation}

\frame{
\frametitle{Lexikontransformation}

\begin{tabular}[t]{@{}ll@{}}
\baro{v}   $\to$ \mbox{v-ditrans}, \mbox{np}, \mbox{np}, \mbox{np}\hspace{0.5cm} & \mbox{v-ditrans} $\to$ geben\\  
\baro{v}   $\to$ \mbox{v-trans},   \mbox{np}, \mbox{np}                        & \mbox{v-trans} $\to$ lieben\\  
\baro{v}   $\to$ \mbox{v-intrans}, \mbox{np}                                   & \mbox{v-intrans} $\to$ schlafen\\
\baro{v}   $\to$ \mbox{v-subjless}\\
\mbox{np}  $\to$ $\epsilon$\\   
\end{tabular}

$\Rightarrow$

\begin{tabular}[t]{@{}ll@{}}
\baro{v}   $\to$ \mbox{v-ditrans}, \mbox{np}, \mbox{np}, \mbox{np}\hspace{0.5cm} & \mbox{v-ditrans} $\to$ geben\\
\baro{v}   $\to$ \mbox{v-trans},   \mbox{np}, \mbox{np}                        & \mbox{v-trans} $\to$ lieben $\vee$ geben\\  
\baro{v}   $\to$ \mbox{v-intrans}, \mbox{np}                                   & \mbox{v-intrans} $\to$ schlafen $\vee$ lieben $\vee$ geben\\
\baro{v}   $\to$ \mbox{v-subjless}                                             & \mbox{v-subjless} $\to$ schlafen $\vee$ lieben $\vee$ geben \\
\end{tabular}


}

\frame{

\small
\frametitle{Lexikontransformation}


V[\subcat \liste{ NP$_1$, NP$_2$, NP$_3$ }] $\to$ geben\\
V[\subcat \liste{ NP$_1$, NP$_2$ }] $\to$ lieben\\
V[\subcat \liste{ NP$_1$ }] $\to$ schlafen\\

$\Rightarrow$

\begin{multicols}{2}
V[\subcat \liste{ NP$_1$, NP$_2$, NP$_3$ }] $\to$ geben\\
V[\subcat \liste{ NP$_1$, NP$_2$ }] $\to$ geben\\
V[\subcat \liste{ NP$_1$, NP$_3$ }] $\to$ geben\\
V[\subcat \liste{ NP$_2$, NP$_3$ }] $\to$ geben\\
V[\subcat \liste{ NP$_1$ }] $\to$ geben\\
V[\subcat \liste{ NP$_2$ }] $\to$ geben\\
V[\subcat \liste{ NP$_3$ }] $\to$ geben\\
V[\subcat \liste{ }] $\to$ geben\\
V[\subcat \liste{ NP$_1$, NP$_2$ }] $\to$ lieben\\
V[\subcat \liste{ NP$_1$ }] $\to$ lieben\\
V[\subcat \liste{ NP$_2$ }] $\to$ lieben\\
V[\subcat \liste{ }] $\to$ lieben\\~\\~\\
V[\subcat \liste{ NP$_1$ }] $\to$ schlafen\\
V[\subcat \liste{ }] $\to$ schlafen
\end{multicols}

}

\frame{
\frametitle{Argumentextraktionslexikonregel}

\hspace{\leftmargini}\onems[word]{
 loc$|$cat \onems{ \visible<3>{\highlight<3>{head$|$mod  \type{none}}} \\
                             subcat \ibox{1} $\oplus$ \liste{ \highlight<1>{\ms{
                                                                loc & \ibox{4}\\
                                                                nonloc$|$slash & \sliste{ \highlight<2>{\ibox{4}} } \\ % \\
                                                                   }} } $\oplus$ \ibox{3} \\
                           } \\
 nonloc$|$slash \sliste{} \\
 } $\to$ \\~\\
\hspace{\leftmargini}\onems[word]{
 loc$|$cat$|$subcat \ibox{1} $\oplus$ \ibox{3} \\
 nonloc$|$slash \sliste{ \highlight<2>{\ibox{4}} } \\
 }

\begin{itemize}[<+->]
\item Ein Argument wird mit "`Spur"' identifiziert.
\item \slasch der "`Spur"' wird zum \slashw des Ausgabezeichens.
\item Aus Adjunkten kann nicht extrahiert werden.
\end{itemize}

}

\frame{


\frametitle{Unterspezifikation im Lexikon}

\citet*{BMS2001a} und \citet{GSag2000a-u}:

\begin{itemize}
\item zwei Listen:
      \begin{itemize}
      \item Argumentstruktur
      \item abhängige Elemente
      \end{itemize}
\item Realisierungsbeschränkungen bilden die eine Liste auf die andere ab.\\
      "`Spuren"' werden nicht in die Liste der abhängigen Argumente aufgenommen.
\end{itemize}

}

\subsection{Zusammenfassung}

\frame{
\frametitle{Zusammenfassung}

\begin{itemize}[<+->]
\item An der Stelle der extrahierten Konstituente steht eine Spur.
\item Spur ist Joker: macht, was im entsprechenden lokalen Kontext gebraucht wird
\item Information wird nach oben über \slasch weitergegeben.
\item Abhängigkeit kann Satzgrenzen kreuzen.
\item Abhängigkeit durch Füller im Schema abgebunden.
\item Alternativen ohne leere Elemente
\end{itemize}

\pause\pause\pause

}


} % \end{teil1}

% definition für Folien: Maximal 999 Beispiele
\exewidth{\exnrfont (234)}

\if 0
\subsection{Extraktionsinseln}

\frame{

\frametitle{Extraktionsverbot / Extraktionsinseln (I)}

\begin{itemize}
\item nicht alle Komplemente/Adjunkte können extrahiert werden
\item Artikel
      \eal
      \ex[]{
      Kennt er den Mann?
      }
      \ex[*]{
      Den kennt er Mann?
      }
      \zl 
\item Sätze, die durch Komplementierer eingeleitet werden
\eal
\ex[]{
"`Wer, glaubt\iw{glauben} er, dass er ist?"' erregte sich ein Politiker vom Nil. (Spiegel, 8/1999)
}
\ex[*]{
{}[Er ein mächtiger Mann ist]$_i$, glaubt er, [dass \_$_i$].
}
\zl
\item \citet{Wegener85b}: ethische Dative (sind Adjunkte)
\eal
\ex[]{
Der fährt dir glatt an den Baum.
}
\ex[*]{
Dir fährt der glatt an den Baum.
}
\ex[*]{
Wem fährt er glatt an den Baum?
}
\zl
\end{itemize}

}
\frame{
\frametitle{Extraktionsverbot / Extraktionsinseln (II)}

\begin{itemize}
\item Reflexivpronomina inhärent reflexiver Verben
\eal
\ex[*]{
Sich\iw{sich|(} hat er gut erholt.\iw{erholen}
}
\ex[*]{
Dir bist du immer gleichgeblieben. \citep*{Hoberg81a}
}
\zl
keine Eigenschaft des Pronomens,\\da Reflexiva bei nicht inhärent refl.\ Verben voranstellbar:
\eal
\ex Sich sah er im Spiegel.
\ex Mit sich\iw{sich|)} bringt\iw{bringen!mit sich $\sim$} dieser Ansatz große Probleme.
\zl
\item bestimmte Adverbien sind nicht vorfeldfähig \citep{Hoberg81a}:
\eal
\ex[]{
Einfach$_{43}$\iw{einfach} geht das nicht! (adverbial gebrauchtes Satzadjektiv)
}
\ex[]{
Er ging einfach$_{18}$ nicht. (pragmatische\is{Pragmatik} Angabe)
}
\ex[*]{
Einfach$_{18}$ ging er nicht.
}
\zl

\end{itemize}
}

\subsection{Irregularitäten}

\frame{
\frametitle{Irregularitäten}


\begin{itemize}
\item idiosynkratisches Verhalten des Akkusativ"={\em es\/} (Kürze, Betonbarkeit):
\eal
\ex[*]{
Es (das Kind) liebt der Vater.
}
\ex[]{
Ihn (den Sohn) liebt der Vater.
}
\ex[]{
Seine Freundin liebt der Vater.
}
\zl
\item {\em es\/} als Prädikat \citep{Askedal90}:
\eal
\ex[]{
Er ist es (tüchtig).
}
\ex[*]{
Es (tüchtig) ist er.
}
\zl
%\item nicht betonbar
% \item Grund funktionale Überladung des {\em es\/} (positionales {\em es\/},
%       Korrelat-{\em es\/}, \ldots)
\end{itemize}

}

\frame[shrink]{

\frametitle{Sind Nominalphrasen Extraktionsinseln? (I)}

Extraktion aus Objekts-NPen
\eal
\ex {}[Über England]$_i$ hat ja wohl jeder schon mal [einen Film \_$_i$] gesehen.\iw{Film}\iw{sehen}
\ex {}[Von Maria]$_i$ hat Karl [ein Bild \_$_i$] gemalt.\iw{Bild}
\ex {}[Von dieser Puppe]$_i$ sucht Maria noch heute [den linken Arm \_$_i$].
\ex {}[Gegen diese Behauptung]$_i$ ist ihm [ein Argument \_$_i$] eingefallen.\iw{Argument}
\ex {}[Mit den geschundenen Kreaturen]$_i$ hat keiner [Mitleid \_$_i$].\iw{Mitleid}
\ex {}[An einer Auf"|klärung]$_i$ bestand [kein Interesse \_$_i$].\iw{Interesse} % mich interessiert X -> Interesse an ...
\ex {}[Von Stuttgart]$_i$ kenne ich nur [den Bahnhof \_$_i$].\iw{Bahnhof}
%\ex {}[Mit wem]$_i$ hast du [keine Lust [ \_$_i$ zu verreisen]]?\iw{Lust}  % Funktionsverbgefüge ??
\ex Stasi-Material, [an dem]$_i$ die Bundesregierung vorgibt, [ein Sicherheitsinteresse \_$_i$] zu haben, (taz, 14.01.95)
\ex {}[Über eine Polizistin in der US"=Kleinstadt High Point] regnete es letzte Woche [Beschwerden \_$_i$]: (taz, 25.01.99)
\zl

}

\frame{

\small
\frametitle{\small Sind Nominalphrasen Extraktionsinseln? (II)}

Extraktion aus NPen in Objekts-PPen
\eal
\ex
{}[Mit Norwegen]$_i$ befinden\iw{befinden} wir uns allerdings 
      [in [einem langfristigen Stellungskrieg\iw{Stellungskrieg}\iw{Krieg} \_$_i$]]. (Wochenpost 26/95)
\ex
{}[Für ihren aus Altersgründen ausgeschiedenen Bundestagsvize Burkhard Hirsch]$_i$ hat sie sich noch
        [auf [keinen Nachfolger \_$_i$]] einigen\iw{einigen auf} können. (taz, 01.09.98)
\zl

Extraktion oder nicht? \citet{Loetscher85}:
\eal
\ex Von diesen 109 Belegen fallen bei 57 (52\,\%) Topic, Thema und Subjekt zusammen. %van de Velde 1980
\ex Aus dem Barock sollten sie sich auf die wichtigsten Lyriker konzentrieren.
\ex Von den Nebensätzen ist mir für zwei Typen bisher keine systematische Beschreibung gelungen.
\zl
bezeichnet Sätze als Ausnahmen, 
PPen in ihrer textgrammatischen Funktion sehr nahe mit diskursbereichsdefinierenden
Angaben vom Typ \emph{was x betrifft}, \emph{x betreffend} verwandt

\ea[?]{
{}[Von Maria]$_i$ denke\iw{nachdenken} ich [über [Bilder\iw{Bild} \_$_i$]] nach.
}
\z


}


\frame{

\small
\frametitle{\small Sind Nominalphrasen Extraktionsinseln? (III)}

Extraktion aus Subjekts-NPen
\eal
\ex {}[Über Strauß]$_i$ hat [ein Witz \_$_i$] die Runde gemacht. \citep{Haider93a}
\ex {}[Zu drastischeren Maß\-nah\-men]$_i$ hat ihm [der Mut \_$_i$] gefehlt. \citep{Haider93a}
\ex {}[Zu diesem Problem]$_i$ haben uns noch [einige Briefe\iw{Brief} \_$_i$] erreicht. \citep{Oppenrieder91a}
\ex {}[Von den Gefangenen]$_i$ hatte eigentlich [keine \_$_i$ ] die Nacht der Bomben überleben sollen. (Bernhard Schink, {\em Der Vorleser\/})
\ex {}[Von der HVA]$_i$ hielten sich [etwa 120 Leute \_$_i$ ] dort in ihren Gebäuden auf. (Spiegel, 3/1999)
\ex {}[Von wem]$_i$ lag [ein Bild \_$_i$] auf dem Tisch. \citep{Suchsland97a}
%\ex Was$_i$ würde [mit ihm \_$_i$ zu besprechen sich denn noch lohnen?\footnote{
%        \citep[S.\,92]{Haider97a}\iafdata{Haider}
\zl
aus Subjekts-NP in Medialkonstruktion
\ea
{}[Von Kontrollverben]$_i$ lassen sich [Perfektformen \_$_i$] bilden, (Im Haupttext von \citep{Suchsland97a})
\z
aus Subjekts-NP eines Zustandspassiv
\ea
{}[Auch zum Schumann"=Platz, wo die EU"=Kommision sitzt,]$_i$ waren [die Zufahrtswege \_$_i$] versperrt. (taz, 18.09.2000)
\z

}

\frame{


\frametitle{Sind "`bewegte"' Phrasen Extraktionsinseln?}

\eal
\ex {}[Von wem]$_i$ hast du [Bilder \_$_i$] niemandem gezeigt?
\ex {}[Von wem]$_i$ hast du niemandem [Bilder \_$_i$] gezeigt?
\zl
\eal
\ex {}[Zum Gartenvereinsvorsitzenden]$_i$ hätte er [das Talent \_$_i$].\iw{Talent}
\ex {}[Zum Gartenvereinsvorsitzenden]$_i$ hätte [das Talent \_$_i$] wohl nur dieser Mann.\label{bsp-gartenvereinsvorsitzender}
\zl

\citet{Jacobs91a} behauptet: Extraktion nur aus verbadjazenten Elementen\\
aber Partitivaufspaltung:
\eal
\ex {}[Von diesen Männern]$_i$ habe ich nur [einem \_$_i$ ] die Hand gegeben.
\ex {}[Von den Gefangenen]$_i$ hatte eigentlich [keine \_$_i$ ] die Nacht der Bomben überleben sollen. (Bernhard Schink, {\em Der Vorleser\/})
\zl
genauso:
\ea
{}[Zum Gartenvereinsvorsitzenden]$_i$ hätte [das Talent \_$_i$] wohl nur dieser Mann gehabt.
\z

}

\frame{

\small
\frametitle{\small Extraktionsverbot und Nominalphrasen}

\eal
\ex[*]{
[Des Studenten]$_i$ hat [der Sohn \_$_i$] die Masern.
}
\ex[*]{
[Daß Rauchen schädlich ist]$_i$ hat Maria [das Argument \_$_i$] gekannt.
}
\zl

Relativsätze auch über Extraktion:
\ea
die Cousine, [von der]$_i$ [\sub{S} du [ein Bild \_$_i$] ins Photoalbum geklebt hast]?
\z

Zwischenstellung: von Nomina abhängige Infinitive
\eal
\ex[]{
ein Maler, der sie um jeden Preis heiraten will und [den auszuschlagen]$_i$ sie [die geradezu heupferdmäßige Dummheit \_$_i$] hat\iw{Dummheit}
}
\ex[]{
bis ins Schlafzimmer, [das abzuschließen]$_i$ ihr [der Wille \_$_i$] fehlte\label{bsp-bis-ins-schlafzimmer}\iw{Wille}
}
\ex[*]{
ein Angebot, das$_i$ sie [die geradezu heupferdmäßige Dummheit [~\_$_i$~auszuschlagen]] hat
}
\ex[*]{
bis ins Schlafzimmer, das$_i$ ihr [der Wille [ \_$_i$ abzuschließen]] fehlte
}
\zl

gesamter Infinitiv kann extrahiert werden, Teile des Infinitivs jedoch nicht

}

\frame{
\frametitle{Der pränominale Bereich und Adjunktinseln}

keine Extraktion aus dem pränominalen Bereich
\eal
\ex[*]{
{}[Ihre Mutter]$_i$ liebt Maria [den \_$_i$ achtenden Mann].\label{bsp-extraktion-achtenden}
}
\ex[*]{
{}[Den Mann] schläft [die \_$_i$ lieben wollende Frau].\label{bsp-lieben-wollende}
}
\ex[\#]{
{}Oft$_i$ schläft [die den Hund \_$_i$ schlagende Frau].\label{bsp-oft-schlagende}
}
\ex[*]{
{}[Schöne]$_i$ kennt Peter [eine \_$_i$ Frau].\label{bsp-extraktion-schoene}% PS94: 388
}
\zl

Adjunkte sind Inseln:

\eal
\ex[*]{
{}[Dem Tisch]$_i$ steht die Flasche [auf \_$_i$].
}
\ex[*]{
{}[Ihre Mutter]$_i$ liebt Maria den Mann, [der \_$_i$ achtet].\label{bsp-extraktion-aus-rs}
}
\ex[*]{
{}[Ein Lied]$_i$ kam Karl [ \_$_i$ singend] herein.
}
\ex[*]{
{}[Das Kind]$_i$ sind wir gekommen, [um \_$_i$ abzuholen].
}
\ex[*]{
{}[Das Kind]$_i$ sind wir gekommen [ \_$_i$ abzuholen].
}
\zl


}

\frame{
\frametitle{Zusammenfassung}



\begin{itemize}
\item Nominalphrasen sind keine Extraktionsinseln
\item Umstellung im \mf ist nicht ausschlaggebend
\item Adjazenz ist nicht ausschlaggebend
\item Informationsstruktur, Gliederung
\item Es gibt jedoch harte Bedingungen (Adjunkte, Genitive, u.s.w.)!
\end{itemize}



}

\frame{


\frametitle{Formale Umsetzung der Restriktionen}

\begin{itemize}
\item nicht"=extrahierbare Komplemente in Subcat"=Liste als {\sc slash} \liste{}
\item damit gleichzeitiger Ausschluß der Extraktion des Komplements und im Komplement tiefer eingebetteter Konstituenten
\item ansonsten nur disjunktive Spezifikation möglich:\\
      {}[{\sc loc} \ibox{1}, {\sc slash} \liste{ \ibox{1} }] $\vee$ [{\sc loc} \ibox{1}, {\sc slash} \liste{ }]
\item damit Extraktion aus Adjunkten nicht blockierbar (es sei denn Adjunkte in Subcat)
\item Blockade im Kopf"=Adjunkt"=Schema (Adjunkttochter = {\sc slash} \liste{ }) nicht möglich
\ea
Da$_i$ hat Karl [ \_$_i$ drin] geschlafen.\iw{drin}
\z
\item Spur sagt: das Element mit dem es kombiniert wird ist {\sc mod} {\it none\/}
\item gegenseitige Selektion $\to$ ?
\item Alternative: Restriktion im {\sc slash}"=Einführungsschema für Adjunkte
\end{itemize}

}

\fi
