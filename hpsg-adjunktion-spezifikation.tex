\section{Adjunktion und Spezifikation}

\subtitle{Adjunktion und Spezifikation}

\huberlintitlepage[22pt]



\outline{

\begin{itemize}
%\item Ziele
\item Wozu Syntax? / Phrasenstrukturgrammatiken
\item Formalismus
\item Valenz und Grammatikregeln
\item Komplementation
\item Semantik
\item \blau{Adjunktion und Spezifikation}
\item Das Lexikon: Typen und Lexikonregeln
\item Topologie des deutschen Satzes
\item Konstituentenreihenfolge
\item Nichtlokale Abhängigkeiten
\item Relativsätze
\item Lokalität
%\item Komplexe Prädikate: Der Verbalkomplex
\end{itemize}
}

\frame{
\frametitle{Literaturhinweis}
%
\begin{itemize}
\item Literatur: \citew[Kapitel~6.1--6.5]{MuellerLehrbuch3}
\end{itemize}

%\vspace{1cm}

%% \rotbf{Achtung, wichtiger Hinweis: Diese Literaturangabe hier bedeutet,\\dass Sie die Literatur zum
%%   nächsten Mal lesen sollen!!!!}
}

\frame{
\frametitle{Argumente und Adjunkte}
\pause
\begin{tabular}{@{\hspace{4mm}}ll}
Adjektive                     & ein {\em interessantes\/} Buch \\
Relativsätze\is{Relativsatz}  & der Mann, {\em den Maria liebt\/}, \\
                              & der Mann, {\em der Maria liebt\/}, \\
Adverbien\is{Adverb}          & Karl lacht {\em oft\/}. \\
\end{tabular}

\pause
\medskip

\begin{itemize}
\item Adjunkte füllen keine semantische Rolle
\item Adjunkte sind optional
\item Adjunkte sind iterierbar
\eal
\ex[*]{
Der Mann der Mann schläft.
}
\ex[]{
ein interessantes neues Buch
}
\zl
% \item Genitive?
% \eal
% \ex der Mantel des Versicherungsvertreters
% \ex der Mantel des Vaters des Versicherungsvertreters
% \zl
% \ea
% Ereignisse dieser Art der letzten Jahre \citep[S.\,257]{Heringer73a}
% \z
\end{itemize}

}

\subsection{Die Syntax von Kopf"=Adjunkt"=Strukturen}


\frame{

\frametitle{Adjunktion}

\begin{itemize}
\item Adjunkt selegiert Kopf via {\sc modified}
%\item Bedeutung des Kopfes wird in Bedeutung des Adjunktes integriert
%\item von dort projiziert
\ea
ein interessantes Buch
\z
~\\
\resizebox{5cm}{!}{
\ms{ 
   phon & \phonliste{ interessantes }\\
   cat & \ms{ head & \ms[adj]
                      { %prd & $-$ \\
                        \blau<1>{mod} &  \blau<2>{\rm $\overline{\mbox{\rm N}}$}\\
                      } \\
               subcat & \liste{} \\
             } \\
}
}
\pause
\item Adjektive selegieren eine fast vollständige Nominalprojektion.
\pause
\item Elemente, die nicht modifizieren, haben \modw{} {\it none\/}.
%% \item Alternative:\\
%%         Kopf enthält Beschreibung möglicher Adjunkte \citep{ps}\\
%%       wegen Iterierbarkeit nicht praktikabel \citep{ps2}
\end{itemize}

}

\frame{

\frametitle{Kopf"=Adjunkt"=Struktur (Selektion)}

~

\centerline{%
\begin{forest}
sm edges
[\nbar
  [{AP[{\sc head$|$mod} \ibox{1}]}
    [interessantes]]
  [\ibox{1} \nbar
    [Buch]]]
\end{forest}
}

}


\frame{

%\hspace{1cm}
Kopf-Adjunkt-Schema (vorläufige Version)\\
\label{ha-schema}
{\it head-adjunct-phrase} \impl\\
\ms{ 
head-dtr      & \blau<1>{\ibox{1}} \\[2mm]
non-head-dtrs & \liste{ \ms{ cat & \ms{ head$|$mod & \blau<1>{\ibox{1}} \\
                                        subcat   & \blau<2>{\liste{}} \\
                                      } \\
                           }}\\
}

\begin{itemize}
\item Der Wert des Selektionsmerkmals des Adjunkts (\ibox{1})\\
      wird mit der Kopftochter identifiziert.
\pause
\item Das Adjunkt muss gesättigt sein ({\sc subcat} \liste{}):
        \eal
        \ex[]{
        die Wurst in der Speisekammer
        }
        \ex[*]{
        die Wurst in
        }
        \zl
\end{itemize}
}

\frame{

\frametitle{Warum ist {\sc mod} ein Kopfmerkmal?}

\begin{itemize}[<+->]
\item Genauso wie Adjektive können Präpositionalphrasen modifizieren.
\item Adjunkte müssen gesättigt sein, damit sie modifizieren können.
\item Das Merkmal, das den zu modifizierenden Kopf selegiert,\\
      muss an der Maximalprojektion des Adjunkts vorhanden sein.
\item P + NP = PP, PP kann \nbar modifizieren.
\item {\sc mod} muss im Lexikon (P) und auf phrasaler Ebene (PP) vorhanden sein\\
        $\to$ Kopfmerkmal (als einfachste Lösung)
\end{itemize}

}

\frame{

\frametitle{Beispieleintrag für Präposition, die Nomen modifiziert}

\ea
die Wurst in der Speisekammer
\z

\ms{
phon & \phonliste{ in } \\
cat & \ms{ head & \ms[prep]{
                   mod &   {\rm $\overline{\mbox{\rm N}}$}\\
                   } \\
           subcat & \sliste{ NP[{\it dat}] } \\
         } \\
}

\pause

\ms{
phon & \phonliste{ in, der, Speisekammer } \\
cat & \ms{ head & \ms[prep]{
                   mod &   {\rm $\overline{\mbox{\rm N}}$}\\
                   } \\
           subcat & \liste{ } \\
         } \\
}

}

\subsection{Die Semantik in Kopf"=Adjunkt"=Strukturen}

\frame[shrink]{
\frametitle{Der Bedeutungsbeitrag in Kopf"=Adjunkt"=Struktur (I)}

~
\vfill
\centerline{%
\begin{forest}
sm edges
[\nbar
  [{AP[{\sc head$|$mod} \ibox{1}]}
    [interessantes]]
  [\ibox{1} \nbar
    [Buch]]]
\end{forest}
}

\begin{itemize}[<+->]
\item Woher kommt die Bedeutungsrepräsentation am Mutterknoten?
\item die Bedeutung von {\em Buch\/} steht fest: buch(X)
\item Möglichkeit: Teilbedeutungen beider Töchter einfach nach oben reichen
\item {\em interessantes\/} (interessant(X)) + {\em Buch\/} (buch(Y)) =
      interessant(X) \& buch(X)
\end{itemize}

\vfill
}

\frame{
\frametitle{Der Bedeutungsbeitrag in Kopf"=Adjunkt"=Struktur (II)}
\begin{itemize}
\item {\em interessantes\/} (interessant(X)) + {\em Buch\/} (buch(Y)) = interessant(X) \& buch(X)

aber:
      \ea
      der angebliche Mörder
      \z
      {\em angebliche\/} (angeblich(X)) + {\em Mörder\/} (mörder(Y)) $\neq$  angeblich(X) \& mörder(X)
\pause
\item Alternative: machen Bedeutung am Adjunkt fest:\\
      Im Lexikoneintrag für {\em interessantes\/} bzw.\ {\em angebliche\/} steht,\\
      wie der Bedeutungsbeitrag der Mutter aussehen wird\\[1mm]
      Bedeutung des modifizierten Kopfes wird im Lexikoneintrag\\
      des Modifikators in die Bedeutung des Modifikators integriert
\end{itemize}


}

\frame{

\frametitle{Kopf"=Adjunkt"=Struktur (Selektion und Bedeutungsbeitrag)}

\savespace
~\vspace{2ex}


\centerline{%
\small
% visible does not work with tikz
% \begin{forest}
% sm edges
% [{\nbar[\cont \rnode{cont1}{\ibox{1}}]}
%   [AP\feattab{\textsc{head$|$mod} \highlight<1>{\ibox{2}},\\
%               \visible<2->{\rnode{cont2}{\cont} \ibox{1} [\textsc{restr} \highlight<2>{\nliste{ interessant(\ibox{3}) } $\oplus$ \ibox{4}}]}} [interessantes]]
%   [{\highlightbox<1>{\ibox{2}} \nbar\visible<2->{[\textsc{cont$|$restr} \highlight<2>{\ibox{4} \nliste{ buch(\ibox{3}) }}]}}
%     [Buch]]]
% \end{forest}
\begin{forest}
sm edges
[{\nbar[\cont \rnode{cont1}{\ibox{1}}]}
  [AP\feattab{\textsc{head$|$mod} \highlight<1>{\ibox{2}},\\
              {\rnode{cont2}{\cont} \ibox{1} [\textsc{restr} \highlight<2>{\nliste{ interessant(\ibox{3}) } $\oplus$ \ibox{4}}]}} [interessantes]]
  [{\highlightbox<1>{\ibox{2}} \nbar{[\textsc{cont$|$restr} \highlight<2>{\ibox{4} \nliste{ buch(\ibox{3}) }}]}}
    [Buch]]]
\end{forest}
\visible<3->{\highlight<3>{\nccurve[angleA=90,angleB=175,ncurvB=1]{<->}{cont1}{cont2}}}
}

\begin{itemize}[<+->]
\item Kopf"=Adjunkt"=Schema identifiziert {\sc mod}"=Wert der Adjunkttochter mit dem Kopf (\ibox{2})
\item Modifikator hat gesamte Bedeutung unter {\sc cont}: \nliste{ interessant(\ibox{3}) } $\oplus$ \ibox{4}
\item semantischer Beitrag für die Phrase wird von dort projiziert (\ibox{1})
\end{itemize}


}


\frame{
\savespace
\footnotesize
\frametitle{Adjektiveintrag mit Bedeutungsrepräsentation}



\resizebox{5cm}{!}{
\ms
 { phon & \phonliste{ interessantes }\\
   cat & \ms{ head & \ms[adj]
                      { %prd & $-$ \\
                        mod &  \blau<1>{\rm $\overline{\mbox{\rm N}}$}\visible<2->{{\upshape :} \ms{ \visible<4->{ind   & \blau<4>{\ibox{1}}} \\
                                                                                               restr & \blau<3>{\ibox{2}} \\
                                                                                 }} \\
                      } \\
               subcat & \liste{} \\
             } \\
   \visible<3->{cont & \blau<5>{\ms{ \visible<4->{ind & \blau<4>{\ibox{1}} \ms{ per & 3 \\
                                 num & sg \\
                                 gen & neu \\
                               }} \\
                     restr & \liste{ \ms[interessant]{ 
                                theme & \blau<4>{\ibox{1}} \\ 
                               } 
                             } $\oplus$ \blau<3>{\ibox{2}}  \\
              }}} \\
}
}
\begin{itemize}
\item Adjektiv selegiert zu modifizierendes Nomen über {\sc mod} 
\pause{}
$\to$\\
      Adjektiv kann auf den \contw und damit auf die Restriktionen des Nomens (\ibox{2}) zugreifen \pause $\to$
      Adjektiv kann die Restriktionen des Nomens bei sich in {\sc restr} einbauen
\pause
\item Teilung des Indexes (\ibox{1}) sorgt dafür,\\dass Adjektiv und Nomen sich auf dasselbe Objekt beziehen
\pause
\item Bedeutungsbeitrag der gesamten Struktur wird vom Adjunkt projiziert
\end{itemize}

}

\frame{

\frametitle{Ergebnis der Kombination}



\mbox{{\it interessantes Buch\/}:}\\
\ms
{ cat & \ms{ head & noun \\
             subcat & \sliste{ det } \\
           } \\
  cont &  \ms
           { ind & \ibox{1} \ms{ per & 3 \\
                                 num & sg \\
                                 gen & neu \\
                               } \\
             restr & \liste{ \ms[interessant]{ theme & \ibox{1} \\ },
                             \ms[buch]{ inst & \ibox{1} \\ } 
                           } \\
           } \\
}


Bedeutung für {\em interessantes Buch\/} ist nicht in {\em Buch\/} sondern
in {\em interessantes\/} repräsentiert $\to$ Projektion des {\sc cont}"=Wertes vom Adjunkt


}

\frame{

\frametitle{Perkolation der Bedeutung in Kopf-Adjunkt-Strukturen}


\textit{head-adjunct-phrase} \impl\\
\ms{ 
cont & \ibox{1} \\[2mm]
non-head-dtrs & \liste{ \ms{ cont & \ibox{1} \\
                           }}\\
}


}

\frame{
\frametitle{Das gesamte Kopf-Adjunkt-Schema}


\textit{head-adjunct-phrase} \impl\\
\ms{ 
cont & \ibox{1} \\
head-dtr      & \ibox{2} \\[2mm]
non-head-dtrs & \liste{ \ms{ cat & \ms{ head$|$mod & \ibox{2} \\
                                        subcat   & \liste{} \\
                                      } \\
                             cont & \ibox{1} \\
                           }}\\
}

Typ \type{head-adjunct-phrase} mit allen Beschränkungen, die zum Typ gehören

}

\subsection{Prinzipien}

\subsubsection{Das Semantikprinzip}
\frame{
\frametitle{Das Semantikprinzip}

%\small

In Strukturen mit Kopf, die keine Kopf-Adjunkt-Strukturen sind,\\
ist der semantische Beitrag der Mutter identisch mit dem der Kopftochter.

\type{head-non-adjunct-phrase} \impl
\ms{
cont & \ibox{1} \\
head-dtr$|$cont & \ibox{1} \\
}
\medskip
\pause

In Kopf-Adjunkt-Strukturen ist der semantische Beitrag der Mutter
identisch mit dem der Adjunkttochter.

\type{head-adjunct-phrase} \impl
\ms{ 
cont & \ibox{1} \\[2mm]
non-head-dtrs & \liste{ \ms{ cont & \ibox{1} \\
                           }}\\
}

\bigskip
\pause
Strukturen mit Kopf ({\it headed-phrase\/}) sind entweder Untertypen von
{\em head-non-adjunct-phrase\/} oder von {\em head-adjunct-phrase\/}.

}


\subsubsection{Das Valenzprinzip}

\frame{
\frametitle{Argumentvererbung in Kopf-Adjunkt-Strukturen}


\begin{itemize}[<+->]
\item {\em Buch\/} hat gleiche Valenz wie {\em interessantes Buch\/}:\\
      Artikel muss noch gesättigt werden

\item Adjunktion verändert Valenz nicht $\to$\\
Valenzinformation der Mutter muss der der Kopftochter entsprechen.

\item formal:

\type{head-non-argument-phrase} \impl \onems{
      cat$|$subcat \ibox{1} \\
      head-dtr$|$cat$|$subcat \ibox{1}\\
      }

In Strukturen vom Typ {\it head-non-argument-phrase\/} werden
keine Komplemente gesättigt. Der \subcatw der Mutter ist identisch mit dem der Kopftochter.


%% \item Anmerkung:\\
%% {\it head-non-argument-phrase\/} und {\it head-argument-phrase\/} sind in der Typhierarchie
%% komplementär definiert.

%% D.\,h., alle Strukturen, die vom Typ {\it headed-phrase\/} sind und nicht vom Typ {\em head-argument-phrase\/},
%% sind vom Typ {\it head-non-argument-phrase\/}.
\end{itemize}

}


\frame{
\small
\frametitle{\normalsize Subkategorisierungsprinzip}




In Strukturen mit Kopf entspricht die Subcat-Liste des Mutterknotens der \subcatl der Kopftochter
minus den als Nicht-Kopftochter realisierten Argumenten.


\type{head-argument-phrase} \impl
\onems{
cat$|$subcat \ibox{1} \\[2mm]
head-dtr$|$cat$|$subcat \ibox{1} $\oplus$ \liste{ \ibox{2} } \\
non-head-dtrs  \liste{\ibox{2}} \\
} 

\pause
\type{head-non-argument-phrase} \impl
\onems{
      cat$|$subcat \ibox{1} \\
      head-dtr$|$cat$|$subcat \ibox{1}\\
      }

\pause
\bigskip

Strukturen mit Kopf ({\it headed-phrase\/}) sind entweder Untertypen von
{\em head-argument-phrase\/} oder von {\em head-non-argument-phrase\/}.

}


\frame{
\frametitle{Typhierarchie für {\it sign\/}}


\centerline{%
\begin{forest}
type hierarchy,
 for tree={
   calign=fixed angles,
   calign angle=60
 } 
[sign %,s sep+=4em
  [word]
  [phrase
    [non-headed-phrase]
    [headed-phrase, for tree={l sep+=\baselineskip}
      [head-non-adjunct-phrase, calign=first
        [head-argument-phrase]
        [\ldots]]
      [head-non-argument-phrase, calign=last
       [,identify=!r2212]
       [head-adjunct-phrase]]]]]
\end{forest}}

}


%% \frame{
%% \frametitle{Typhierarchie für {\it sign\/} (Überblick)}


%% \epsfxsize=0.9\textwidth
%% \centerline{\mbox{\epsffile{types-sign-hc-ha.eps}}}


%% \centerline{Betrachteter Ausschnitt {\bf fett} markiert}


%% }

\frame{
\frametitle{Kopf"=Adjunkt"=Struktur (HFP, Selektion, Semantik, \ldots)}

~\vspace{2ex}

\small

\hspace{2ex}\begin{forest}
sm edges
[\nbar\feattab{\head   \rnode{h1}{\ibox{1}},\\
               \subcat \rnode{sc1}{\ibox{2}},\\
               \rnode{cont1}{\cont} \ibox{3}}
  [AP\feattab{\textsc{head$|$mod} \ibox{4},\\
              \rnode{cont2}{\cont} \ibox{3} [{\sc restr} \nliste{ interessant(\ibox{5}) } $\oplus$ \ibox{6}]}
    [interessantes]]
  [\ibox{4} \nbar\feattab{\head   \rnode{h2}{\ibox{1}},\\
                          \subcat \rnode{sc2}{\ibox{2}} \sliste{ det },\\
                          {\sc cont$|$restr} \ibox{6} \nliste{ buch(\ibox{5}) }}
    [Buch]]]
\end{forest}
\visible<2->{\highlight<2>{\nccurve[angleA=0,angleB=45]{<->}{h1}{h2}}}
\visible<3->{\highlight<3>{\nccurve[angleA=0,angleB=45]{<->}{sc1}{sc2}}}
\visible<4->{\highlight<4>{\nccurve[angleA=170,angleB=175,ncurvB=1.2]{<->}{cont1}{cont2}}}
}


\subsection{Kapselnde Modifikation}

\frame{

{\footnotesize
\frametitle{Kapselnde Modifikation}

\ea
Jeder Soldat ist ein potentieller Mörder.
\z

\ea
$\ll m"order, instance:X \gg$  %;1
\z
\pause
\ea
$\ll potentiell, arg : ~\{ \ll m"order, instance:X \gg \} \gg$  % ; 1
\z

\pause
{\em mutmaßlich-}, {\em an\-geb\-lich-}, {\em potentiell-}  nach \citep*{ps2}:\\
%
\resizebox{5cm}{!}{\ms{
  cat & \ms{ head & \ms[adj]
                     { %prd & $-$ \\
                       mod & \baro{N}{\rm :} \ms{ ind & \ibox{1} \\
                                                  restr & \ibox{2} \\
                                                                        } \\
                     } \\
             subcat & \liste{ } \\
           } \\
  cont & \ms{ ind & \ibox{1} \\
                    restr & \liste{ \ms[mutmaßlich]{ 
                                      psoa-arg & \ibox{2} \\ 
                                      } 
                                  } \\
                   } \\
}
}\\

nur Annäherung, zu Einzelheiten siehe \citew{Kasper97a} bzw.\ \citew{Mueller99a}
}
}


% \frame{
% 
% \frametitle{Interpretation von Adjektiven}
% 


% \ea
% das rote Buch
% \z
% \(
% x_1|\sliste{ \ll buch, instance: x_1; 1 \gg, \ll rot, instance: x_1; 1 \gg }
% \)

% \citet*[Kapitel~8.3]{ps2}\ia{Pollard}: {\sc standard}

% \eal
% \ex Karl ist ein guter Arzt.
% \ex Das ist ein gutes Buch.
% \ex Frank möchte eine gute Zensur bekommen.
% \zl

% McConnell-Ginet:
% \ea
% Heute spielt die Mannschaft der Computerlinguistik-Abteilung gegen die Mannschaft
% der Informatiker Basketball. In unserer Mannschaft fehlt noch ein Spieler.
% Der Kapitän hat Peter beauftragt, noch einen guten Linguisten zu finden, der
% mitspielen will.
% \z

% {\sc standard} durch Kontext bereitgestellt
% 

% }


\begin{comment}
\frame{

\frametitle{Typen oder Implikationen?}

\begin{itemize}
\item Kodierung der Prinzipien im Typsystem: \citet{Krieger94a} und \citet{Sag97a}
\item alternativ: Implikationen mit evtl.\ komplexen Antezedentien
\item \citet{ps2}:\\
      (alte Notation mit expliziter Erwähnung der Töchter)
\begin{prinzip-break}[Kopfmerkmalprinzip]
$
\\
\begin{array}{@{}l}
 \ms{ dtrs & \ms[head\-ed-phrase]{} \\ } \Rightarrow \\ \\
 \ms{
   synsem$|$loc$|$cat$|$head & \ibox{1} \\
   dtrs$|$head-dtr$|$synsem$|$loc$|$cat$|$head & \ibox{1} \\
 }
\end{array}$
\end{prinzip-break}
\end{itemize}

}



\frame{


{\small
\begin{prinzip}[Subkategorisierungsprinzip~({\it Subcat~Principle\/})]
\label{prinzip-subcat}
$
\\
$
In einem phrasalen Zeichen, dessen {\sc dtrs}-Wert vom Typ {\it head\-ed-phrase\/} ist, ist
der {\sc sub\-cat}-Wert der Kopf"|tochter die Verknüpfung der Sub\-cat"=Liste des phrasalen Zeichens
mit der Liste der {\sc synsem}-Werte der Elemente in der {\sc comp-dtrs}"=Liste.
\end{prinzip}


Mit der Markmalsgeometrie von 1987:
$
\\
\\
\begin{array}{l}
 \ms
 { dtrs & \ms[head\-ed-phrase]{} \\
 } \Rightarrow \\
 \ms
 { synsem & \ms{ loc$|$cat$|$subcat & \ibox{1} \\ } \\
   dtrs & \ms{ head-dtr &  \ms{ synsem$|$loc$|$cat$|$subcat & \ibox{1} $\oplus$ \ibox{2} \\ } \\
               comp-dtrs & \ibox{2}  \\
             } \\
 }
\end{array}
\\
$
\begin{itemize}
\item Nicht-Kopf-Komplement-Strukturen sind: {\sc comp-dtrs} = $\liste{}$
\item $\to$ {\it head-non-argument-phrase\/} Untertyp von {\it headed-phrase\/}
\item $\to$ ähnliche Typen in beiden Ansätzen erforderlich
\end{itemize}
}

}

\end{comment}


\subsection{Spezifikator-Kopf-Strukturen}
\frame{

\frametitle{Das Spezifikatorprinzip -- Possessivkonstruktionen}


Nominalstrukturen: NP = Det, \nbar

\eal
\ex Karls Geschenk
\ex seine Frau
\zl
Kopfnomen füllt semantische Rolle in der Relation des Possesivums:\\
besitzen(karl, geschenk)

\pause
\begin{prinzip-break}[Spezifikatorprinzip ({\sc spec}-Principle)] 
\label{prinzip-spec}\is{Prinzip!Spezifikator-}
Wenn eine Tochter, die keine Kopf"|tochter ist,\\
in einer Kopf"|struktur einen von {\it none\/} verschiedenen {\sc spec}-Wert besitzt,\\
so ist dieser token-identisch mit
%dem {\sc synsem}-Wert 
der Kopf"|tochter.
\end{prinzip-break}


}

\frame{

\mbox{{\it seine\/}:} \\
\ms{
cat & \ms{ head & \ms[det]{
%                       cas & nom $\vee$ acc \\
%                       gen & fem \\
%                       dtype & 3 \\
                       spec & {\rm N\ind{1}} \\
                      } \\
            subcat & \liste{} \\
           } \\
cont & \ms{ ind & \ibox{2} \ms{ 
                                     per & 3 \\
                                     num & sg \\
                                     gen & mas $\vee$ neu \\
                                    } \\
                          restr & \liste{ \ms[besitzen]{ 
                                           location & \ibox{2} \\
                                           thema    & \ibox{1} \\
                                           }} \\
                        }  \\
%context$|$inds \liste{ \ibox{2} }  \\  % N'-Index fehlt
}

Der Index des Nomens in der NP (\iboxt{1}) ist über {\sc spec} erreichbar.

}

\subsection{Übungsaufgaben}


\frame{
\frametitle{Übungsaufgaben}

\begin{enumerate}
\item Wie sieht der Lexikoneintrag für das Adjektiv \emph{großem}, wie es in (\mex{1})
vorkommt, aus?
\eal
\ex mit großem Tamtam
\ex mit großem Eifer
\zl
\end{enumerate}

}
