%%%%%%%%%%%%%%%%%%%%%%%%%%%%%%%%%%%%%%%%%%%%%%%%%%%%%%%%%
%%   $RCSfile: hpsg-lexikon.tex,v $
%%  $Revision: 1.3 $
%%      $Date: 2007/05/27 13:27:12 $
%%     Author: Stefan Mueller (CL Uni-Bremen)
%%    Purpose: 
%%   Language: LaTeX
%%%%%%%%%%%%%%%%%%%%%%%%%%%%%%%%%%%%%%%%%%%%%%%%%%%%%%%%%

\subtitle{Das Lexikon}

\huberlintitlepage[22pt]


\section{Das Lexikon}

\outline{

\begin{itemize}
\item Ziele
\item Wozu Syntax? / Phrasenstrukturgrammatiken
\item Formalismus
\item Valenz und Grammatikregeln
\item Komplementation
\item Semantik
\item Adjunktion und Spezifikation
\item \blau{Das Lexikon: Typen und Lexikonregeln}
\item Topologie des deutschen Satzes
\item Konstituentenreihenfolge
\item Nichtlokale Abhängigkeiten
\item Relativsätze
\item Lokalität
%\item Komplexe Prädikate: Der Verbalkomplex
\end{itemize}
}

\frame{
\frametitle{Literaturhinweise}
%
\begin{itemize}
\item Literatur: \citew[Kapitel~7.1--7.4]{MuellerLehrbuch3}
\item Außerdem: Handbuchartikel \citew{DavisKoenig2021a}
\end{itemize}

\vspace{1cm}

%% \rotbf{Achtung, wichtiger Hinweis: Diese Literaturangabe hier bedeutet,\\dass Sie die Literatur zum
%%   nächsten Mal lesen sollen!!!!}
}

\frame{
\frametitle{Das Lexikon}


\begin{itemize}[<+->]
\item Lexikalisierung $\to$ enorme Reduktion der Anzahl der Dominanzschemata
\item Lexikoneinträge sehr komplex
\item Strukturierung und Klassifizierung $\to$\\ Erfassung von Generalisierungen \& Vermeidung von Redundanz
\item Typhierarchien und Lexikonregeln
\end{itemize}

}

\subsection{Vertikale Generalisisierungen: Typhierarchien}

\frame{

\frametitle{Die Komplexität eines Lexikoneintrags für ein Zählnomen}
\ms{
 phon & \phonliste{ \highlight{Frau} } \\
 cat  & \ms{ head & \ms[noun]{ } \\
             subcat & \liste{ det } \\
             \ldots & \ldots\\
           } \\
 cont & \ms{
           ind   & \ibox{1} \ms{ per & 3 \\
% kommt aus der Morphologie            num & sg \\
                                       gen & fem \\
                                     } \\
           restr & \liste{ \ms[\highlight{frau}]{ inst & \ibox{1} \\ }} \\
}\\
 \ldots & \ldots\\
}

nur kleiner Teil idiosynkratisch
}


\frame{

\frametitle{Zerlegung der Information}

\begin{itemize}
\item[a.] alle Nomina\\[2mm]
    \scalebox{0.8}{\onems{ cat$|$head \type{noun} \\
             cont {\it nom-obj\/} \\
        }}\medskip
\pause
\item[b.] alle referentiellen nichtpronominalen Nomina, die einen Determinator verlangen
    (zusätzlich zu a) \nopagebreak[4]\\[2mm]
    \scalebox{0.8}{\onems{ cat$|$subcat \liste{ det } \\
          cont  \ms{ ind & \ibox{1} \ms{ per & 3 \\ }\\
           restr & \liste{ \ms[psoa]{ inst & \ibox{1} \\ }, \ldots } \\
          } \\ 
        }}\medskip
\begin{samepage}
\pause
\item[c.] alle femininen Nomina (zusätzlich zu a)\\[2mm]
    \scalebox{0.8}{\ms{ cont$|$ind$|$gen & fem \\
        }}
\end{samepage}
\end{itemize}


}

\frame{

\frametitle{Die Komplexität eines Lexikoneintrags für ein Verb}

\mbox{{\it helf-\/} (Lexikoneintrag (Wurzel)):}\\
\ms{
phon & \phonliste{ \highlight{helf} }\\
cat  & \ms{ head   & verb \\
            subcat & \liste{ NP[{\it nom\/}]\ind{1}, NP[{\it dat}]\ind{2}  } \\
          } \\
cont &  \ms[\highlight{helfen}]{ agens       & \ibox{1} \\
                                 experiencer & \ibox{2} \\
                               } \\
}


}

\frame{

\frametitle{\normalsize Zerlegung der Information}

\begin{itemize}
\item[a.] alle Verben \\[2mm]
            \scalebox{0.8}{\onems{ cat$|$head \type{verb} \\
                                   cont       \type{psoa} \\
           }}
\pause
\item[b.] bivalente Verben mit Dativobjekt 
           (zusätzlich zu a) \\[2mm]
           \scalebox{0.8}{\onems{ cat$|$subcat \liste{ NP[{\it nom\/}], NP[{\it dat\/}] } \\
           }}
\pause
\item[c.] \begin{minipage}[t]{\linewidth}alle bivalenten Verben mit {\sc agens} und {\sc experiencer} \\ 
           (zusätzlich zu a) \\[2mm]
           \scalebox{0.8}{\ms{ cat$|$subcat & \liste{ \onems{ cont$|$ind \ibox{1}\\}, \onems{ cont$|$ind \ibox{2}\\}
                                          } \\ 
                cont & \ms[agens-exp-rel]{
                         agens & \ibox{1} \\
                         experiencer & \ibox{2} \\
                        } \\
           }}\end{minipage}
\end{itemize}
%
}

% \frame{
% \frametitle{Partitionen}

% \(
% \ms[sign]{}~ =~ \ms[word]{}~ \vee ~\ms[phrase]{} \\[2mm]
% \ms[word]{}~ \wedge~ \ms[phrase]{}~ = \bot
% \)
% %

% %
% \begin{figure}[htbp]
% \setlength{\unitlength}{0.01in}%
% %\begin{center}
% \centering
% \begin{picture}(230,100)
% %\put(  0,  0){\framebox(230,100){}}
% \put( 95,70){\line(-4,-3){ 65.600}}
% \put( 95,70){\line( 5,-3){ 84.559}}
% \put(  0, 0){\makebox(0,0)[lb]{{\it word}}}
% \put(160, 0){\makebox(0,0)[lb]{{\it phrase}}}
% \put( 82,80){\makebox(0,0)[lb]{{\it sign}}}
% \end{picture}
% %\end{center}
% \caption{\label{abb-sign}%
% Partition des Typs {\it sign}}
% \end{figure}
% %

% }

\frame[shrink]{

\frametitle{Auszug aus einer möglichen Typhierarchie}


% \hfill\resizebox{0.8\linewidth}{!}{%
% \begin{tabular}{@{}c@{\hspace{3mm}}c@{\hspace{3mm}}c@{\hspace{3mm}}c@{\hspace{3mm}}c@{\hspace{3mm}}c@{\hspace{3mm}}c@{\hspace{3mm}}c@{\hspace{3mm}}c@{\hspace{3mm}}c@{\hspace{3mm}}c@{\hspace{3mm}}c@{\hspace{3mm}}c@{\hspace{3mm}}c@{\hspace{3mm}}c@{\hspace{3mm}}c@{\hspace{3mm}}c@{\hspace{3mm}}c@{\hspace{3mm}}c@{\hspace{3mm}}c@{\hspace{3mm}}c@{\hspace{3mm}}c@{\hspace{3mm}}c@{\hspace{3mm}}c@{\hspace{3mm}}c@{\hspace{3mm}}c@{\hspace{3mm}}c@{\hspace{3mm}}c@{\hspace{3mm}}c@{}}
% \multicolumn{8}{c}{\rnode{lo}{\it sign}}\\[18ex]
% \rnode{root}{\it root} & \rnode{verb}{\it verb-sign} & \rnode{noun}{\it noun-sign} &
% \rnode{saturated}{\it saturated} & \multicolumn{3}{c}{\rnode{unsaturated}{\it unsaturated}} &  
% \multicolumn{3}{c}{\rnode{nominal}{\it nominal-sem-sign}} & & \multicolumn{2}{c}{\rnode{verbal}{\it verbal-sem-sign}}\\
%                       & \rnode{verb-c}{[ {\sc head} {\it verb} ]} & \rnode{noun-c}{[ {\sc head} {\it noun} ]} &
% [ {\sc subcat} \sliste{} ] & \multicolumn{3}{c}{\rnode{unsaturated-c}{[ {\sc subcat} \sliste{ { [], \ldots} } ]}} & \multicolumn{3}{c}{\rnode{nominal-c}{[ {\sc cont} {\it nom-obj} ]}} & & \multicolumn{2}{c}{\rnode{verbal-c}{[ {\sc cont} {\it psoa} ]}}\\[6ex]
% &&&\multicolumn{2}{c}{\rnode{det-sc}{\it det-sc}} & \rnode{trans}{\it nom-arg} & \rnode{ditrans}{\it nom-dat-arg} &&
% \rnode{1}{\it 1} & \rnode{2}{\it 2} & \rnode{3}{\it 3} & \rnode{agent}{\it agent} & \rnode{exp}{\it experiencer}\\
% &&&\multicolumn{2}{c}{\rnode{det-sc-c}{[{\sc subcat} \sliste{ { det }} ]}}&\rnode{acc-o-c}{[ {\sc subcat} \sliste{ { NP[{\it nom}]} } ]} &\rnode{dat-o-c}{[ {\sc subcat} \sliste{ { NP[{\it nom}], NP[{\it dat}]} } ]}&&&& \rnode{3-c}{[{\sc ind$|$per} 3]} & \rnode{agent-c}{[{\sc cont} agent-rel]} & \rnode{exp-c}{[{\sc cont} exp-rel]}\\[6ex]
% &&&&&&&&&&&\multicolumn{2}{c}{\rnode{ag-exp}{agent-exp}}\\
% &&&&& & &&&&\rnode{3-aux-n}{}&\multicolumn{2}{c}{\rnode{ag-exp-aux-n}{}}\\[24ex]
% &\rnode{tr-dat-v-root}{\it trans-dat-verb-root~} & \rnode{cn-root}{\it count-noun-root}\\[12ex]
% &\rnode{helf}{helf-}&\rnode{frau}{Frau-}\\
% \end{tabular}
% \ncline{lo}{root}%
% \ncline{lo}{noun}\ncline{lo}{verb}%
% \ncline{lo}{saturated}\ncline{lo}{unsaturated}%
% \ncline{lo}{nominal}\ncline{lo}{verbal}%
% \ncline{unsaturated-c}{det-sc}\ncline{unsaturated-c}{trans}\ncline{unsaturated-c}{ditrans}%
% \ncline{nominal-c}{1}\ncline{nominal-c}{2}\ncline{nominal-c}{3}%
% \ncline{verbal-c}{agent}\ncline{verbal-c}{exp}%
% \ncline{root}{cn-root}\ncline{noun-c}{cn-root}\ncline{det-sc-c}{cn-root}%
% % todo \ncline{3-c}[b]{3-aux-n}\ncline{3-aux-n}{cn-root}%
% \ncline{3-c}{3-aux-n}\ncline{3-aux-n}{cn-root}%
% \ncline{agent-c}{ag-exp}\ncline{exp-c}{ag-exp}%
% %todo \ncline{root}{tr-dat-v-root}\ncline{verb-c}{tr-dat-v-root}\ncline{dat-o-c}{tr-dat-v-root}\ncline{ag-exp}[b]{ag-exp-aux-n}\ncline{ag-exp-aux-n}{tr-dat-v-root}%
% \ncline{root}{tr-dat-v-root}\ncline{verb-c}{tr-dat-v-root}\ncline{dat-o-c}{tr-dat-v-root}\ncline{ag-exp}{ag-exp-aux-n}\ncline{ag-exp-aux-n}{tr-dat-v-root}%
% {\makedash{4pt}%
% \ncline{tr-dat-v-root}{helf}\ncline{cn-root}{frau}}%
% }\hfill\hfill\mbox{}

\begin{itemize}[<+->]
\item bei Merkmalsspezifikationen entsprechende Pfade dazudenken:\\
      {}[ {\sc subcat} \liste{}] steht für [{\sc cat$|$subcat} \liste{} ]
\item Beschränkungen für Typen gelten auch für Untertypen (Vererbung)
\item Instanzen mit Strichlinie verbunden
\end{itemize}
}


\frame{
\frametitle{Beispiele für Lexikoneinträge}



\onems[count-noun-root]{
 phon   \phonliste{ Frau } \\
cont$|$restr  \liste{ \ms[frau]{}} \\
}


\ms[trans-dat-verb-root]{
phon & \phonliste{ helf } \\
cont & helfen \\
}



}

\subsection{Horizontale Generalisierungen: Lexikonregeln}

\frame{

\frametitle{Horizontale und vertikale Generalisierungen}

\begin{itemize}
\item In Typhierarchien werden linguistische Objekte kreuzklassifiziert (Lexikoneinträge, Schemata).
\pause
\item Wir drücken Generalisierungen über Klassen von linguistischen Objekten aus.
\pause
\item Wir können sagen, was bestimmte Wörter gemeinsam haben.
      \begin{itemize}
      \item {\em Frau\/} und {\em Mann\/}
      \item {\em Frau\/} und {\em Salz\/}
      \item {\em Frau\/} und {\em Plan\/}
      \end{itemize}
\pause
\item Aber es gibt andere Regularitäten:
      \begin{itemize}
      \item {\em treten\/} und {\em getreten\/} wie in {\em wird getreten\/}
      \item {\em lieben\/} und {\em geliebt\/} wie in {\em wird geliebt\/}
      \end{itemize}
\pause
\item Die Wörter könnten ebenfalls in der Hierarchie repräsentiert werden\\ (als Untertypen von intransitiv und transitiv),
      aber dann wäre nicht erfaßt,\\ dass die Valenzänderung durch denselben Prozeß ausgelöst wird.
\end{itemize}
}

\frame{
\frametitle{Lexikonregeln}

%\savespace
\begin{itemize}
\item Statt dessen: Lexikonregeln\\
\citet{Jackendoff75a}, \citet*{Williams81a}, \citet{Bresnan82a},\\
\citet*{SURT83a}, \\
\citet*{FPW85a}, \citet{Flickinger87},\\
\citet{CB92a}, \citet{Meurers2000b}
\pause
\item Beispiel Passiv:
      Eine Lexikonregel setzt die Beschreibung eines Stamms zur Beschreibung einer Passivform in Beziehung.
\pause
\item verschiedene Interpretationen der Bedeutung von Lexikonregeln:\\
      Meta Level Lexical Rules (MLR) vs.\\
Description Level Lexical Rules (DLR)\\
      Eine detailierte Diskussion findet man bei \citew{Meurers2000b}.
\end{itemize}
}



\frame{

\frametitle{Lexikonregel für Passiv in MLR-Notation}


\mbox{Lexikonregel für persönliches Passiv nach \citet{Kiss92}:} \\
\onems[stem]{
           cat~ \ms{ head & verb \\
                     subcat & \liste{ NP[{\it nom\/}], NP[{\it acc\/}]$_{\ibox{1}}$ } $\oplus$ \ibox{2} \\
                   } \\
}   $\mapsto$ \\
\onems[word]{
  cat \ms{ head & \ms{ vform & passiv-part \\
                              } \\
           subcat & \liste{ NP[{\it nom\/}]$_{\ibox{1}}$ } $\oplus$ \ibox{2} \\
         } \\
}

\eal
\ex Der Mann schlägt den Hund.
\ex Der Hund wird geschlagen.
\zl

}


\frame{%[shrink=15]{
\frametitle{Konventionen für die Bedeutung von Lexikonregeln}

\begin{itemize}
\item Alle Information, die im Ausgabezeichen nicht erwähnt wird,\\
      wird vom Eingabezeichen übernommen.
\item Beispiel: Passiv ist bedeutungserhaltend.\\
      Die {\sc cont}"=Werte von Ein- und Ausgabe sind identisch.\\

      Linking-Information bleibt erhalten:\\
\medskip

\begin{tabular}{@{}ll@{}}
Aktiv: & Passiv:\\
      \scalebox{0.8}{\ms{
      cat & \onems{ subcat      \liste{ NP[{\it nom\/}]\ind{1}, NP[{\it acc\/}]\ind{2} } \\
             }\\
      cont & \ms[schlagen]{
            agens       & \ibox{1}\\
            patients & \ibox{2}\\
            }\\ 
      }} & \scalebox{0.8}{\ms{
      cat  & \onems{ subcat      \liste{ NP[{\it nom\/}]\ind{2} }
             }\\
      cont & \ms[schlagen]{
            agens       & \ibox{1}\\
            patients & \ibox{2}\\
            }\\ 
      }}\\
\end{tabular}
\end{itemize}

}


\frame{
\frametitle{Lexikonregel für das persönliche Passiv in DLR"=Notation}

\savespace
\onems[acc-passive-lexical-rule]{
     cat \onems{ head$|$vform \type{passiv-part} \\
              subcat  \liste{ NP[{\it nom\/}]$_{\ibox{1}}$ } $\oplus$ \ibox{2} \\
            } \\
lex-dtr \onems[stem]{
        cat~ \ms{ head & verb\\
                  subcat & \liste{ NP[{\it nom\/}], NP[{\it acc\/}]$_{\ibox{1}}$ } $\oplus$ \ibox{2} \\
                } \\
     }\\
}

\begin{itemize}
\item wie unäre Regel mit Mutter und Tochter, jedoch auf Lexikon beschränkt
\pause
\item {\it word\/} $\succ$ {\it acc-passive-lexical-rule\/}
\pause
\item Da LRs getypt sind, sind Generalisierungen über Lexikonregeln möglich.
\pause
\item DLRs sind vollständig in den Formalismus integriert
\end{itemize}

}



\frame{
\frametitle{Lexikonregel für das persönliche Passiv mit Morphologie}

\savespace
\scalebox{1}{
\onems[acc-passive-lexical-rule]{
phon $f(\ibox{1})$\\
cat \onems{ head$|$vform \type{passiv-part} \\
         subcat \liste{ NP[{\it nom\/}]$_{\ibox{2}}$ } $\oplus$ \ibox{3} \\
        } \\
lex-dtr \onems[stem]{
           phon  \ibox{1}\\
           cat$|$subcat  \liste{ NP[{\it nom\/}], NP[{\it acc\/}]$_{\ibox{2}}$ } $\oplus$ \ibox{3} \\
     }\\
}
}

\begin{itemize}
\item $f$ ist Funktion, die für den {\sc phon}"=Wert der {\sc lex-dtr} die Partizipform liefert\\ ({\em red\/} $\to$ {\em geredet\/})
\pause
\item alternativ Kopf-Affix-Strukturen (ähnlich zu binär verzweigenden syntaktischen Strukturen)
\end{itemize}

}

\subsection{Kopf-Affix-Strukturen vs.\ Lexikonregeln}
\frame{
\frametitle{Kopf-Affix-Strukturen vs.\ Lexikonregeln}

\begin{itemize}
\item Description"=Level Lexikonregeln\\
      (\citealp{Orgun96a}; 
\citealp{Riehemann98a}; \citealp{AW98a};
\citealp{Koenig99a}; \citealp{Mueller2002b})\nocite{Dowty79a}
\pause
\item Kopf-Affix-Ansätze\\
      (\citealp{KN93a}\iaright{Krieger}\iaright{Nerbonne};
\citealp{Krieger94a}\ia{Krieger}; \citealp{Eynde94}\ia{van Eynde};\\\citealp{Lebeth94}\ia{Lebeth})
\bigskip
\pause
\item In vielen Fällen sind die Ansätze ineinander überführbar \citep{Mueller2002b}.
\pause
\item Manchmal als Vorteil betrachtet, dass man bei Ansatz mit Lexikonregeln ohne hunderte von leeren Affixen
      für Nullflexion und Konversion auskommt.
\pause
\item Morpheme, die Stämme verkürzen, werden bei Verwendung von LR nicht gebraucht.
\end{itemize}

}









\subsection{Übungsaufgaben}


\frame[shrink=15]{
\frametitle{Übungsaufgaben}

\begin{enumerate}
\item Schreiben Sie eine Lexikonregel, die für Adjektivstämme wie den
\stem{reif} einen Lexikoneintrag für die attributive Verwendung (\emph{reifes})
lizenziert.

\medskip
\oneline{%
\ms{
cat & \ms{ head   & \ms[adj]{
                    mod & none\\
                    }\\
           subcat & \liste{ NP\ind{1} }\\
         }\\
cont & \ms[reif]{
       thema & \ibox{1}\\
       }\\
}\hspace{2ex}%
\ms{
cat & \ms{ head   & \ms[adj]{
                    cas & nom $\vee$ acc\\
                    mod & {\rm $\overline{\mbox{\rm N}}$:} \ms{ ind   & \ibox{1} \\
                                                                restr & \ibox{2} \\
                                                              } \\
                    }\\
           subcat & \liste{ }\\
         }\\
cont & \ms{ ind & \ibox{1} \ms{ per & 3 \\
                                num & sg\\
                                gen & neu\\
                              } \\
            restr & \liste{ \ms[reif]{ 
                             theme & \ibox{1} \\ 
                          }} $\oplus$ \ibox{2}  \\
              } \\
}}

\medskip
Die \phonwe können dabei unberücksichtigt bleiben. Wichtig ist, dass die
Regel für alle Adjektivstämme funktioniert, also \zb auch für \stem{groß}/""\emph{großem}.
\end{enumerate}

}

