%%%%%%%%%%%%%%%%%%%%%%%%%%%%%%%%%%%%%%%%%%%%%%%%%%%%%%%%%
%%   $RCSfile: hpsg-valenz-grammatikregeln.tex,v $
%%  $Revision: 1.3 $
%%      $Date: 2007/05/27 13:27:12 $
%%     Author: Stefan Mueller (CL Uni-Bremen)
%%    Purpose: 
%%   Language: LaTeX
%%%%%%%%%%%%%%%%%%%%%%%%%%%%%%%%%%%%%%%%%%%%%%%%%%%%%%%%%

\section{Valenz und Grammatikregeln}

\subtitle{Valenz, Grammatikregeln und Komplementation}

\huberlintitlepage[22pt]


\iftoggle{teil1}{
\outline{

\begin{itemize}
%\item Ziele
\item Wozu Syntax? / Phrasenstrukturgrammatiken
\item Formalismus
\item \blau{Valenz und Grammatikregeln}
\item Dominanzstrukturen und Prinzipien
\item Semantik
\item Adjunktion und Spezifikation
\item Das Lexikon: Typen und Lexikonregeln
\item Topologie des deutschen Satzes
\item Konstituentenreihenfolge
\item Nichtlokale Abhängigkeiten
\item Relativsätze
\item Lokalität
%\item Komplexe Prädikate: Der Verbalkomplex
\end{itemize}
}
} % \end{teil1}

\iftoggle{teil2}{
\outline{
\begin{itemize}
\item Wiederholung
      \begin{itemize}
\item Wozu Syntax? / Phrasenstrukturgrammatiken
\item Formalismus
\item \blau{Valenz und Grammatikregeln}
\item Dominanzstrukturen und Prinzipien
\item Semantik
\item Topologie des deutschen Satzes
\item Konstituentenreihenfolge
\item Nichtlokale Abhängigkeiten
\end{itemize}
\item Kongruenz
\item Kasus
\item Der Verbalkomplex
\item Kohärenz, Inkohärenz, Anhebung und Kontrolle
\item Passiv
\item Partikelverben
\item Morphologie
\end{itemize}
}
} % \end{teil2}


\frame{
\frametitle{Literaturhinweise}
%
\begin{itemize}
\item Literatur: \citew[Kapitel~3]{MuellerLehrbuch4}
\end{itemize}


%% \rotbf{Achtung, wichtiger Hinweis: Diese Literaturangabe hier bedeutet,\\dass Sie die Literatur zum
%%   nächsten Mal lesen sollen!!!!}
}

\iftoggle{hpsglight}{}{%
\frame{
\frametitle{Valenz in der Chemie und in der Linguistik}

\vfill

\centerline{
\begin{forest}
[O
  [H] 
  [H] ]
\end{forest}
\hspace{5em}
\begin{forest}
[helfen
 [Aicke]
 [Conny] ]
\end{forest}
}


\vfill

}

\frame{
\frametitle{Valenz und Grammatikregeln: PSG}


\begin{itemize}
\item große Anzahl von Regeln:\\
      \begin{tabular}[t]{l@{~$\to$~}l@{\hspace{6em}}l}
      S &  NP, V               & \emph{X schläft}\\
      S &  NP, NP, V           & \emph{X Y liebt}\\
      S &  NP, PP[\type{über}], V           & \emph{X über Y spricht}\\
      S &  NP, NP, NP, V       & \emph{X Y Z gibt}\\
      S &  NP, NP, PP[\type{mit}], V       & \emph{X Y mit Z dient}\\
      \end{tabular}
\pause
\item Verben müssen mit passender Regel verwendet werden.
\pause
\item Valenz doppelt kodiert: In Regeln und in Lexikoneinträgen.
\end{itemize}
}

\frame{
\frametitle{Phrasale vs.\ lexikalische Ansätze}

\begin{itemize}
\item Phrasale Ansätze der 70er und 80er wurden für lexikon-orientierte Ansätze aufgegeben.\\
  \parencites{Jacobson87b}[Section~5.5]{MuellerGT-Eng1}{MWArgSt}
\pause
\item Gründe: 
\begin{itemize}
\item Voranstellung von Teilphrasen (Partial VP Fronting)\\ \citep{Nerbonne86a,Johnson86a} 
\pause
\item Interaktionen mit Morphologie \citep[Section~5.5.1]{MuellerGT-Eng1}
\end{itemize}
\pause
\item Come Back der phrasalen Ansätze in Construction Grammar \citep{Goldberg95a}, diese
  funktionieren aber nicht.

\parencites{Mueller2006d,MuellerPersian,MuellerUnifying,MWArgSt,MWArgStReply,MuellerFCG,MuellerLFGphrasal,MuellerPotentialStructure,MuellerGT-Eng4,MuellerCxG}


\end{itemize}

}

\frame{

\frametitle{Valenz und Grammatikregeln: HPSG}

\begin{itemize}
\item Argumente als komplexe Kategorien in der lexikalischen Repräsentation
      eines Kopfes repräsentiert\\
      (wie Kategorialgrammatik)
\pause
\item \begin{tabular}[t]{@{}lll}
      Verb             & \comps\\
      \emph{schlafen} & \sliste{ NP }\\
      \emph{lieben}   & \sliste{ NP, NP }\\
      \emph{sprechen} & \sliste{ NP, PP[\type{über}] }\\
      \emph{geben}    & \sliste{ NP, NP, NP }\\
      \emph{dienen}   & \sliste{ NP, NP, PP[\type{mit}] }\\  
      \end{tabular}
\end{itemize}
}

\frame{
\frametitle{Beispielstruktur mit Valenzinformation (I)}

\vfill
\hfill\begin{forest}
sm edges
[{V[\comps \sliste{}]}
  [{\ibox{1} NP[\type{nom}]}
    [Aicke]]
  [{V[\comps \sliste{ \ibox{1} }]}
    [schläft]]]
\end{forest}\hfill\hfill\mbox{}
\vfill
V[\comps \sliste{ }] entspricht hierbei einer vollständigen Phrase\\
(VP oder auch S)
\vfill
}

\frame{
\frametitle{Beispielstruktur mit Valenzinformation (II)}

\vfill
\centerline{%
\begin{forest}
sm edges
[{V[\comps \sliste{}]},visible on=4-
  [{\ibox{1} NP[\type{nom}]},visible on=3-
    [Aicke,visible on=3-]]
  [{V[\comps \sliste{ \ibox{1} }] },visible on=2-
    [{\ibox{2} NP[\type{acc}]} 
      [Conny]]
    [{V[\comps \sliste{ \ibox{1}, \ibox{2} }] }
      [erwartet]]]]
\end{forest}}
\vfill

}


\frame{
\frametitle{Beschränkungsbasierte Theorien und Psycholinguistik}

\centerfit{
\begin{forest}
sm edges
[{V[\comps \sliste{}]},visible on=4-
  [{NP[\type{nom}]},visible on=3- 
    [Aicke,visible on=3-] ]
  [{V[\comps \sliste{ NP[\type{nom}] }] },visible on=4-
    [{NP[\type{acc}]},visible on=5- 
      [Conny,visible on=5-] ]
    [{V[\comps \sliste{ NP[\type{nom}], NP[\type{acc}] } ] },visible on=6- [erwartet,visible on=7-]] ] ]
\end{forest}}

\begin{itemize}
\item Erklärungen im Folgenden immer von unten nach oben.
\pause
\item Das ist in Theorien wie HPSG aber nicht zwingend. 

Sehr wichtig aus psycholinguistischer Sicht, denn Verarbeitung ist inkrementell.\\
\parencites{Marslen-Wilson75a,TSKES96a,SW2011a,Wasow2021a}
\pause

\end{itemize}




}


\frame{

\frametitle{Valenz und Grammatikregeln: HPSG}

\begin{itemize}
\item spezifische Regeln für Kopf-Argument-Kombination:\\
      \begin{tabular}[t]{@{}lll}
      V[COMPS \ibox{A}] & $\to$ & \ibox{B}\hspace{2em} V[COMPS \ibox{A} $\oplus$ \sliste{ \ibox{B} } ]\\
      \end{tabular}
\pause
\item Dabei ist $\oplus$ eine Relation zur Verknüpfung zweier Listen:\\
      \begin{tabular}{@{}l@{~}l@{}}
      \phonliste{ a, b } =& \phonliste{ a } $\oplus$ \phonliste{ b } oder\\
                    & \phonliste{} $\oplus$ \phonliste{ a, b } oder\\
                    & \phonliste{ a, b } $\oplus$ \phonliste{}\\
      \end{tabular}
\end{itemize}
}

\frame{
\frametitle{Valenz und Grammatikregeln (II)}

\centerline{
\begin{forest}
sm edges, for tree={l sep+=\baselineskip}
[{V[\comps \sliste{}]}
  [\ibox{1} NP
    [Aicke]]
  [{V[\comps \sliste{ \ibox{1} }]},name=v2
    [\ibox{2} NP 
      [Conny]]
    [{V[\comps \sliste{ \ibox{1}, \ibox{2} }]},name=v3
      [erwartet]]]]
\node()[above right of=v2,xshift=12em,yshift=0\baselineskip]{\small \begin{tabular}{@{}l}
                                    V[COMPS \ibox{A}] $\to$ \ibox{B} V[COMPS \ibox{A} $\oplus$ \sliste{ \ibox{B} } ] \\ 
                                    A = \sliste{ }, B = \ibox{1}
                                    \end{tabular}};
\node()[above right of=v3,xshift=12em,yshift=0\baselineskip]{\small \begin{tabular}{@{}l}
                                    V[COMPS \ibox{A}] $\to$ \ibox{B} V[COMPS \ibox{A} $\oplus$ \sliste{ \ibox{B} } ] \\ 
                                    A = \sliste{ \ibox{1} }, B = \ibox{2}
                                    \end{tabular}};
\end{forest}}

}


\frame{
%
\frametitle{Generalisierung der Regeln}

\begin{itemize}
\item spezifische Regeln für Kopf-Komplement-Kombination:\\
      \begin{tabular}[t]{@{}lll}
      \blau{V}[COMPS \ibox{A}] & $\to$ & \blau{\ibox{B}}\hspace{2em} \blau{V}[COMPS \ibox{A} $\oplus$ \sliste{ \ibox{B} } ]\\
      \blau{A}[COMPS \ibox{A}] & $\to$ & \blau{\ibox{B}}\hspace{2em} \blau{A}[COMPS \ibox{A} $\oplus$ \sliste{ \ibox{B} } ]\\
      \blau{N}[COMPS \ibox{A}] & $\to$ & \blau{\ibox{B}}\hspace{2em} \blau{N}[COMPS \ibox{A} $\oplus$ \sliste{ \ibox{B} } ]\\
      \blau{P}[COMPS \ibox{A}] & $\to$ & \blau{P}[COMPS \ibox{A} $\oplus$ \sliste{ \ibox{B} } ]\hspace{2em} \blau{\ibox{B}}\\
      \end{tabular}
\pause
\item Abstraktion von der Abfolge:
      \begin{tabular}[t]{@{}lll}
      \blau{V}[COMPS \ibox{A}] & $\to$ & \blau{V}[COMPS \ibox{A} $\oplus$ \sliste{ \ibox{B} } ]\hspace{2em} \blau{\ibox{B}}\\
      \blau{A}[COMPS \ibox{A}] & $\to$ & \blau{A}[COMPS \ibox{A} $\oplus$ \sliste{ \ibox{B} } ]\hspace{2em} \blau{\ibox{B}}\\
      \blau{N}[COMPS \ibox{A}] & $\to$ & \blau{N}[COMPS \ibox{A} $\oplus$ \sliste{ \ibox{B} } ]\hspace{2em} \blau{\ibox{B}}\\
      \blau{P}[COMPS \ibox{A}] & $\to$ & \blau{P}[COMPS \ibox{A} $\oplus$ \sliste{ \ibox{B} } ]\hspace{2em} \blau{\ibox{B}}\\
      \end{tabular}
\pause
\item generalisiertes, abstraktes Schema (H = Kopf):\\
      \begin{tabular}[t]{@{}lll}
      \blau{H}[COMPS \ibox{A}] & $\to$ & \blau{H}[COMPS \ibox{A} $\oplus$ \sliste{ \ibox{B} } ]\hspace{2em} \blau{\ibox{B}}\\
      \end{tabular}

\end{itemize}
}

\frame{
%
\frametitle{Verwendung der Regeln}

\begin{itemize}
\item generalisiertes, abstraktes Schema (H = Kopf):\\
      \begin{tabular}[t]{@{}lll}
      \blau{H}[COMPS \ibox{A}] & $\to$ & \blau{H}[COMPS \ibox{A} $\oplus$ \sliste{ \ibox{B} } ]~~~~ \blau{\ibox{B}}\\
      \end{tabular}
\pause
\item mögliche Instantiierungen des Schemas:
      \begin{tabular}[t]{@{}llll@{}}
      \blau{V}[COMPS \ibox{A}] & $\to$ & \blau{V}[COMPS \ibox{A} \eliste{} $\oplus$ \sliste{ \ibox{B} NP } ] & \blau{\ibox{B} NP}\\
                                &       & Conny erwartet & Aicke\\
                                &       & schläft        & Aicke\\\\
\pause
      \blau{V}[COMPS \ibox{A}] & $\to$ & \blau{V}[COMPS \ibox{A} \sliste{ NP } $\oplus$ \sliste{ \ibox{B} NP } ] & \blau{\ibox{B} NP}\\
                                &       & erwartet       & Conny\\\\
\pause
      \blau{N}[COMPS \ibox{A}] & $\to$ & \blau{N}[COMPS \ibox{A} \eliste{} $\oplus$ \sliste{ \ibox{B} Det } ] & \blau{\ibox{B} Det}\\
                                &       & Kind & das\\
      \end{tabular}
\end{itemize}
}


% Head ist hier noch nicht eingeführt
% \begin{comment}
% \frame{

% {\footnotesize
% \frametitle{Abkürzungen (unvollständige Strukturen)}


% \begin{tabular}{@{}lp{4cm}@{\hspace{8mm}}l@{}p{4cm}}
% DET    & $\ms{ head & \ms[det]{} \\
%                comps & \liste{} \\                     
%              }$ & 
% VP     & $\ms{ head & \ms[verb]{ subj & \liste{ $[~]$ } \\ 
%                                } \\
%                comps & \liste{ } \\
%              }$\\ \\
% $\overline{\mbox{N}}$ & $\ms{ head & \ms[noun]{} \\
%                               comps & \liste{ DET }\\
%              }$ &
% S      & $\ms{ head & \ms[verb]{ vform & fin \\ } \\
%                comps & \liste{} \\
%              }$\\ \\
% NP & $\ms{ head & \ms[noun]{} \\
%            comps & \liste{} \\
%              }$ &
% NP[\type{nom}]~~ & $\ms{ head & \ms[noun]{cas & nom \\ } \\
%                       comps & \liste{ } \\
%              }$ \\
% \end{tabular}

% }

% }
% \end{comment}
}% else hpsglight

\frame{
\frametitle{Repräsentation der Valenz in Merkmalsbeschreibungen}


\mbox{\emph{gibt} (finite Form):}\\
\ms{ phon & \phonliste{ gibt } \\
     part-of-speech & verb\\
     comps & \liste{ NP[\type{nom}], NP[\type{dat}], NP[\type{acc}]   } \\
}

\medskip

NP[\type{nom}], NP[\type{dat}] und NP[\type{acc}] stehen für komplexe Merkmalsbeschreibungen.

% \(
% \ms{ phon & \phonliste{ gibt } \\
%      p-o-s & verb\\
%              comps & \liste{ \onems{ head   \ms[noun]{cas & nom \\ } \\
%                                       comps \liste{ } \\
%                                  }, 
%                               \onems{ head \ms[noun]{cas & acc \\ } \\
%                                    comps \liste{ } \\
%                               }, 
%                               \onems{ head \ms[noun]{cas & dat \\ } \\
%                                       comps \liste{ } \\
%                                  }   } \\
% }
% \)

}


\subsection{Sepzifikatoren}

\frame{
\frametitle{Spezifikatoren}

\begin{itemize}
\item Neben \comps gibt es noch \spr als Valenzmerkmal.\\
      \spr für Analyse von Nominalstrukturen:
\eal
\ex die Tochter eines Mitarbeiters
\ex das Bild vom Gleimtunnel
\ex das Vorlesen von Büchern
\zl 

\pause

\centerfit{%
\begin{forest}
sm edges
[{N[\spr \eliste, \comps \eliste]}
   [Det [die] ]
   [N\feattab{
      \spr \nliste{ Det }, \comps \sliste{}}
     [N\feattab{
         \spr \nliste{ Det },\\
         \comps \nliste{ NP[\type{gen}] }} [Tochter] ]
        [{NP[\type{gen}]} [eines Mitarbeiters,roof] ] ] ]
\end{forest}}

\end{itemize}

}

\frame{
\frametitle{Spezifikator-Schema}

Parallel zum Kopf-Komplement-Schema:
\ea
\label{regelschema-psg-spr}
\begin{tabular}[t]{@{}lll}
H[\spr \ibox{1}] & $\to$ & \ibox{2}~~~ H[\spr \ibox{1}  $\oplus$ \sliste{ \ibox{2} }  ]\\
\end{tabular}
\z

}

\frame{
\frametitle{Spezifikatoren in SVO-Sprachen}

Englisch: Subjekt vor dem Verb (\spr), andere Argumente folgen ihm (\comps):

\centerfit{%
\begin{forest}
sm edges
[{V[\spr \eliste, \comps \eliste]}
   [{NP[\type{nom}]} [Aicke] ]
   [V\feattab{
      \spr \sliste{ NP[\type{nom}] }, \comps \sliste{}}
     [V\feattab{
         \spr \sliste{ NP[\type{nom}] },\\
         \comps \sliste{ NP[\type{acc}] }} [expects] ]
        [{NP[\type{acc}]} [Conny] ] ] ]
\end{forest}}


}

\frame{
\frametitle{Die Argumentstruktur}

\begin{itemize}
\item \argst als Repräsentation aller Argumente. 
\item Verteilung der Argumente auf die Valenzmerkmale in Abhängigkeit von der Sprache.
\end{itemize}

\ea
\label{ex-spr-comps-arg-st-Englisch}
\oneline{%
\begin{tabular}[t]{@{}ll@{~~}l@{~~}l@{}}
Verb          & \spr                      & \comps                                     & \argst\\
\emph{sleep}  & \sliste{ NP[\type{nom}] } & \sliste{}                                  & \sliste{ NP[\type{nom}] }\\
\emph{expect} & \sliste{ NP[\type{nom}] } & \sliste{ NP[\type{acc}] }                  & \sliste{ NP[\type{nom}], NP[\type{acc}] }\\
\emph{speak}  & \sliste{ NP[\type{nom}] } & \sliste{ PP[\type{about}] }                & \sliste{ NP[\type{nom}], PP[\type{about}] }\\
\emph{give}   & \sliste{ NP[\type{nom}] } & \sliste{ NP[\type{acc}], NP[\type{acc}] }  & \sliste{ NP[\type{nom}], NP[\type{acc}], NP[\type{acc}] }\\
\emph{serve}  & \sliste{ NP[\type{nom}] } & \sliste{ NP[\type{acc}], PP[\type{with}] } & \sliste{ NP[\type{nom}], NP[\type{acc}], PP[\type{with}] }\\  
\end{tabular}}
\z

\pause
\ea
\label{ex-spr-comps-arg-st-Deutsch}
\oneline{%
\begin{tabular}[t]{@{}ll@{~~}l@{~~}l@{}}
Verb            & \spr                      & \comps                                      & \argst\\
\emph{schlafen}  & \sliste{ } & \sliste{ NP[\type{nom}] }                                  & \sliste{ NP[\type{nom}] }\\
\emph{erwarten} & \sliste{ } & \sliste{ NP[\type{nom}], NP[\type{acc}] }                  & \sliste{ NP[\type{nom}], NP[\type{acc}] }\\
\emph{sprechen} & \sliste{ } & \sliste{ NP[\type{nom}], PP[\type{über}] }                 & \sliste{ NP[\type{nom}], PP[\type{about}] }\\
\emph{geben}    & \sliste{ } & \sliste{ NP[\type{nom}], NP[\type{dat}], NP[\type{acc}] }  & \sliste{ NP[\type{nom}], NP[\type{dat}], NP[\type{acc}] }\\
\emph{dienen}   & \sliste{ } & \sliste{ NP[\type{nom}], NP[\type{dat}], PP[\type{with}] } & \sliste{ NP[\type{nom}], NP[\type{dat}], PP[\type{with}] }\\  
\end{tabular}}
\z

}

\frame{
\frametitle{Argumentrealisierungsprinzip}

\ea
\label{Prinzip-Argumentrealisierung-einfach}%
Argumentrealisierungsprinzip\is{Prinzip!Argumentrealisierungs-} (vereinfachte Version):\\
\type{word} \impl
\avm{
[spr    & \1\\
 comps  & \2\\
 arg-st & \1 \+ \2]
}
%\itdopt{This allows non-cannonicals in \ibox{1}. Not sure this is what is intended.} 
\z

}

\subsection{Adjunkte}

\frame{

\frametitle{Adjunkte}

\begin{itemize}
\item Die Form von Adjunkten ist eher frei. 
\item Wortgruppen ganz verschiedener Kategorien können Köpfe modifizieren.
\item Andererseits haben Adjunkte genau Vorstellungen, wie der modifizierte Kopf aussehen muss.


\eal
\ex[]{
der laute Gesang
}
\ex[*]{
Aicke singt laute.
}
\ex[]{
Aicke singt laut.
}
\zl

\item Analog zu \spr und \comps selegieren Adjunkte ihren Kopf über \textsc{modified}.
\end{itemize}

}

\frame{
\frametitle{Attributive Adjektive}

\begin{itemize}
\item Adjektive, Nomina modifizierende Präpositionalphrasen und Relativsätze
selegieren \nbar = fast vollständige Nominalprojektion, \dash ein Nomen, das
nur noch mit einem Determinierer kombiniert werden muss, um eine vollständige
NP zu bilden. 

\ea\is{Adjektiv}
\label{le-interessantes}
\emph{interessantes}:\\
\ms{ 
   phon & \phonliste{ interessantes }\\
   p-o-s & adj\\
   mod &  \nbar\\
   spr   & \eliste\\
   comps & \liste{} \\
}
\z

\end{itemize}

}


\frame{
\frametitle{Adjektive mit Komplementen}

\begin{itemize}
\item Adjektive können Komplemente haben:
\ea
ein dem König treues Mädchen
\z

\ea
\label{le-treue}
\emph{treues}:\\
\ms{ 
   phon & \phonliste{ treues }\\
   p-o-s & adj\\ %prd & $-$ \\
   mod &  \nbar\\
   spr   & \eliste\\
   comps & \liste{ NP[\type{dat}] } \\
}
\z
\pause
\item \emph{dem König treues} bildet Adjektivphrase, die \emph{Mädchen} modifiziert.

\pause
\item \modw muss bei Kombination nach oben gegeben werden.

\end{itemize}

}

\frame{
\frametitle{Projektion des \modwes}

\begin{itemize}
\item 
Die Adjektivphrase \emph{dem König treues} hat dieselben syntaktischen Eigenschaften wie \emph{interessantes}:
\ea
\label{avm-dem-koenig-treues}
\emph{dem König treues}:\\
\ms{ 
   phon & \phonliste{ dem König treues }\\
   p-o-s & adj\\ %prd & $-$ \\
   mod &  \nbar\\
   spr & \eliste\\
   comps & \eliste{ } \\
}
\z

\item Darauf kommen wir im nächsten Kapitel zurück. \textsc{mod} ist ein Kopfmerkmal.
\item Das Kopfmerkmalsprinzip sorgt dafür, dass der \modw der gesamten Projektion mit dem
\modw des Lexikoneintrags für \emph{treues} identisch ist. 

\end{itemize}
}

\frame[shrink=4]{
\frametitle{Selektion in Kopf"=Adjunkt"=Strukturen}

\centerline{
\begin{forest}
sm edges
[\nbar
  [{AP[\textsc{mod} \ibox{1}]}
    [interessantes]]
  [\ibox{1} \nbar
    [Buch]]]
\end{forest}
}

\begin{itemize}
\item In Kopf-Argument-Strukturen bestimmt der Kopf die Form der Argumente. 
\item kann in der \compsl festlegen, dass eine Argument-NP vollständig sein muss, \dash, dass die \sprl und die \compsl des
entsprechenden Arguments leer sein muss. 
\item Bei Adjunkten hat der Kopf keinen Zugriff auf das Adjunkt. 
\item Deshalb muss in der Grammatikregel, die Kopf und Adjunkt kombiniert, festgelegt werden,
dass das Adjunkt gesättigt ist, \dash, dass \sprl und \compsl der Adjunkttochter leer sind. 
\ea
\label{regelschema-psg-adjunct}
\oneline{\begin{tabular}[t]{@{}lll@{}}
H[\spr \ibox{1}, \comps \ibox{2}] & $\to$ & [\textsc{mod} \ibox{3}, \spr \eliste, \comps \eliste]~~~ \ibox{3}~H[\spr \ibox{1}, \comps \ibox{2} ]\\
\end{tabular}}
\z
\end{itemize}

}
% In (\mex{0}) wählt die Adjunkttochter über \ibox{3} den Kopf, den sie modifizieren will. \spr und
% \comps der Adjunkttochter müssen leer sein. Der \sprw und der \compsw der Kopftochter werden mit dem
% \sprw und dem \compsw der Mutter identifiziert (\ibox{1} und \ibox{2}).

\frame[shrink=4]{
\frametitle{Präpositionalphrasen}

Schema schließt Phrasen wie (\mex{1}b) korrekt aus:
\eal
\label{ex-die-Wurst-in}
\ex[]{
die Wurst in der Speisekammer
}
\ex[*]{
die Wurst in
}
\zl

\pause

\ea
Lexikoneintrag für \emph{in}:\\
\ms{
phon & \phonliste{ in } \\
p-o-s & prep\\
mod &   \nbar\\
spr & \eliste\\
comps & \sliste{ NP[\type{dat}] } \\
}
\z
\pause
Nach der Kombination mit der Nominalphrase \emph{der Speisekammer} bekommt man:
\ea
Repräsentation für \emph{in der Speisekammer}:\\
\ms{
phon & \phonliste{ in der Speisekammer } \\
p-o-s & prep\\
mod &   \nbar\\
spr & \eliste\\
comps & \eliste{ } \\
}
\z
Diese Repräsentation entspricht der des Adjektivs \emph{interessantes} und kann 
-- abgesehen von der Stellung der PP -- auch genauso verwendet
werden: Die PP modifiziert eine \nbar.
}

\frame{
\frametitle{Köpfe, die keine Adjunkte sind}

\begin{itemize}
\item Köpfe, die nur als Argumente verwendet werden können und nicht selbst modifizieren,
haben als \modw \type{none}. 
\item Dadurch können sie in Kopf"=Adjunkt"=Strukturen nicht an die Stelle
der Nicht"=Kopf"|tochter treten, da der \modw der Nicht"=Kopf"|tochter mit der Kopf"|tochter kompatibel
sein muss.
\end{itemize}

}


\iftoggle{hpsglight}{}{

%% \iftoggle{teil1}{
%% \frame{

%% \frametitle{Reihenfolge der Elemente in der Comps-Liste}

%% Obliqueness-Hierarchie:\\
%% \citet{KC77a}\ia{Keenan}\ia{Comrie}, \citet{Pullum77a}\ia{Pullum},
%% \citet{ps}\ia{Pollard}\ia{Sag} und Grewendorf (\citeyear{Grewendorf85a}; 
%% \citeyear{Grewendorf88a})\ia{Grewendorf}
%% \begin{table}[H]
%% \resizebox{\linewidth}{!}{
%% \begin{tabular}{@{}l@{\hspace{1ex}}l@{\hspace{1ex}}l@{\hspace{1ex}}l@{\hspace{1ex}}l@{\hspace{1ex}}l}
%% SUBJECT $=>$ & DIRECT $=>$ & INDIRECT $=>$ & OBLIQUES $=>$ & GENITIVES $=>$  & OBJECTS OF \\
%%              & OBJECT      & OBJECT        &               &                 & COMPARISON 
%% \end{tabular}%
%% }\label{page-obliquen-h}
%% \end{table}

%% \begin{itemize}
%% \item syntaktische Aktivität der grammatischen Funktion
%% \item höhere Elemente kommen eher in syntaktischen Konstruktionen vor
%%       \begin{itemize}
%%       \item Ellipse \citep{Klein85}
%%       \item Vorfeldellipse\is{Vorfeldellipse@\type{Vorfeldellipse}} \citep{Fries88b}\ia{Fries}
%%       \item freie Relativsätze
%%             \citep{Bausewein90,Pittner95b,Mueller99c}\ia{Bausewein}\ia{Pittner}\ia{Müller}
%%       \item Passiv\is{passive} \citep{KC77a}
%%       \item Zustandsprädikate \citep{Mueller2000d,Mueller2001c}
%%       \item Bindungstheorie (\citealp{Grewendorf85a}\ia{Grewendorf}; \citealp{ps2})\ia{Pollard}\ia{Sag}
%%       \end{itemize}
%% \end{itemize}

%% }

%% \frame{

%% \frametitle{Alternative Anordnung begründet mit "`Verbnähe"'}

%% \begin{tabular}{@{}l@{\hspace{1ex}}l@{\hspace{1ex}}l@{\hspace{1ex}}l@{\hspace{1ex}}l@{\hspace{1ex}}l}
%% Nominativ $=>$ & Dativ $=>$ & Akkusativ\\
%% \end{tabular}

%% \begin{itemize}
%% \item Konstituentenstellung
%% \item angeblich Voranstellung mit Dativ nicht möglich\\
%%       (\citealt{Haftka81}; \citealt{Haider82};
%%       \citet{Wegener90}; \citet{Zifonun92a})
%% \item aber: \citet{Uszkoreit87a}, \citet{SS88a}, %zitieren Thiersch82a-unread
%%       \citet{Oppenrieder91a} und \citet{Grewendorf93}
%% \end{itemize}

%% \eal
%% %\label{bsp-syntax-pvp-besonders}
%% \ex Besonders Einsteigern empfehlen\iw{empfehlen} möchte ich Quarterdeck Mosaic, dessen gelungene grafische 
%%       Oberfläche und Benutzerführung auf angenehme Weise über die ersten Hürden 
%%       hinweghilft, obwohl sich die Funktionalität auch nicht zu verstecken braucht. (c't, 9/95, S.\,156)
%% \ex Der Nachwelt hinterlassen\iw{hinterlassen} hat sie eine aufgeschlagene \emph{Hör zu} und einen kurzen
%%       Abschiedsbrief: \ldots{} (taz, 18.11.98, S.\,20)
%% \zl
%% \ea Viel anfangen\iw{anfangen mit} konnte er damit nicht. (Wochenpost, 41/95, S.\,34)
%% \z

%% }
%% } % \end{teil1}


\subsection{Übungsaufgaben}

\frame{
\frametitle{Übungsaufgaben}

\begin{enumerate}
\item Geben Sie die Valenzlisten der für folgende Wörter an:
      \eal
      \ex er
      \ex seine (in \emph{seine Ankündigung})
      \ex schnarcht
      \ex denkt
      \zl
\end{enumerate}


}


}% end hpsglight
