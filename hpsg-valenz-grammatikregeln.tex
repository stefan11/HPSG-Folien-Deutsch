%%%%%%%%%%%%%%%%%%%%%%%%%%%%%%%%%%%%%%%%%%%%%%%%%%%%%%%%%
%%   $RCSfile: hpsg-valenz-grammatikregeln.tex,v $
%%  $Revision: 1.3 $
%%      $Date: 2007/05/27 13:27:12 $
%%     Author: Stefan Mueller (CL Uni-Bremen)
%%    Purpose: 
%%   Language: LaTeX
%%%%%%%%%%%%%%%%%%%%%%%%%%%%%%%%%%%%%%%%%%%%%%%%%%%%%%%%%

\subtitle{Valenz, Grammatikregeln und Komplementation}

\huberlintitlepage[22pt]

\section{Valenz und Grammatikregeln}

\iftoggle{teil1}{
\outline{

\begin{itemize}
%\item Ziele
\item Wozu Syntax? / Phrasenstrukturgrammatiken
\item Formalismus
\item \blau{Valenz und Grammatikregeln}
\item Komplementation
\item Semantik
\item Adjunktion und Spezifikation
\item Das Lexikon: Typen und Lexikonregeln
\item Topologie des deutschen Satzes
\item Konstituentenreihenfolge
\item Nichtlokale Abhängigkeiten
\item Relativsätze
\item Lokalität
%\item Komplexe Prädikate: Der Verbalkomplex
\end{itemize}
}
} % \end{teil1}

\iftoggle{teil2}{
\outline{
\begin{itemize}
\item Wiederholung
      \begin{itemize}
\item Wozu Syntax? / Phrasenstrukturgrammatiken
\item Formalismus
\item \blau{Valenz und Grammatikregeln}
\item Komplementation
\item Semantik
\item Topologie des deutschen Satzes
\item Konstituentenreihenfolge
\item Nichtlokale Abhängigkeiten
\end{itemize}
\item Kongruenz
\item Kasus
\item Der Verbalkomplex
\item Kohärenz, Inkohärenz, Anhebung und Kontrolle
\item Passiv
\item Partikelverben
\item Morphologie
\end{itemize}
}
} % \end{teil2}


\frame{
\frametitle{Valenz und Grammatikregeln}
%
\begin{itemize}
\item Literatur: \citew[Kapitel~3.1]{MuellerLehrbuch3}
\end{itemize}

\vspace{1cm}


\vspace{1cm}

\rotbf{Bitte zum nächsten Mal lesen}

\vspace{1cm}

\iftoggle{hpsglight}{%
Damit alles kompatibel zum Lehrbuch bleibt,\\
nehmen wir hier auch das \subcatm für die Valenz an.

\subcat = \spr + \comps

}

%% \rotbf{Achtung, wichtiger Hinweis: Diese Literaturangabe hier bedeutet,\\dass Sie die Literatur zum
%%   nächsten Mal lesen sollen!!!!}
}

\iftoggle{hpsglight}{}{%
\frame{
\frametitle{Valenz in der Chemie und in der Linguistik}

\vfill

\centerline{
\begin{forest}
[O
  [H] 
  [H] ]
\end{forest}
\hspace{5em}
\begin{forest}
[helfen
 [Peter]
 [Maria] ]
\end{forest}
}


\vfill

}

\frame{
\frametitle{Valenz und Grammatikregeln: PSG}


\begin{itemize}
\item große Anzahl von Regeln:\\
      \begin{tabular}[t]{l@{~$\to$~}l@{\hspace{6em}}l}
      S &  NP, V               & \emph{X schläft}\\
      S &  NP, NP, V           & \emph{X Y liebt}\\
      S &  NP, PP[\type{über}], V           & \emph{X über y spricht}\\
      S &  NP, NP, NP, V       & \emph{X Y Z gibt}\\
      S &  NP, NP, PP[\type{mit}], V       & \emph{X Y mit Z dient}\\
      \end{tabular}
\pause
\item Verben müssen mit passender Regel verwendet werden.
\end{itemize}

}

\frame{

\frametitle{Valenz und Grammatikregeln: HPSG}

\begin{itemize}
\item Argumente als komplexe Kategorien in der lexikalischen Repräsentation
      eines Kopfes repräsentiert\\
      (wie Kategorialgrammatik)
\pause
\item \begin{tabular}[t]{@{}lll}
      Verb             & \subcat\\
      \emph{schlafen} & \sliste{ NP }\\
      \emph{lieben}   & \sliste{ NP, NP }\\
      \emph{sprechen} & \sliste{ NP, PP[\type{über}] }\\
      \emph{geben}    & \sliste{ NP, NP, NP }\\
      \emph{dienen}   & \sliste{ NP, NP, PP[\type{mit}] }\\  
      \end{tabular}
\end{itemize}
}

\frame{
\frametitle{Beispielstruktur mit Valenzinformation (I)}

\vfill
\hfill\begin{forest}
sm edges
[{V[\subcat \sliste{} ]}
  [\ibox{1} NP
    [Aicke]]
  [{V[\subcat \sliste{ \ibox{1} }]}
    [schläft]]]
\end{forest}\hfill\hfill\mbox{}
\vfill
V[\subcat \sliste{ }] entspricht hierbei einer vollständigen Phrase\\
(VP oder auch S)
\vfill
}

\frame{
\frametitle{Beispielstruktur mit Valenzinformation (II)}

\vfill
\centerline{%
\begin{forest}
sm edges
[{V[\subcat \sliste{} ]}
  [\ibox{1} NP
    [Aicke]]
  [{V[\subcat \sliste{ \ibox{1} }]}
    [\ibox{2} NP 
      [Conny]]
    [{V[\subcat \sliste{ \ibox{1}, \ibox{2} }]}
      [erwartet]]]]
\end{forest}}
\vfill

}

\frame{

\frametitle{Valenz und Grammatikregeln: HPSG}

\begin{itemize}
\item spezifische Regeln für Kopf-Argument-Kombination:\\
      \begin{tabular}[t]{@{}lll}
      V[SUBCAT \ibox{A}] & $\to$ & \ibox{B}\hspace{2em} V[SUBCAT \ibox{A} $\oplus$ \sliste{ \ibox{B} } ]\\
      \end{tabular}
\pause
\item Dabei ist $\oplus$ eine Relation zur Verknüpfung zweier Listen:\\
      \begin{tabular}{@{}l@{~}l@{}}
      \phonliste{ a, b } =& \phonliste{ a } $\oplus$ \phonliste{ b } oder\\
                    & \phonliste{} $\oplus$ \phonliste{ a, b } oder\\
                    & \phonliste{ a, b } $\oplus$ \phonliste{}\\
      \end{tabular}
\end{itemize}
}

\frame{
\frametitle{Valenz und Grammatikregeln (II)}

\centerline{
\begin{forest}
sm edges, for tree={l sep+=\baselineskip}
[{V[\subcat \sliste{} ]}
  [\ibox{1} NP
    [Aicke]]
  [{V[\subcat \sliste{ \ibox{1} }]},name=v2
    [\ibox{2} NP 
      [Conny]]
    [{V[\subcat \sliste{ \ibox{1}, \ibox{2} }]},name=v3
      [erwartet]]]]
\node()[above right of=v2,xshift=12em,yshift=0\baselineskip]{\small \begin{tabular}{@{}l}
                                    V[SUBCAT \ibox{A}] $\to$ \ibox{B} V[SUBCAT \ibox{A} $\oplus$ \sliste{ \ibox{B} } ] \\ 
                                    A = \sliste{ }, B = \ibox{1}
                                    \end{tabular}};
\node()[above right of=v3,xshift=12em,yshift=0\baselineskip]{\small \begin{tabular}{@{}l}
                                    V[SUBCAT \ibox{A}] $\to$ \ibox{B} V[SUBCAT \ibox{A} $\oplus$ \sliste{ \ibox{B} } ] \\ 
                                    A = \sliste{ \ibox{1} }, B = \ibox{2}
                                    \end{tabular}};
\end{forest}}

}


\frame{
%
\frametitle{Generalisierung der Regeln}

\begin{itemize}
\item spezifische Regeln für Kopf-Komplement-Kombination:\\
      \begin{tabular}[t]{@{}lll}
      \blau{V}[SUBCAT \ibox{A}] & $\to$ & \blau{\ibox{B}}\hspace{2em} \blau{V}[SUBCAT \ibox{A} $\oplus$ \sliste{ \ibox{B} } ]\\
      \blau{A}[SUBCAT \ibox{A}] & $\to$ & \blau{\ibox{B}}\hspace{2em} \blau{A}[SUBCAT \ibox{A} $\oplus$ \sliste{ \ibox{B} } ]\\
      \blau{N}[SUBCAT \ibox{A}] & $\to$ & \blau{\ibox{B}}\hspace{2em} \blau{N}[SUBCAT \ibox{A} $\oplus$ \sliste{ \ibox{B} } ]\\
      \blau{P}[SUBCAT \ibox{A}] & $\to$ & \blau{P}[SUBCAT \ibox{A} $\oplus$ \sliste{ \ibox{B} } ]\hspace{2em} \blau{\ibox{B}}\\
      \end{tabular}
\pause
\item Abstraktion von der Abfolge:
      \begin{tabular}[t]{@{}lll}
      \blau{V}[SUBCAT \ibox{A}] & $\to$ & \blau{V}[SUBCAT \ibox{A} $\oplus$ \sliste{ \ibox{B} } ]\hspace{2em} \blau{\ibox{B}}\\
      \blau{A}[SUBCAT \ibox{A}] & $\to$ & \blau{A}[SUBCAT \ibox{A} $\oplus$ \sliste{ \ibox{B} } ]\hspace{2em} \blau{\ibox{B}}\\
      \blau{N}[SUBCAT \ibox{A}] & $\to$ & \blau{N}[SUBCAT \ibox{A} $\oplus$ \sliste{ \ibox{B} } ]\hspace{2em} \blau{\ibox{B}}\\
      \blau{P}[SUBCAT \ibox{A}] & $\to$ & \blau{P}[SUBCAT \ibox{A} $\oplus$ \sliste{ \ibox{B} } ]\hspace{2em} \blau{\ibox{B}}\\
      \end{tabular}
\pause
\item generalisiertes, abstraktes Schema (H = Kopf):\\
      \begin{tabular}[t]{@{}lll}
      \blau{H}[SUBCAT \ibox{A}] & $\to$ & \blau{H}[SUBCAT \ibox{A} $\oplus$ \sliste{ \ibox{B} } ]\hspace{2em} \blau{\ibox{B}}\\
      \end{tabular}

\end{itemize}
}

\frame{
%
\frametitle{Verwendung der Regeln}

\begin{itemize}
\item generalisiertes, abstraktes Schema (H = Kopf):\\
      \begin{tabular}[t]{@{}lll}
      \blau{H}[SUBCAT \ibox{A}] & $\to$ & \blau{H}[SUBCAT \ibox{A} $\oplus$ \sliste{ \ibox{B} } ]~~~~ \blau{\ibox{B}}\\
      \end{tabular}
\pause
\item mögliche Instantiierungen des Schemas:
      \begin{tabular}[t]{@{}llll@{}}
      \blau{V}[SUBCAT \ibox{A}] & $\to$ & \blau{V}[SUBCAT \ibox{A} \eliste{} $\oplus$ \sliste{ \ibox{B} NP } ] & \blau{\ibox{B} NP}\\
                                &       & Conny erwartet & Aicke\\
                                &       & schläft        & Aicke\\\\
\pause
      \blau{V}[SUBCAT \ibox{A}] & $\to$ & \blau{V}[SUBCAT \ibox{A} \sliste{ NP } $\oplus$ \sliste{ \ibox{B} NP } ] & \blau{\ibox{B} NP}\\
                                &       & erwartet       & Conny\\\\
\pause
      \blau{N}[SUBCAT \ibox{A}] & $\to$ & \blau{N}[SUBCAT \ibox{A} \eliste{} $\oplus$ \sliste{ \ibox{B} Det } ] & \blau{\ibox{B} Det}\\
                                &       & Kind & das\\
      \end{tabular}
\end{itemize}
}


% Head ist hier noch nicht eingeführt
% \begin{comment}
% \frame{

% {\footnotesize
% \frametitle{Abkürzungen (unvollständige Strukturen)}


% \begin{tabular}{@{}lp{4cm}@{\hspace{8mm}}l@{}p{4cm}}
% DET    & $\ms{ head & \ms[det]{} \\
%                subcat & \liste{} \\                     
%              }$ & 
% VP     & $\ms{ head & \ms[verb]{ subj & \liste{ $[~]$ } \\ 
%                                } \\
%                subcat & \liste{ } \\
%              }$\\ \\
% $\overline{\mbox{N}}$ & $\ms{ head & \ms[noun]{} \\
%                               subcat & \liste{ DET }\\
%              }$ &
% S      & $\ms{ head & \ms[verb]{ vform & fin \\ } \\
%                subcat & \liste{} \\
%              }$\\ \\
% NP & $\ms{ head & \ms[noun]{} \\
%            subcat & \liste{} \\
%              }$ &
% NP[\type{nom}]~~ & $\ms{ head & \ms[noun]{cas & nom \\ } \\
%                       subcat & \liste{ } \\
%              }$ \\
% \end{tabular}

% }

% }
% \end{comment}
}% else hpsglight

\frame{
\frametitle{Repräsentation der Valenz in Merkmalsbeschreibungen}


\mbox{\emph{gibt} (finite Form):}\\
\ms{ phon & \phonliste{ gibt } \\
     part-of-speech & verb\\
     subcat & \liste{ NP[\type{nom}], NP[\type{dat}], NP[\type{acc}]   } \\
}

\medskip

NP[\type{nom}], NP[\type{acc}] und NP[\type{dat}] stehen für komplexe Merkmalsbeschreibungen.

% \(
% \ms{ phon & \phonliste{ gibt } \\
%      p-o-s & verb\\
%              subcat & \liste{ \onems{ head   \ms[noun]{cas & nom \\ } \\
%                                       subcat \liste{ } \\
%                                  }, 
%                               \onems{ head \ms[noun]{cas & acc \\ } \\
%                                    subcat \liste{ } \\
%                               }, 
%                               \onems{ head \ms[noun]{cas & dat \\ } \\
%                                       subcat \liste{ } \\
%                                  }   } \\
% }
% \)

}

\iftoggle{hpsglight}{}{

%% \iftoggle{teil1}{
%% \frame{

%% \frametitle{Reihenfolge der Elemente in der Subcat-Liste}

%% Obliqueness-Hierarchie:\\
%% \citet{KC77a}\ia{Keenan}\ia{Comrie}, \citet{Pullum77a}\ia{Pullum},
%% \citet{ps}\ia{Pollard}\ia{Sag} und Grewendorf (\citeyear{Grewendorf85a}; 
%% \citeyear{Grewendorf88a})\ia{Grewendorf}
%% \begin{table}[H]
%% \resizebox{\linewidth}{!}{
%% \begin{tabular}{@{}l@{\hspace{1ex}}l@{\hspace{1ex}}l@{\hspace{1ex}}l@{\hspace{1ex}}l@{\hspace{1ex}}l}
%% SUBJECT $=>$ & DIRECT $=>$ & INDIRECT $=>$ & OBLIQUES $=>$ & GENITIVES $=>$  & OBJECTS OF \\
%%              & OBJECT      & OBJECT        &               &                 & COMPARISON 
%% \end{tabular}%
%% }\label{page-obliquen-h}
%% \end{table}

%% \begin{itemize}
%% \item syntaktische Aktivität der grammatischen Funktion
%% \item höhere Elemente kommen eher in syntaktischen Konstruktionen vor
%%       \begin{itemize}
%%       \item Ellipse \citep{Klein85}
%%       \item Vorfeldellipse\is{Vorfeldellipse@\type{Vorfeldellipse}} \citep{Fries88b}\ia{Fries}
%%       \item freie Relativsätze
%%             \citep{Bausewein90,Pittner95b,Mueller99c}\ia{Bausewein}\ia{Pittner}\ia{Müller}
%%       \item Passiv\is{passive} \citep{KC77a}
%%       \item Zustandsprädikate \citep{Mueller2000d,Mueller2001c}
%%       \item Bindungstheorie (\citealp{Grewendorf85a}\ia{Grewendorf}; \citealp{ps2})\ia{Pollard}\ia{Sag}
%%       \end{itemize}
%% \end{itemize}

%% }

%% \frame{

%% \frametitle{Alternative Anordnung begründet mit "`Verbnähe"'}

%% \begin{tabular}{@{}l@{\hspace{1ex}}l@{\hspace{1ex}}l@{\hspace{1ex}}l@{\hspace{1ex}}l@{\hspace{1ex}}l}
%% Nominativ $=>$ & Dativ $=>$ & Akkusativ\\
%% \end{tabular}

%% \begin{itemize}
%% \item Konstituentenstellung
%% \item angeblich Voranstellung mit Dativ nicht möglich\\
%%       (\citealt{Haftka81}; \citealt{Haider82};
%%       \citet{Wegener90}; \citet{Zifonun92a})
%% \item aber: \citet{Uszkoreit87a}, \citet{SS88a}, %zitieren Thiersch82a-unread
%%       \citet{Oppenrieder91a} und \citet{Grewendorf93}
%% \end{itemize}

%% \eal
%% %\label{bsp-syntax-pvp-besonders}
%% \ex Besonders Einsteigern empfehlen\iw{empfehlen} möchte ich Quarterdeck Mosaic, dessen gelungene grafische 
%%       Oberfläche und Benutzerführung auf angenehme Weise über die ersten Hürden 
%%       hinweghilft, obwohl sich die Funktionalität auch nicht zu verstecken braucht. (c't, 9/95, S.\,156)
%% \ex Der Nachwelt hinterlassen\iw{hinterlassen} hat sie eine aufgeschlagene \emph{Hör zu} und einen kurzen
%%       Abschiedsbrief: \ldots{} (taz, 18.11.98, S.\,20)
%% \zl
%% \ea Viel anfangen\iw{anfangen mit} konnte er damit nicht. (Wochenpost, 41/95, S.\,34)
%% \z

%% }
%% } % \end{teil1}

\subsection{Übungsaufgaben}

\frame{
\frametitle{Übungsaufgaben}

\begin{enumerate}
\item Geben Sie die Valenzlisten der für folgende Wörter an:
      \eal
      \ex er
      \ex seine (in \emph{seine Ankündigung})
      \ex schnarcht
      \ex denkt
      \zl
\end{enumerate}


}

}% end hpsglight
