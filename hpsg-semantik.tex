
\section{Semantik}


\iftoggle{teil1}{
\outline{

\begin{itemize}
\item Wozu Syntax? / Phrasenstrukturgrammatiken
\item Formalismus
\item Valenz und Grammatikregeln
\item Komplementation
\item \blaubf{Semantik}
\item Adjunktion und Spezifikation
\item Das Lexikon: Typen und Lexikonregeln
\item Topologie des deutschen Satzes
\item Konstituentenreihenfolge
\item Nichtlokale Abhängigkeiten
\item Relativsätze
\item Lokalität
%\item Komplexe Prädikate: Der Verbalkomplex
\end{itemize}
}
} % \end{teil1}

\iftoggle{teil2}{
\outline{
\begin{itemize}
\item Wiederholung
      \begin{itemize}
\item Wozu Syntax? / Phrasenstrukturgrammatiken
\item Formalismus
\item Valenz und Grammatikregeln
\item Komplementation
\item \blau{Semantik}
\item Topologie des deutschen Satzes
\item Konstituentenreihenfolge
\item Nichtlokale Abhängigkeiten
\end{itemize}
\item Kongruenz
\item Kasus
\item Der Verbalkomplex
\item Kohärenz, Inkohärenz, Anhebung und Kontrolle
\item Passiv
\item Partikelverben
\item Morphologie
\end{itemize}
}
} % \end{teil2}

\frame{
\frametitle{Semantik}
%
\begin{itemize}
\item Literatur: \citew[Kapitel~5]{MuellerLehrbuch3}
\end{itemize}

\vspace{1cm}

\rotbf{Achtung, wichtiger Hinweis: Diese Literaturangabe hier bedeutet,\\dass Sie die Literatur zum
   nächsten Mal lesen sollen!!!!}
}


\iftoggle{teil1}{
\frame{
\frametitle{\hypertarget{semantics-t}{Semantik}}

\begin{itemize}
\item \citet{ps} und \citet{GSag2000a-u} nehmen Situationssemantik an
      \citep*{BP83a,CMP90,Devlin92}\nocite{BP87a}.
\pause
\item Einige aktuellere Arbeiten benutzen \emph{Minimal Recursion Semantics} \citep*{CFPS2005a}.
\pause
\item Im Folgenden werden wir Situationssemantik nutzen.
\end{itemize}

}
} % \end{teil1}

\subsection{Die Situationssemantik}


\frame{

\frametitle{Individuen, Sachverhalte und Situationen}

\begin{itemize}[<+->]
\item beschreiben Situationen
\item Situationen sind durch Sachverhalte charakterisiert
\item Dinge von einer gewissen zeitlichen Dauer, die zur kausalen Ordnung der
Welt gehö\-ren, die man wahrnehmen kann, auf die man reagieren kann: Individuen ({\em Karl\/}, {\em die Frau\/},
{\em die Angst\/}, {\em das Versprechen\/})
\item Sachverhalte = Relationen zwischen Individuen
\end{itemize}
}

\frame{
\frametitle{Relationen und semantische Rollen}

\begin{itemize}
\item Relationen
      \begin{itemize}
      \item nullstellig: {\em regnen\/} ({\em Es regnet.\/}) \citep[Kapitel~2.8]{Kunze93}
\pause
      \item einstellig: {\em  sterben\/} ({\em Es stirbt.\/})
\pause
      \item zweistellig: {\em lieben\/}  ({\em Es liebt ihn.\/})
\pause
      \item dreistellig: {\em geben\/}   ({\em Es gibt ihm den Aufsatz.\/})
\pause
      \item vierstellig: {\em kaufen\/}  ({\em Es kauft den Mantel vom Händler für fünf Mark.\/})
      \end{itemize}
\pause
\item semantische Rollen:  \citet{Fillmore68,Fillmore77}, \citet*{Kunze91}\\
      {\sc agens}, {\sc patiens}, {\sc experiencer}, {\sc source}, {\sc goal}, 
      {\sc thema}, {\sc location}, {\sc trans-obj}, {\sc instrument}, {\sc means} und {\sc proposition}
\pause
\item Rollen wichtig für Generalisierungen:\\
      Verbindung zwischen Syntax und Semantik ({\it Linking\/})
\end{itemize}
}

\frame{
\frametitle{Sachverhalte}



\begin{itemize}
\item Sachverhalt: {\it state of affairs\/} ({\it soa\/})\\
\begin{tabular}{@{}ll}
Verb:     & $\ll schlagen, agens:X, patiens:Y\gg$\\
Adjektiv: & $\ll interessant, thema: X\gg$\\
Nomen:    & $\ll mann, instance: X\gg$\\
\end{tabular}
\end{itemize}

}

\frame{
\frametitle{Parametrisierte Sachverhalte}

\begin{itemize}
\item parametrisierter Sachverhalt: {\it parmetrized state of affairs\/} ({\it psoa\/})
\begin{itemize}
\item Verb:
\ea
Der Mann schlägt den Hund.
\z
$\ll schlagen, agens:X, patiens:Y\gg$\\
$X|\ll mann, instance:X\gg,$\\
$Y|\ll hund, instance:Y\gg$
\pause
\medskip
\item Adjektiv:
\ea
Das Buch ist interessant.
\z
$\ll interessant, thema:X\gg$\\
$X|\ll buch, instance:X\gg$
\end{itemize}
\end{itemize}

}

\subsection{Die Repräsentation parametrisierter Sachverhalte mit Hilfe von Merkmalstrukturen}

\frame{
\frametitle{Sachverhalte und Repräsentation mit Merkmalstrukturen}


$\ll schlagen, agens:X, patiens:Y\gg$\\
\ms[schlagen]
{ 
 agens & X \\
 patiens & Y\\
}

\pause
$\ll mann, in\-stance\!: X\gg$\\
\ms[mann]
{ 
 inst & X \\
}


}

\subsection{Repräsentation des CONT-Wertes}

\frame[shrink]{
\frametitlefit{Repräsentation in Merkmalsbeschreibungen: der \contw}

\begin{itemize}
\item mögliche Datenstruktur ({\sc cont} = {\sc content}):\\
      \ms{ phon   & list~of~phoneme strings\\
           head   & head \\
           subcat & list\\
           cont   & cont\\
         }\\
\pause
\item stärkere Gliederung, Unterteilung in syntaktische und semantische Information
      ({\sc cat} = {\sc category})\\
      \ms{ phon & list~of~phoneme strings\\
           cat  & \ms[cat]{ head   & head\\
                            subcat & list\\
                          } \\
           cont & cont\\
         }
\item $\to$ möglich, nur syntaktische Information zu teilen
\end{itemize}
}

\frame{
\frametitle{Teilung syntaktischer Information in Koordinationen}

\begin{itemize}
\item symmetrische Koordination: der \catw ist identisch
      \ms{ phon & list~of~phoneme strings\\
           cat  & \ms[cat]{ head   & head\\
                            subcat & list\\
                          } \\
           cont & cont\\
         }
\item Beispiele:
      \eal
      \ex {}[der Mann und die Frau]
      \ex Er [kennt und liebt] diese Schallplatte.
      \ex Er ist [dumm und arrogant].
      \zl
\end{itemize}
}


\subsection{Nominale Objekte}
\frame{
\frametitle{Semantischer Beitrag nominaler Objekte}

\small
\begin{itemize}
\item semantischer Index + zugehörige Restriktionen

\mbox{{\it Buch\/}:}\\
\ms
{ cat & \ms{ head   & noun \\
             subcat & \liste{ det } \\
           } \\
  cont &  \ms
           { ind & \ibox{1} \ms{ per & 3 \\
                                 num & sg \\
                                 gen & neu \\
                               } \\
             restr & \liste{ \ms[buch]{ inst & \ibox{1} \\ }} \\
           } \\
}
\pause
\item Person, Numerus und Genus sind für die Bestimmung der Referenz/Koreferenz wichtig:
      \ea
      Die Frau$_i$ kauft ein Buch$_j$. Sie$_i$ liest es$_j$.
      \z
\end{itemize}

}

\begin{comment}
\frame{

\frametitle{Typen nominaler Objekte}

\bigskip
\begin{figure}[htbp]
\epsfxsize=0.5\textwidth
\centerline{\mbox{\epsffile{types-nom-obj.eps}}}
\end{figure}
{\footnotesize
\begin{itemize}
\item nom-obj
\begin{itemize}
\item npro = nicht-pronominal ({\em Frau\/}, {\em Hütte\/})
\item pro = pronominal
\begin{itemize}
\item ppro = Personalpronomina ({\em er\/}, {\em sie\/})
\item ana = Anaphren
\begin{itemize}
\item refl = Reflexiva ({\em sich\/})
\item recp = Reziprok ({\em einander\/})
\end{itemize}
\end{itemize}
\end{itemize}
\end{itemize}
}

}

\frame{

\frametitle{Referentielle und expletive Indizes}

\bigskip
\begin{itemize}
\item ind
\begin{itemize}
\item expl = Expletiva ({\em es\/}, {\em das\/})
\item ref = referentiell
\end{itemize}
\end{itemize}
Expletiva sind bedeutungsleer, ohne Referenz

expletives {\em es\/}:
\eal
\label{bsp-expletiva-subcat}
\ex Es regnet.
\ex Es gibt einen Weihnachtsmann.
\ex Es trug ihn aus der Kurve. \citep{Uszkoreit87a}
\ex Es schüttelte mich vor Ekel.
\ex Er hat es weit / zum Professor gebracht. \citep{Puetz82a}
\ex Er nimmt es mit zehn Gegnern auf.
%\item Beim Hören diese Gitarrensolos lief es mir kalt den Rücken hinunter.
%\item Es kam zu Handgreiflichkeiten.
\zl

auch {\em das\/}:
\ea
Das regnet ja wieder.
\z
}

\frame{
\frametitle{Abkürzungen}


\begin{tabular}{@{}lp{3.5cm}@{~~~}l@{}p{4cm}@{}}
NP$_{[3,sg,fem]}$     & \onems{ cat \onems{ head \type{noun} \\
                                            subcat \liste{} \\
                                          } \\
                                cont$|$ind \ms{ per & 3 \\
                                                num & sg \\
                                                gen & fem \\
                                              } \\
                              }  &
NP$_{\ibox{1}}$ & \ms{ cat & \onems{ head \type{noun} \\
		                     subcat \liste{} \\
                                   } \\
                       cont & \ms
                              { ind & \ibox{1} {\it ref\/} \\
                              } \\
                     }\\\\
NP$_{expl}$ & \ms{ cat & \ms{ head & noun \\
                              subcat & \liste{} \\
                            } \\
                   cont & \ms{ ind & \ms[expl]{} \\
                             }\\
                 } &
\is{:}%
\baro{N}: \ibox{1} & \ms{ cat & \ms{ head & noun \\
                                     subcat & \sliste{ det} \\
                                   } \\
                          cont & \ibox{1} \\
                        } \\
\end{tabular}

}
\end{comment}
\frame{
\frametitle{Abkürzungen}


\begin{tabular}{@{}lp{3.5cm}@{\hspace{14mm}}l@{}p{4cm}@{}}
NP$_{[3,sg,fem]}$     & \onems{ cat \onems{ head \type{noun} \\
                                            subcat \liste{} \\
                                          } \\
                                cont$|$ind \ms{ per & 3 \\
                                                num & sg \\
                                                gen & fem \\
                                              } \\
                              }  &
\pause
NP$_{\ibox{1}}$ & \ms{ cat & \onems{ head \type{noun} \\
		                     subcat \liste{} \\
                                   } \\
                       cont & \ms
                              { ind & \ibox{1}\\
                              } \\
                     }\\\\
\pause
\baro{N}: \ibox{1} & \ms{ cat & \ms{ head & noun \\
                                     subcat & \liste{ det} \\
                                   } \\
                          cont & \ibox{1} \\
                        } \\
\end{tabular}

}

\subsection{Repräsentation parametrisierter Sachverhaltet mit Merkmalstrukturen}

\frame[shrink]{
\frametitle{Sachverhalte und Repräsentation mit Merkmalstrukturen}


$\ll schlagen, agens:X, patiens:Y\gg$\\
$X|\ll mann, instance:X\gg,$\\
$Y|\ll hund, instance:Y\gg$\\
\bigskip
\ms[schlagen]
{ 
 agens & \ibox{1} \\
 patiens & \ibox{2} \\
}
\medskip

\ms{
ind & \ibox{1} \ms{ per & 3 \\
                    num & sg \\
                    gen & mas \\
                  } \\
restr & \liste{ \ms[mann]{ 
                inst & \ibox{1} \\
                }
        }\\
}
\hspace{1.5cm}
\ms{
ind & \ibox{2} \ms{ per & 3 \\
                    num & sg \\
                    gen & mas \\
                  }\\
restr & \liste{ \ms[hund]{
                inst & \ibox{2} \\
                }
        } \\
}

}

\subsection{Linking}

\frame{
\frametitle{Repräsentation in Merkmalsbeschreibungen und Linking}

\begin{itemize}
\item Linking zwischen Valenz und semantischem Beitrag\\

\psset{nodesep=1pt}

\label{le-geben}
\mbox{{\it gibt\/} (finite Form):}\\
\ms
{ cat & \ms{ head & \ms[verb]
                     { vform & fin \\} \\
             subcat & \liste{ NP[{\it nom\/}]\rnode{11}{\ind{1}}, NP[{\it acc}]\rnode{21}{\ind{2}}, NP[{\it dat}]\rnode{31}{\ind{3}}   } \\
           } \\
  cont &  \ms[geben]{ agens & \rnode{12}{\ibox{1}} \\
                      thema & \rnode{22}{\ibox{2}} \\
                      goal  & \rnode{32}{\ibox{3}} \\
           } \\
}
\pause
\item Die referentiellen Indizes der Nominalphrasen sind mit den semantischen Rollen identifiziert.

\rot<beamer>{
% \aanodeconnect[bl]{11}[r]{12}
% \aanodeconnect[bl]{21}[r]{22}
% \aanodeconnect[bl]{31}[r]{32}
\ncline{<->}{11}{12}
\ncline{<->}{21}{22}
\ncline{<->}{31}{32}
}
\end{itemize}

}

\iftoggle{teil1}{
\frame{

\frametitle{Generalisierungen für Verbklassen}

\begin{itemize}
\item typbasiert: Verben mit Agens, mit Agens und Thema, mit Agens und Patiens
\item verschiedene Valenz/Linking-Muster\\
\ea
\onems
{ cat$|$subcat \liste{ []\ind{1}, []\ind{2}, []\ind{3}  }\\[1mm]
  cont  \ms[agens-thema-goal-rel~]{ agens & \ibox{1} \\
                      thema & \ibox{2} \\
                      goal  & \ibox{3} \\
           } \\
}
\z
\pause
\item Der Typ für die Relation {\it geben\/} ist Untertyp von {\it agens-thema-goal-rel\/}.\\
      Lexikoneintrag für \stem{geb} hat das Linking-Muster in (\mex{0}).
\pause
\item Generalisierungen darüber, wie welche Argumente realisiert werden können,
      lassen sich ebenfalls erfassen.
%\item zu Linkingtheorien in HPSG siehe 
\nocite{Davis96a-u,Wechsler91a-u}
\end{itemize}
}
} % \end{teil1}

\subsection{Der semantische Beitrag von Phrasen}
\frame{

\frametitle{Projektion des semantischen Beitrags des Kopfes}

\centering%
\scalebox{0.75}{\begin{tabular}[t]{@{}c@{\hspace{4mm}}c@{\hspace{4mm}}c@{\hspace{4mm}}c@{\hspace{4mm}}c}%llll}
\multicolumn{3}{l}{\rnode{1}{V[\begin{tabular}[t]{@{}l}
  {\it fin\/}, {\sc subcat} \sliste{ }]\\
  \end{tabular}}
}\\
\\
\hspace{2mm}C                     & \hspace{15mm}H\\*[3ex]
\rnode{2}{\ibox{1} NP[{\it nom\/}]} & \multicolumn{2}{l}{\rnode{3}{V[\begin{tabular}[t]{@{}l}
                                                                    {\it fin\/}, {\sc subcat} \sliste{ \ibox{1} }]\\
                                                                    \end{tabular}}
                                                       }\\
\\
& \hspace{5mm}C                     & \hspace{18mm}H\\*[3ex]
& \rnode{4}{\ibox{2} NP[{\it acc\/}]} & \multicolumn{2}{l}{\rnode{5}{V[\begin{tabular}[t]{@{}l}
                                                                    {\it fin\/}, {\sc subcat} \sliste{ \ibox{1}, \ibox{2} }]\\
                                                                    \end{tabular}}
                                                       }\\
\\
&                                   & \hspace{4mm}C                     & \hspace{-5mm}H & \hspace{2mm} $geben(e,b,m)$\\*[3ex]
&                                   & \rnode{6}{\ibox{3} NP[{\it dat\/}]}& \rnode{7}{V[\begin{tabular}[t]{@{}l}
                                                                                   {\it fin\/}, {\sc subcat} \sliste{ \ibox{1}, \ibox{2}, \ibox{3} }]\\
%                                                                                   {\sc vcomp} \sliste{  }, \\
                                                                                   \end{tabular}}\\
\\*[6ex]
\rnode{8}{er} & d\rnode{9}{as Buc}h                     & d\rnode{10}{em Man}n              & \rnode{11}{gibt}\\
\end{tabular}
\ncline{1}{2}\ncline{1}{3}%
\ncline{3}{4}\ncline{3}{5}%
\ncline{5}{6}\ncline{5}{7}%
\ncline{7}{11}%
\ncline{2}{8}%
%\nodetriangle{6}{10}%
%\nodetriangle{4}{9}%
%\aanodecurve[r]{7}[r]{5}{2cm}%
%\anodecurve[r]{5}[r]{3}{2cm}%
%\anodecurve[r]{3}[r]{1}{2cm}%
}

}

\iftoggle{teil1}{

\frame{
\frametitle{Semantikprinzip (Ausschnitt)}


% In Strukturen mit Kopf, die keine Kopf-Adjunkt-Strukturen sind, ist der
% semantische Beitrag der Mutter identisch mit dem der Kopftochter.

In Strukturen, in denen es eine Kopftochter gibt,\\
ist der semantische Beitrag der Mutter identisch mit dem der Kopftochter.

\(
\ms{
cont & \ibox{1} \\
head-dtr$|$cont & \ibox{1} \\
}
\)

Anmerkung:\\
Diese Beschränkung gilt nicht für Kopf-Adjunkt-Strukturen.
Kopf-Adjunkt-Strukturen werden später behandelt.

}
} % \end{teil1}

\iftoggle{teil2}{
\frame{

\frametitle{Semantikprinzip (Ausschnitt)}


% In Strukturen mit Kopf, die keine Kopf-Adjunkt-Strukturen sind, ist der
% semantische Beitrag der Mutter identisch mit dem der Kopftochter.

In Strukturen, in denen es eine Kopftochter gibt, ist der
semantische Beitrag der Mutter identisch mit dem der Kopftochter.

\(
\ms{
cont & \ibox{1} \\
head-dtr$|$cont & \ibox{1} \\
}
\)

%% Anmerkung:\\
%% Diese Beschränkung gilt nicht für Kopf-Adjunkt-Strukturen.
%% Kopf-Adjunkt-Strukturen werden später behandelt.

}
} % \end{teil2}


\iftoggle{teil1}{
\frame{
\frametitle{Typhierarchie für {\it sign\/}}


\centerline{
\begin{forest}
type hierarchy
[sign
  [word]
  [phrase
    [non-headed-phrase]
    [headed-phrase
      [\ldots]
      [head-non-adjunct-phrase
        [head-argument-phrase]]
      [\ldots]]]]
\end{forest}}


}
} % \end{teil1}

\begin{comment}
\frame{
\frametitle{Typhierarchie für {\it sign\/} (Überblick)}



\epsfxsize=0.9\textwidth
\centerline{\mbox{\epsffile{types-sign-hc-hna.eps}}}


\centerline{Betrachteter Ausschnitt {\bf fett} markiert}


}
\end{comment}


\iftoggle{teil1}{
\frame{
\frametitle{Kopf-Komplement-Schema + HFP + SemP}


\onems[head-argument-phrase~]{
cat  \ms{ head   & \ibox{1} \\
           subcat & \ibox{2} \\ 
         }\\
\blau{cont  \ibox{3}}\\[2mm]
head-dtr   \ms{ cat & \ms{ head   & \ibox{1} \\
                            subcat & \ibox{2} $\oplus$ \sliste{ \ibox{4} } \\
                          } \\
                 \blau{cont} & \blau{\ibox{3}} \\
               }\\
non-head-dtrs  \sliste{ \ibox{4} } \\
} 

Typ {\it head-argument-phrase\/} mit von {\it headed-phrase\/} und 
{\it head-non-adjunct-phrase\/} ererbter Information


}

\if 0
\frame{

\frametitle{Perkolation der Indizes}


\begin{itemize}
\item Bisher ist nur Semantik des Kopfes am obersten Knoten repräsentiert.
\pause
\item Jede NP führt einen Index/Quantor ein.
\scalebox{0.85}{\onems
{   cont \blau<1>{\ibox{1}} \ms{ ind & \ibox{2} \ms{  per & 3 \\
                                            num & sg \\
                                            gen & mas \\
                                         } \\
                       restr & \liste{\ms[mann]{
                                      instance & \ibox{2} \\
                                      }} \\
                              } \\
   \blau<1>{context$|$inds   {\rm \sliste{ \ibox{1} }}} \\
 }
}
\pause
\item Indizes werden projiziert.

\type{headed-phrase} $\to$\\
\scalebox{0.85}{\onems{
context$|$inds  \ibox{1} $\oplus$ \ibox{2} \\
head-dtr$|$context$|$inds \ibox{1} \\
non-head-dtrs \liste{ \onems{
                       context$|$inds \ibox{2} \\
                      }}\\
}
}

\end{itemize}

}
\fi

} % \end{teil1}

\subsection{Übungsaufgaben}

\frame{
\frametitle{Übungsaufgaben}

\begin{enumerate}
\item Wie kann man den semantischen Beitrag von \emph{lacht} repräsentieren?
\item Geben Sie eine Merkmalstruktur für \emph{er lacht} in (\mex{1}) an:
      \ea
      {}[dass] er lacht
      \z
\end{enumerate}

}
