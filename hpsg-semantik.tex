
\section{Semantik}

\subtitle{Semantik}

\huberlintitlepage[22pt]




\iftoggle{teil1}{
\outline{

\begin{itemize}
\item Wozu Syntax? / Phrasenstrukturgrammatiken
\item Formalismus
\item Valenz und Grammatikregeln
\item Dominanzstrukturen und Prinzipien
\item \blaubf{Semantik}
\item Adjunktion und Spezifikation
\item Das Lexikon: Typen und Lexikonregeln
\item Topologie des deutschen Satzes
\item Konstituentenreihenfolge
\item Nichtlokale Abhängigkeiten
\item Relativsätze
\item Lokalität
%\item Komplexe Prädikate: Der Verbalkomplex
\end{itemize}
}
} % \end{teil1}

\iftoggle{teil2}{
\outline{
\begin{itemize}
\item Wiederholung
      \begin{itemize}
\item Wozu Syntax? / Phrasenstrukturgrammatiken
\item Formalismus
\item Valenz und Grammatikregeln
\item Dominanzstrukturen und Prinzipien
\item \blau{Semantik}
\item Topologie des deutschen Satzes
\item Konstituentenreihenfolge
\item Nichtlokale Abhängigkeiten
\end{itemize}
\item Kongruenz
\item Kasus
\item Der Verbalkomplex
\item Kohärenz, Inkohärenz, Anhebung und Kontrolle
\item Passiv
\item Partikelverben
\item Morphologie
\end{itemize}
}
} % \end{teil2}

\frame{
\frametitle{Literaturhinweis}
%
\begin{itemize}
\item Lesestoff: \citew[Kapitel~5]{MuellerLehrbuch3}
\item Außerdem: 
\begin{itemize}
\item Überblickskapitel Semantik im HPSG-Handbuch \citep{KoenigRichter2021a}
\item Überblickskapitel Linking im HPSG-Handbuch \citep*{DKW2021a}
\end{itemize}
\end{itemize}

% \vspace{1cm}

% \rotbf{Achtung, wichtiger Hinweis: Diese Literaturangabe hier bedeutet,\\dass Sie die Literatur zum
%    nächsten Mal lesen sollen!!!!}
}

\subsection{Ansätze}

\iftoggle{teil1}{
\frame{
\frametitle{\hypertarget{semantics-t}{Semantik}: Überblick über verwendete Ansätze}

\begin{itemize}
\item \citet{ps} und \citet{GSag2000a-u} nehmen Situationssemantik an
      \citep*{BP83a,CMP90,Devlin92}\nocite{BP87a}.
\pause
\item Ab den frühen 1990ern gab es Arbeiten zur Unterspezifikationssemantik
\citep*{Nerbonne93a,CFMRS95a-u}.
\pause
\item \emph{Discourse Representation Theory} (DRT, \citealt{KR93a}) mit Underspecified Discourse Representation Structures (UDRS, \citealt{FR95a-u})
\pause
\item aktuellere Arbeiten verwenden:
\begin{itemize}
\item \emph{Minimal Recursion Semantics} (MRS; \citealt*{CFPS2005a})
\item \emph{Lexical Resource Semantics} (LRS; \citealt*{RichterSailer2004a-u})
\end{itemize}
\pause
\item Im Folgenden werden wir MRS nutzen.
\end{itemize}

}
} % \end{teil1}

\iftoggle{teil1}{

\subsection{Minimal Recursion Semantics}


\frame{
\frametitle{Warum MRS? Warum Unterspezifikation?}

\begin{itemize}
\item Mit MRS kann man Formeln beschreiben.
\pause
\item Beschreibungen können unspezifiziert und dadurch mehrdeutig sein.
\pause
\item Man muss die Mehrdeutigkeiten dann nicht ausbuchstabieren.
\pause
\item Beispiele für Mehrdeutigkeiten auf verschiedenen Sprachebenen:
\eal
\ex Spracherkennung
\ex Unbekannte haben Mittwochabend bei einer %FDP-Wahlkampfveranstaltung\\
FDP-Wahlkampfveranstaltung mit FDP-Chef Guido Westerwelle Farbbeutel geworfen.\footnote{
taz, 21.5.2004, S.\,7
}
\ex
\label{ex-Jede-Tochter-eines-Mitarbeiters-schläft}
Jede Tochter eines Mitarbeiters schläft.
\zl 

\end{itemize}




}

\frame{
\frametitle{Massive Mehrdeutigkeit durch Quantoren}

Ca.\ 200 Lesarten:
\ea
Many people feel that most sentences exhibit too few quantifier scope ambiguities for much effort to be devoted
to this problem, but a casual inspection of several sentences from any text should convince almost everyone
otherwise.\footnote{%
  Aus dem Haupttext von \citew[\page 57]{Hobbs1983a-u}.
}
\z



}


\frame{
\frametitle{Relationen und semantische Rollen}

\begin{itemize}
\item Die Flexion von Wörtern ist für die Bedeutung irrelevant:
\eal
\label{ex-Alle-Delfine}
\ex Jeder Delfin liebt ein Känguru.
\ex Alle Delfine lieben ein Känguru.
\zl
Deshalb einheitliche Form, die Nennform, für semantische Relationen.

\begin{itemize}
\item einstellig:  \relation{regnen}   (\emph{Es regnet.}) 
\item zweistellig: \relation{sterben}  (\emph{Aicke stirbt.})
\item dreistellig: \relation{lieben}   (\emph{Aicke liebt Conny.})
\item vierstellig: \relation{geben}    (\emph{Aicke gibt Conny den Aufsatz.})
\item fünfstellig: \relation{kaufen}   (\emph{Aicke kauft den gebrauchten Mantel vom Conny für fünf Euro.})
\end{itemize}

\item $e$ für Ereignis:
\eal
\ex Aicke liebt Conny.
\ex \relation{lieben}(e, Aicke$'$, Conny$'$)
\zl

\end{itemize}

}

\frame{
\frametitle{Diskursuniversum}

Individuen können Individuenkonstanten zugeordnet sein:

	\begin{tikzpicture}[scale=0.75]
		\draw (0,0) rectangle (8,5);
		%
		\node[anchor=west] at (1.5,0.5) {Kirby};
		\node[anchor=west] at (2.3,1.7) {Aicke};
		\node[anchor=east] at (1.8,3) {sweety};
		\node[anchor=west] at (3.5,3.5) {Conny};
		%
		\draw (1,1) -- (1.6,0.7); %Kirby
		\draw (2,2) -- (2.4,1.9); %Aicke
		\draw (2,2) -- (1.4,2.7); %sweety
		\draw (3,3) -- (3.6,3.3); %Conny
		%
		\draw[fill=fugreen, draw=black] (1,1) circle (0.1);
		\draw[fill=fugreen, draw=black] (2,2) circle (0.1);
		\draw[fill=fugreen, draw=black] (3,3) circle (0.1);
		\draw[fill=fugreen, draw=black] (4,2) circle (0.1);
		\draw[fill=fugreen, draw=black] (7,4) circle (0.1);
	\end{tikzpicture}

}

\frame{
\frametitle{Diskursuniversum mit der Menge der blonden Individuen}


	\begin{tikzpicture}[scale=0.75]
		\draw (0,0) rectangle (8,5);
		\draw (1.5,1.5) circle (1);
		\draw[fill=fugreen, draw=black] (1,1) circle (0.1);
		\draw[fill=fugreen, draw=black] (2,2) circle (0.1);
		\draw[fill=fugreen, draw=black] (3,3) circle (0.1);
		\draw[fill=fugreen, draw=black] (4,2) circle (0.1);
		\draw[fill=fugreen, draw=black] (7,4) circle (0.1);
		%
		\node[anchor=west] at (3,0.5) {blond};
		\draw (2.3,0.9) -- (3,0.7);
	\end{tikzpicture}


}

\frame{
\frametitle{Blonde Personen und Kinder}

	\begin{tikzpicture}[scale=0.75]
		\draw (0,0) rectangle (8,5);
		% 
		\begin{scope}
			\clip (3,2) circle (1.5);
			\fill[pattern=north east lines] (1.5,1.5) circle (1);
		\end{scope}
		% 
		\draw (1.5,1.5) circle (1);
		\draw (3,2) circle (1.5);
		% 
		\draw[fill=fugreen, draw=black] (1,1) circle (0.1);
		\draw[fill=fugreen, draw=black] (2,2) circle (0.1);
		\draw[fill=fugreen, draw=black] (3,3) circle (0.1);
		\draw[fill=fugreen, draw=black] (4,2) circle (0.1);
		\draw[fill=fugreen, draw=black] (7,4) circle (0.1);
		% 
		\node[anchor=east] at (1.5,3.5) {blond};
		\draw (0.8,3.3) -- (1.1,2.4);
		% 
		\node[anchor=west] at (5,3) {Kind};
		\draw (4.1,3) -- (5,3);
	\end{tikzpicture}

}

\frame{
\frametitle{Schnittmenge aus Blonden, Schlafenden und Kindern}

Beide Sätze im vorliegenden Diskursuniversum wahr.
\eal
\ex Alle blonden Kinder schlafen.
\ex Ein blondes Kind schläft.
\zl

\begin{tikzpicture}[scale=0.75]
		\draw (0,0) rectangle (8,5);
		\draw (1.5,1.5) circle (1); %klug
		\draw (3,2) circle (1.5);  %Kind
		\draw[rotate around={50:(2.4,2.4)}] (2.5,2.5) ellipse (1.5 and 0.9); %schlafen 
		%
		\begin{scope}
			\clip (1.5,1.5) circle (1);
			\clip (3,2) circle (1.5);
			\clip[rotate around={50:(2.4,2.4)}] (2.5,2.5) ellipse (1.5 and 0.9);
			\fill[pattern=north east lines] (0,0) rectangle (8,5);
		\end{scope}
		%
		\draw[fill=fugreen, draw=black] (1,1) circle (0.1);
		\draw[fill=fugreen, draw=black] (2,2) circle (0.1);
		\draw[fill=fugreen, draw=black] (3,3) circle (0.1);
		\draw[fill=fugreen, draw=black] (4,2) circle (0.1);
		\draw[fill=fugreen, draw=black] (7,4) circle (0.1);
		%
		\node[anchor=east] at (1.5,3.5) {blond};
		\draw (0.8,3.3) -- (1.1,2.4);
		%
		\node[anchor=west] at (5,3) {Kind};
		\draw (4.1,3) -- (5,3);
		%
		\node[anchor=west] at (3,4.4) {schlafen};
		\draw (2.8,3.8) -- (3,4.3);
	\end{tikzpicture}



}

\frame{
\frametitle{Was muss der Fall sein, damit ein Satz wahr wird?}

\eal
\label{ex-Alle-Kinder-schlafen}
\ex Alle Kinder schlafen.
\ex Für alle x, für die gilt, dass sie Kind sind, gilt auch, dass sie schlafen.
\ex \blau{$\forall$}x kind(x) \blau{$\to$} schlafen(x)
\zl
\pause
\eal
\ex Ein Kind schläft.
\ex Es gibt mindestens ein Kind und für dieses Kind gilt, dass es schläft.
\ex \blau{$\exists$}x kind(x) \blau{$\wedge$} schlafen(x)
\zl
\pause
\eal
\label{ex-Das-Kind-schläft}
\ex Das Kind schläft.
\ex Es gibt ein (bestimmtes) Kind und für dieses Kind gilt, dass es schläft.
\ex \blau{$\iota$}x kind(x) \blau{$\wedge$} schlafen(x)
\zl

}

\frame{% [shrink=10]{
\frametitle{Repräsentation von Relationen mit Merkmalbeschreibungen}

%\small
Möglichkeit:
\ea
\ms[schlafen]{
 event & e\\ 
 agens & x \\
}
\z

Semantische Rollen: \textsc{agens}, 
\textsc{patiens}, 
\textsc{experiencer}, 
\textsc{source}, 
\textsc{goal}, 
\textsc{thema}, \ldots
% \textsc{location}, 
% \textsc{trans-obj}, 
% \textsc{instrument}, 
% \textsc{means} und 
% \textsc{proposition}.

\pause

\oneline{Generalisierungen: Passiv unterdrückt Agens-Rolle \parencites{Haider86}{Wunderlich87c}.}

\pause

Aber:
\ea
eine radioaktive Wolke, die vom Wind nach Skandinavien getrieben wurde\footnote{
  Mannheimer Morgen, 30.04.1986, S.\,1. Zitiert nach \citew[126]{Mueller2002b}.
}
\z
Ist der Wind ein Agens?
\pause

Ist Subjekt von \emph{sehen} nicht Experiencer? \parencites{Devlin92}{Dowty2000a}

\ea
Der Einbrecher wurde am frühen Morgen gesehen.
\z

}

\frame{
\frametitle{Kodierung mit \argzero, \argone, \ldots}


Besser: 
\begin{itemize}
\item Rollen wie Proto"=Agens und Proto"=Patiens \citet{Dowty91a} oder 
\item Actor und Undergoer \citep*{VanVL97a-u} %Davis2001a-u,KD2003a-u,DKW2024a,
\item Oder ganz radikal \argzero, \argone, \argtwo \citep{mrs}
\end{itemize}
\pause



schlafen(e, x), helfen(e, x, y) und kleinkind(x):
\ea
\ms[schlafen~]{ 
 arg0 & e \\
 arg1 & x \\
}\hspace{2em}
\ms[helfen]{ 
 arg0 & e \\
 arg1 & x \\
 arg2 & y \\
}
\hspace{2em}
\ms[kleinkind]{ 
 arg0 & x \\
}
\z


}

\subsection{Repräsentation des \contwes}

\frame[shrink]{
\frametitlefit{Repräsentation mit Merkmalsbeschreibungen: der \contw}

\begin{itemize}
\item mögliche Datenstruktur (\textsc{cont} = \textsc{content}):\\
      \ms{ phon   & list~of~phoneme strings\\
           \blau{head}   & \blau{head}\\
           \blau{spr}    & \blau{list}\\
           \blau{comps}  & \blau{list}\\
           \blau{arg-st} & \blau{list}\\
           cont   & cont\\
         }\\
\pause
\item stärkere Gliederung, Unterteilung in syntaktische und semantische Information
      ({\sc cat} = {\sc category})\\
      \ms{ phon & list~of~phoneme strings\\
           cat  & \blau{\ms[cat]{ head & head\\
                            spr  & list\\
                            comps & list\\
                            arg-st & list\\
                          }} \\
           cont & cont\\
         }
\item $\to$ möglich, nur syntaktische Information zu teilen
\end{itemize}
}

\frame{
\frametitle{Teilung syntaktischer Information in Koordinationen}

\begin{itemize}
\item symmetrische Koordination: der \catw ist identisch
      \ms{ phon & list~of~phoneme strings\\
           cat  & \ms[cat]{ head   & head\\
                            spr  & list\\
                            comps & list\\
                            arg-st & list\\
                          } \\
           cont & cont\\
         }
\item Beispiele:
\eal
\ex {}[der Delfin] und [der Hund]
\ex Aicke [kennt] und [liebt] dieses Lied.
\ex Der Delphin ist [schnell] und [wendig].
\zl
\end{itemize}
}


\subsection{Nominale Objekte}
\frame{
\frametitle{Semantischer Beitrag nominaler Objekte}

\small
\begin{itemize}
\item semantischer Index + zugehörige Restriktionen

Lexikoneintrag für \emph{Buch}:\\
\ms{ 
  cat & \ms{ head & noun\\
             spr   & \nliste{ Det } \\
             comps & \eliste \\
           } \\
  cont &  \ms
           { ind & \ibox{1} \ms{ per & 3 \\
                                 num & sg \\
                                 gen & neu \\
                               } \\
             rels & \liste{ \ms[buch]{ arg0 & \ibox{1} \\ }} \\
           } \\
}
\pause
\item Person, Numerus und Genus sind für die Bestimmung der Referenz/Koreferenz wichtig:
      \ea
      Die Frau$_i$ kauft ein Buch$_j$. Sie$_i$ liest es$_j$.
      \z
\end{itemize}

}



\frame{
\frametitle{Abkürzungen}


\begin{tabular}{@{}lp{3.5cm}@{\hspace{14mm}}lp{4cm}@{}}
NP$_{[3,sg,fem]}$     & \onems{ cat \onems{ head \type{noun} \\
                                           spr \eliste \\
                                           comps \eliste \\
                                          } \\
                                cont$|$ind \ms{ per & 3 \\
                                                num & sg \\
                                                gen & fem \\
                                              } \\
                              }  &
\pause
NP$_{\ibox{1}}$ & \ms{ cat & \onems{ head \type{noun} \\
                                           spr \eliste \\
                                           comps \eliste \\
                                   } \\
                       cont & \ms
                              { ind & \ibox{1}\\
                              } \\
                     }\\\\
\pause
\baro{N}: \ibox{1} & \ms{ cat & \ms{ head & noun \\
                                     spr & \nliste{ Det } \\
                                     comps \eliste \\
                                   } \\
                          cont & \ibox{1} \\
                        } \\
\end{tabular}

}


\subsection{Der semantische Beitrag von Verben und Linking}
\label{sec-Linking}

\frame{
\frametitle{Der semantische Beitrag von Verben und Linking}

\eas
\label{le-gibt}%
\emph{gibt}:\\
\ms
{ cat & \ms{ head & \ms[verb]
                     { vform & fin \\} \\
             arg-st & \liste{ NP[\type{nom}]\tikzmarknode{i1}{\ind{1}}, NP[\type{dat}]\tikzmarknode{i2}{\ind{2}}, NP[\type{acc}]\tikzmarknode{i3}{\ind{3}}   } \\
           } \\
  cont &  \ms{
          ind & \ibox{4} event\\
          rels & \liste{ \ms[geben]{ 
                          arg0 & \ibox{4}\\
                          arg1 & \tikzmarknode{r1}{\ibox{1}} \\
                          arg2 & \tikzmarknode{r2}{\ibox{2}} \\
                          arg3 & \tikzmarknode{r3}{\ibox{3}} \\
                          } } }\\
}
%
\visible<2->{
\begin{tikzpicture}[remember picture,overlay,<->]
\draw[brown] (i1) to (r1);
\draw[brown] (i2) to (r2);
\draw[brown] (i3.south west) to (r3);
\end{tikzpicture}
\zs
}

\visible<3>{
No more $\lambda$s! Die $\lambda$s sind in \argst bzw.\ den Valenzmerkmalen.
}

}


\frame{
\frametitle{Linking allgemein}

\eas
\onems
{ cat|arg-st \liste{ []\tikzmarknode{i1}{\ind{1}}, []\tikzmarknode{i2}{\ind{2}}, []\tikzmarknode{i3}{\ind{3}}   } \\
  cont|rels  \liste{ \ms{ 
%                          arg0 & \ibox{4}\\
                          arg1 & \tikzmarknode{r1}{\ibox{1}} \\
                          arg2 & \tikzmarknode{r2}{\ibox{2}} \\
                          arg3 & \tikzmarknode{r3}{\ibox{3}} \\
                          } }\\
}
%
% \begin{tikzpicture}[remember picture,overlay,<->]
% \draw[brown] (i1) to (r1);
% \draw[brown] (i2) to (r2);
% \draw[brown] (i3.south west) to (r3);
% \end{tikzpicture}
\zs


Linking"=Theorien im Rahmen der HPSG: \citew{Davis96a-u,Wechsler95a-u}

Überblick: \citew{HPSGHandbookLinking}.


}

\subsection{Komposition}

\frame{
\frametitle{Komposition}

\ea
Der kleine Affe schläft.
\z

\begin{forest}
sm edges
[{V$_e$[\nliste{ def\_q(x, \ldots), klein(x), affe(x), schlafen(x) }]}
  [{NP$_x$[\nliste{ def\_q(x, \ldots), klein(x), affe(x) }]}
    [{Det$_x$[\nliste{ def\_q(x, \ldots) }]} [der]]
    [{N$_x$[\nliste{ klein(x), affe(x) }]}
      [{Adj$_x$[\nliste{ klein(x) }]} [kleine]]
      [{N$_x$[\nliste{ affe(x) }]} [Affe]]]]
  [{V$_e$[\nliste{ schlafen(e,x) }]} [schläft]]]
\end{forest}

\vfill

Index wird hochgegeben: x bei nominalen Projektionen, e bei verbalen

\rels werden einfach verknüpft.

\vfill

}

\frame{
\frametitle{Index und Relationen}

Weitergabe des semantischen Indexes:

\ea
\label{Semantik-Index-headed-phrase}
\type{headed-phrase} \impl\\
\ms{
cont|ind & \ibox{1}\\
head-dtr|cont|ind & \ibox{1}\\
}
\z

\pause

\rels für Phrasen:
\ea
\label{constraint-RELS}
\type{phrase} \impl\\
\onems{
cont|rels  \ibox{1} $\oplus$ \ibox{2}\\
dtrs  \sliste{ [cont|rels \ibox{1} ], [cont|rels \ibox{2} ] }\\
}
\z

}

\frame{
\frametitle{Allgemeinere Berechnung des \relswes}


Ganz allgemein, unabhängig von der Anzahl der Töchter.
\ea
\type{phrase} \impl\\
\onems{
cont|rels  \texttt{collect-rels}(\ibox{1})\\
dtrs  \ibox{1} \\
}
\z

\pause

\ea
\label{constraint-collect-rels}
\texttt{collect-rels}([],[]).\\
\texttt{collect-rels}([[\textsc{cont|rels} Rels1]|Dtrs],Rels) :=\\
\hfill \texttt{collect-rels}(Dtrs,Rels2), \texttt{append}(Rels1,Rels2,Rels).
\z

}

\subsection{Quantoren}

\frame{
\frametitle{Quantoren: Notation}


\eal
\label{ex-Alle-Kinder-schlafen-zwei}
\ex Alle Kinder schlafen.
\ex $\forall$x \relationunmarked{kind}(x) $\to$ \relationunmarked{schlafen}(x)
\ex Für jede Entität x in unserem Diskursuniversum, für die gilt, dass sie ein Kind ist, gilt auch,
dass sie schläft.
\zl

Notation für Quant, die 
\ea
\label{ex-every-kind-schlafen}
every\_q(x, kind(x), schlafen(x))
\z


}

\frame{
\frametitle{Töchter und Mitarbeiter}


\ea
\label{ex-Jede-Tochter-eines-Mitarbeiters-schläft-zwei}
Jede Tochter eines Mitarbeiters schläft.
\z

\pause

\eal
\ex $\forall$x ($\exists$y \relationunmarked{mitarbeiter}(y) $\wedge$ \relationunmarked{tochter}(x,y)) $\to$
\relationunmarked{schlafen}(x)
\ex Für jede Entität x in unserem Diskursuniversum für die gilt, dass es im Diskursuniversum eine Entität y gibt, die
Mitarbeiter ist und deren Tochter x ist, gilt auch, dass sie (x) schläft.
\zl

\pause

\eal
\ex $\exists$y \relationunmarked{mitarbeiter}(y) $\wedge$ ($\forall$x \relationunmarked{tochter}(x,y) $\to$
\relationunmarked{schlafen}(x))
\ex Es gibt im Diskursuniversum eine Entität y, für die gilt, dass sie ein Mitarbeiter ist und dass für alle Entitäten
x im Diskursuniversum, für die gilt, dass sie die Tochter von y sind, auch gilt, dass sie schlafen.
\zl

\pause

\eal
\label{ex-Quantoren-Mitarbeiter-Tochter}
\ex every\_q(x, exist\_q(y, mitarbeiter(y), tochter(x,y)), schlafen(x))
\ex exist\_q(y, mitarbeiter(y), every\_q(x, tochter(x,y), schlafen(x)))
\zl 

}

\frame{
\frametitle{Wohlgeformtheitsbedingungen und Unterspzifikation}

Variablen müssen von Quantor gebunden werden.

(\mex{1}) nicht wohlgeformt, weil das
$x$ von schlafen(x) außerhalb des Skopus des Allquantors ist, der x einführt und binden muss. \citep[131]{Kratzer95a}.
\ea
exist\_q(y, \rot{every\_q(x, tochter(x,y), mitarbeiter(y))},  schlafen(\rot{x}))
\z

\pause

Man kann nun Beschreibungen entwickeln, die die Skopusbeziehungen unterspezifiziert lassen.

\eal
\label{ex-Quantoren-Mitarbeiter-Tochter}
\ex \blau<2>{every\_q(x, \rot<3>{exist\_q(y, mitarbeiter(y)}, tochter(x,y))}, schlafen(x))
\ex \rot<3>{exist\_q(y, mitarbeiter(y)}, \blau<2>{every\_q(x, tochter(x,y)}, schlafen(x)))
\zl 

tochter(x,y) muss sich in der Restriktion des Allquantors befinden.

\pause
mitarbeiter(y) in der Restriktion des Existenzquantors.

}

\frame{
\frametitle{Handels und Labels}



\eal
\ex $\forall$x \relationunmarked{kind}(x) $\to$ \relationunmarked{schlafen}(x)
\ex every\_q(x, kind(x), schlafen(x))
\ex h1:every\_q(x, h2, h3), h2:kind(x), h3:schlafen(x))
\zl

Nicht die Ausdrücke selbst werden beim Quantor eingesetzt,\\
sondern Zeiger auf diese.


}


\frame{
\frametitle{Konjunktion ist Identifikation von Labels}

\eal
\ex Alle kleinen Kinder schlafen.
\ex\label{ex-Aussagenlogik-Alle kleinen Kinder schlafen}
$\forall$x (\relationunmarked{klein}(x) $\wedge$ \relationunmarked{kind}(x)) $\to$ \relationunmarked{schlafen}(x)
\ex Für jede Entität x in unserem Diskursuniversum, für die gilt,\\
    dass sie klein und ein Kind ist, gilt auch, dass sie schläft.
\zl

\pause

(\mex{1}a) oder (\mex{1}b)?
\eal
\ex h1:every\_q(x, h2, h3), h2:und(h4:klein(x), h5:kind(x)), h3:schlafen(x)
\ex h1:every\_q(x, h2, h3), h2:klein(x), h2:kind(x), h3:schlafen(x)
\zl

\citet[\page 284]{CFPS2005a} Schachtelungen tragen nicht zur Bedeutung bei. Teilen der Labels reicht.

}

\frame{
\frametitle{Grafische Darstellung der Dominanz}


\centerline{%
\begin{tikzpicture}[scoped mrs]
\node(every)    at (2,6)   {h1:every\_q(\subnode{x}{x}, \subnode{h2}{h2}, \subnode{h3}{h3})};
\node(kind)     at (0.5,4) {h2:klein(x), h2:kind(x)};
\node(schlafen) at (4,4)   {h3:schlafen(x)};
%
\draw (h2) to (kind);
\draw (h3) to (schlafen);
\end{tikzpicture}}

}

\frame{
\frametitle{Unterspezifikation}

\eal
\ex Alle Kinder schlafen.
% \ex \oneline{\textmrs{ h0, \{ h1:every\_q(x, h2, h3), h4:kind(x),
%               h5:schlafen(x) \}, \{ h2 \qeq h4 \}  }}
\ex \{ h1:every\_q(x, h2, h3), h4:kind(x), h5:schlafen(x) \}, \{ h2 \qeq h4 \}
\zl

\centerline{%
\begin{tikzpicture}[mrs]
\node(every)  at (2,6) {h1:every\_q(\subnode{x}{x}, \subnode{h2}{h2}, \subnode{h3}{h3})};
\node(kind) at (0.5,4)  {h4:kind(x)};
\node(schlafen)    at (4,4) {h5:schlafen(x)};
%
\draw (h2) to (kind);
%\draw[color=RED] (x) to (schlafen);
\end{tikzpicture}}

kind(x) ist in der Restriktion des Quantors.

Die Variable x von schlafen(x) muss durch einen Quantor gebunden werden.

Das ist alles.

}

\frame{
\frametitle{Dominanzbeschränkungen wegen Variablenbindung}

\hfill%
\begin{tikzpicture}[mrs]
\node(every)  at (2,6) {h1:every\_q(\subnode{x}{x}, \subnode{h2}{h2}, \subnode{h3}{h3})};
\node(kind) at (0.5,4)  {h4:kind(x)};
\node(schlafen)    at (4,4) {h5:schlafen(x)};
%
\draw (h2) to (kind);
\draw[color=RED] (x) to (schlafen);
\end{tikzpicture}
\hfill
\begin{tikzpicture}[mrs]
\node(every)  at (2,6) {h1:every\_q(\subnode{x}{x}, \subnode{h2}{h2}, \subnode{h3}{h3})};
\node(kind) at (0.5,4)  {h4:kind(x)};
\node(schlafen)    at (4,4) {h5:schlafen(x)};
%
\draw (h2) to (kind);
\draw (h3) to (schlafen);
\end{tikzpicture}\hfill\mbox{}

Variable muss gebunden werden, also muss schlafen(x) irgendwie unter dem Ausdruck sein, der x
einführt.

Da h2 schon vergeben ist (Dort dürfen nur Quantoren noch dazwischentreten.),\\
muss schlafen(x) von h3 dominiert werden.

}

\frame{
\frametitle{Nur ein Quantor und nur eine Lösung}

\hfill
\begin{tikzpicture}[mrs]
\node(every)  at (2,6) {h1:every\_q(\subnode{x}{x}, \subnode{h2}{h2}, \subnode{h3}{h3})};
\node(kind) at (0.5,4)  {h4:kind(x)};
\node(schlafen)    at (4,4) {h5:schlafen(x)};
%
\draw (h2) to (kind);
\draw (h3) to (schlafen);
\end{tikzpicture}\hfill
\begin{tikzpicture}[scoped mrs]
\node(every)  at (2,6) {h1:every\_q(\subnode{x}{x}, \subnode{h2}{h2}, \subnode{h3}{h3})};
\node(kind) at (0.5,4)  {h4:kind(x)};
\node(schlafen)    at (4,4) {h5:schlafen(x)};
%
\draw (h2) to (kind);
\draw (h3) to (schlafen);
\end{tikzpicture}\hfill\mbox{}

Links: Dominanzbeschränkungen aus MRS + Variablenbindung

Rechts: geskopte MRS

}

\frame{
\frametitle{Unterspezifikation mit mehreren Quantoren}

\ea
\{ h1:every\_q(x, h2, h3), h4:tochter(x,y), h5:exist\_q(y, h6, h7), 
\hphantom{\textlangle~}h8:mitarbeiter(y), h9:schlafen(x) \}, \{ h2 \qeq h4, h6 \qeq h8 \}
\z


Das Handle auf der rechten Seite wird mit dem
auf der linken identifiziert, wobei aber noch Quantoren dazwischentreten können. 

Genauer: Wenn h \qeq l gilt, dann muss entweder h = l sein, oder\\
es muss eine Folge Q$_1$, \ldots, Q$_n$ von Quantoren geben, für die gilt, \\
dass h identisch mit dem Label von Q$_1$ ist und l identisch mit dem
Body-Argument von Q$_n$ und für alle Paare in der Folge Q$_m$, Q$_{m+1}$ ist das Label von Q$_{m+1}$
identisch mit dem Body-Argument des Quantors Q$_m$ \citep[\page 297]{MRS}. 

}


\frame{
\frametitle{Dominanzgraph für MRS}

\vfill
\begin{tikzpicture}[mrs]
\node(exists)  at (-2,6) {h5:exist\_q(\subnode{y}{y}, \subnode{h6}{h6}, \subnode{h7}{h7})};
\node(tochter) at (0.5,4)  {h4:tochter(x,y)};
\node(every)   at (2,6)  {h1:every\_q(\subnode{x}{x}, \subnode{h2}{h2}, \subnode{h3}{h3})};
\node(mitarbeiter) at (-3,4) {h8:mitarbeiter(y)};
\node(schlafen)    at (4,4) {h9:schlafen(x)};
%
\draw[color=RED] (y) to (tochter);
\draw (h2) to (tochter);
\draw (h6) to (mitarbeiter);
\draw[color=RED] (x) to (schlafen);
\end{tikzpicture}
\pause
\vfill
\begin{tikzpicture}[mrs]
\node(exists)  at (-2,6) {h5:exist\_q(y, \subnode{h6}{h6}, \subnode{h7}{h7})};
\node(tochter) at (0.5,4)  {h4:tochter(x,y)};
\node(every)   at (2,6)  {h1:every\_q(x, \subnode{h2}{h2}, \subnode{h3}{h3})};
\node(mitarbeiter) at (-3,4) {h8:mitarbeiter(y)};
\node(schlafen)    at (4,4) {h9:schlafen(x)};
\draw[color=RED] (h7) to (tochter);
\draw (h2) to (tochter);
\draw (h6) to (mitarbeiter);
\draw[color=RED] (h3) to (schlafen);
\end{tikzpicture}\vfill


}

\frame{
\frametitle{Geskopte MRSen}

\vfill
\begin{tikzpicture}[scoped mrs]
%\draw (-5,-1) to[grid with coordinates] (5,3);
\node(exists)  at (-2,2) {h5:exist\_q(y, \subnode{h6}{h6}, \subnode{h7}{h7})};
\node(tochter) at (0.5,0)  {h4:tochter(x,y)};
\node(every)   at (2,2)  {h1:every\_q(x, \subnode{h2}{h2}, \subnode{h3}{h3})};
\node(mitarbeiter) at (-3,0) {h8:mitarbeiter(y)};
\node(schlafen)    at (4,0) {h9:schlafen(x)};
\draw (h7) to (tochter);
\draw (h2) .. controls +(-.5,-3) and +(.5,2).. (exists.north);
\draw (h6) to (mitarbeiter);
\draw (h3) to (schlafen);
\end{tikzpicture}
\pause
\vfill
\begin{tikzpicture}[scoped mrs]
%\draw (-5,-1) to[grid with coordinates] (5,3);
\node(exists)  at (-2,2) {h5:exist\_q(y, \subnode{h6}{h6}, \subnode{h7}{h7})};
\node(tochter) at (0.5,0)  {h4:tochter(x,y)};
\node(every)   at (2,2)  {h1:every\_q(x, \subnode{h2}{h2}, \subnode{h3}{h3})};
\node(mitarbeiter) at (-3,0) {h8:mitarbeiter(y)};
\node(schlafen)    at (4,0) {h9:schlafen(x)};
\draw   (h7) .. controls +(.5,-1.5) and +(-1.5,2).. (every.north);
\draw   (h2) to (tochter);
\draw   (h6) to (mitarbeiter);
\draw   (h3) to (schlafen);
\end{tikzpicture}
\vfill


}

\frame{
\frametitle{Lexikoneintrag von \emph{Tochter}}

\ms{
cat & \ms{ head & \ms[noun]{
                  case & \type{nom}\\
                  }\\
           spr   & \nliste{ Det }\\
           comps & \nliste{ NP[\type{gen}]\ind{1} }\\
         }\\
cont & \ms{
       ltop & \ibox{2}\\
       ind  & \ibox{3}\\
       rels & \liste{ \ms[tochter\_rel]{
                      lbl  & \ibox{2}\\ 
                      arg0 & \ibox{3}\\ 
                      arg1 & \ibox{1}\\ 
                      } }\\
       hcons & \eliste
       }
}

\ltop zeigt auf das local top. Das jeweils aktuelle Label.

Der \ltopw von \emph{Tochter} ist das Label der \relation{tochter}-Relation.

Der Index \iboxb{3} ist \argzero.

}

\frame{
\frametitle{Quantoren in Merkmal-Wert-Darstellung}

\eal
\ex \ms[every\_q]{
                      arg0 & \ibox{2}\\ 
                      rstr & \ibox{3}\\ 
                      body & \etag\\
                      }
\ex every\_q(\ibox{2}, \ibox{3}, \etag)
\zl

\body wird nie angegeben. Wird dann bei Skopusauflösung der MRS gefüllt.

}

\frame{
\frametitle{Lexikoneintrag für \emph{jede}}

\ea
\scalebox{.9}{%
\ms{
cat & \ms{ head & \ms[det~~~~~~~~~~~~~~~~~~~~~~~~~~~~~~]{
                  case  \type{nom}\\
                  spec|cont \ms{ \visible<2>{ltop} & \visible<2>{\blau{\ibox{1}}}\\
                                 ind  & \blau<1>{\ibox{2}}\\
                               }
                  }\\
           spr   & \eliste\\
           comps & \eliste\\
         }\\
cont & \ms{
       rels & \liste{ \ms[every\_q]{
                      arg0 & \blau<1>{\ibox{2}}\\ 
                      rstr & \blau<2>{\ibox{3}}\\ 
%                      body &\\
                      } }\\[5mm]
       \visible<2>{hcons} & \visible<2>{\liste{ \blau{\ms[qeq]{
                       harg & \ibox{3}\\
                       larg & \ibox{1}\\
                       }}
               }}\\
       }
}}
\z

Quantor greift über \spec auf zugehöriges Nomen zu und teilt Variable \iboxb{2}\pause und fügt
Skopus-Beschränkung (\ibox{3} \qeq \ibox{1}) ein.

}

\frame{
\frametitle{Jede Tochter eines Mitarbeiter schläft.}

\oneline{%
\begin{forest}
sm edges
[V\feattab{\rels \nliste{ h1:every(x,h2,h3), h4:tochter(x,y), h5:exist(y,h6,h7), h8:mitarbeiter(y),
    h9:schlafen(x) },\\
               \hcons \nliste{ h2 \qeq h4, h6 \qeq h8 }}
  [NP\feattab{\rels \nliste{ h1:every(x,h2,h3), h4:tochter(x,y), h5:exist(y,h6,h7), h8:mitarbeiter(y) },\\
               \hcons \nliste{ h2 \qeq h4, h6 \qeq h8 }}
    [Det\feattab{\rels \nliste{ h1:every(x,h2,h3) },\\
               \hcons \nliste{ h2 \qeq h4 }} [jede] ]
    [N\feattab{\rels \nliste{ h4:tochter(x,y), h5:exist(y,h6,h7), h8:mitarbeiter(y) },\\
               \hcons \nliste{ h6 \qeq h8 }}
      [N\feattab{\rels \nliste{ h4:tochter(x,y) },\\
                 \hcons \eliste} [Tochter] ]
      [NP\feattab{\rels \nliste{ h5:exist(y,h6,h7), h8:mitarbeiter(y) },\\
                  \hcons \nliste{ h6 \qeq h8 }} 
         [Det\feattab{\rels \nliste{ h5:exist(y,h6,h7) },\\
                      \hcons \nliste{ h6 \qeq h8 }} [eines]]
         [N\feattab{\rels \nliste{ h8:mitarbeiter(y) },\\
                    \hcons \eliste} [Mitarbeiters]] ] ] ]
   [V\feattab{\rels \nliste{ h9:schlafen(x) },\\
              \hcons \eliste} [schläft]]]
\end{forest}
}

}

\frame{
\frametitle{Das Spezifikatorprinzip}

\begin{prinzip-break}[Spezifikatorprinzip (\textsc{spec}-Principle)] 
\label{prinzip-spec}\is{Prinzip!Spezifikator-}
Wenn eine Tochter, die keine Kopf"|tochter ist, in einer Kopf"|struktur
einen von \type{none} verschiedenen \textsc{spec}-Wert besitzt, so ist dieser token-identisch mit
%dem \textsc{synsem}-Wert 
der Kopf"|tochter.
\end{prinzip-break}

\pause

Formal:
\ea
\ms{ non-head-dtrs & \liste{ [\textsc{cat|head|spec} \type{sign}] } } \impl\\
\hfill\onems{ head-dtr \ibox{1}\\
        non-head-dtrs  \nliste{ [\textsc{cat|head|spec} \ibox{1}] } }
\z

}

\exewidth{(235)}

\frame{
\frametitle{Possessivpronomina}

\ea
ihr Buch
\z

\ea
Lexikoneintrag für das Possessivpronomen \emph{ihr}:\\
\scalebox{.4}{%
\ms{ 
  cat & \ms{ head  & \ms[det]{
                     case & nom\\
                     spec & \ms{ cont & \ms{ \visible<2->{ltop} & \visible<2->{\blau<2,4>{\ibox{1}}}\\
                                             ind  & \blau<1,3>{\ibox{2}}\\
                                           }}\\
                     }\\
             spr   & \eliste\\
             comps & \eliste \\
           } \\
  cont &  \ms{
%       ltop & \ibox{3}\\
       ind  & \blau<3>{\ibox{3} \ms{
                       per & 3\\
                       num & sg\\
                       gen & fem\\
                      }}\\ 
       rels & \liste{ \blau<1>{\ms[def\_q]{
                      arg0 & \ibox{2}\\ 
                      rstr & \ibox{4}\\ 
%                      body &\\
                      }}\visible<3->{, \blau<3>{\ms[besitzen\_rel]{
                            lbl  & \blau<4>{\ibox{1}}\\     % pronoun_q(x,h, h) h1:pronoun_rel(x),
                                % h1:besitzen(x,y) ^ h1:buch(y)
%                           arg0 & event\\
                            arg1 & \ibox{3}\\
                            arg2 & \ibox{2}
                      }}} }\\
       \visible<2->{hcons} & \visible<2->{\liste{ \blau<2>{\ms[qeq]{
                       harg & \ibox{4}\\
                       larg & \ibox{1}\\
                       }}}
               }\\
           } \\
}}
\z

Possessivum führt Quantor ein. Quantor bindet Variable des Nomens \iboxb{2}.

\pause

Quantor skopt über Nomen (\ibox{4} \qeq \ibox{1}).

\pause

Possessivum führt \relation{besitzen}-Relation ein: besitzen(\ibox{3}, \ibox{2})

\pause

Diese teilt Label mit Nomen.

}

\frame{
\frametitle{MRS für \emph{ihr Buch}}



\ea
\label{MRS ihr Buch}
\{ h1:def\_q(x, h2, h3), h4:besitzen(e1, i1, x), h4:buch(x) \},
\{ h2 \qeq h4 \}
\z

i1 ist der referentielle Index für \emph{ihr}.


}



\frame{
\frametitle{Tucholsky und \emph{potentielle Mörder}}

\eal
%% Darin heißt es, dass "Soldaten nicht nur potentielle Mörder sind, sondern im wahrsten Sinne des Wortes bezahlte Killer".21.10.1994 taz Inland 97 Zeilen, hans-hermann kotte S. 5
%% Augst hatte 1989 bei einer Diskussion Soldaten als "potentielle Mörder" bezeichnet. 10.10.1994 taz Aktuelles 14 Zeilen, S. 2

%% Er hatte sich in einem Leserbrief mit dem Ausspruch "Alle Soldaten sind potentielle Mörder" solidarisiert.22.9.1994 taz Tagesthema 91 Zeilen, kotte S. 3
\ex Gewalt provoziere immer Gegengewalt und:\\
    "`Soldaten sind potentielle Mörder."'\footnote{
23.12.1993, taz berlin, S.\,18.
}

\ex
Soldaten sind Mörder.\footnote{
  Tucholsky, Kurt, (1931), "`Der bewachte Kriegsschauplatz"', \emph{Die Weltbühne}, S.\,31.
}
\ex
\relationunmarked{mörder}(x)
\zl


}


\frame{
\frametitle{Vorläufiger Lexikoneiontrag für \emph{kleinen}}

\ea
\scalebox{.7}{%
\ms{
cat & \ms{ head & \ms[adj]{
                  mod & \ms{ cat & \ms{ head & noun\\
                                        spr  & \nliste{ Det }\\
                                        comps & \eliste }\\
                             cont & \ms{ ltop & \blau{\ibox{1}}\\
                                         ind  & \ibox{2}\\
                                       }}}\\
           spr & \eliste\\
           comps & \eliste\\
         }\\
cont & \ms{ ltop & \blau{\ibox{1}}\\
%            ind  & \ibox{2}\\
            rels & \liste{ \ms[klein]{
                           lbl  & \blau{\ibox{1}}\\ 
                           arg1 & \ibox{2}\\
                           } }\\
            hcons & \eliste\\
          }\\
}}
\z

Das würde das Label des Nomens mit dem des Adjektivs identifizieren.

\pause

\eal
\ex \textmrs{ h1, \{ h1:klein(x), h1:mörder(x) \}, \{  \} }
\pause
\ex \textmrs{ h1, \{ h1:potentiell(x), h1:mörder(x) \}, \{  \} } \pause (falsch!)
\ex \textmrs{ h1, \{ h1:potentiell(h2), h2:mörder(x) \}, \{  \} } (richtig)
\zl

}

\frame{
\frametitle{Lexikoneintrag für \emph{potentielle}}
\ms{
cat & \ms{ head & \ms[adj]{
                  mod & \ms{ cat & \ms{ head & noun\\
                                        spr  & \nliste{ Det }\\
                                        comps & \eliste }\\
                             cont & \ms{ ltop & \ibox{1}\\
                                         ind  & \ibox{2}\\
                                       }}}\\
           spr & \eliste\\
           comps & \eliste\\
         }\\
cont & \ms{ ltop & \ibox{3}\\
%            ind  & \ibox{2}\\
            rels & \liste{ \ms[potentiell]{
                           lbl  & \ibox{3}\\
                           arg1 & \ibox{1}\\
                           } }\\
            hcons & \eliste\\
          }\\
}

}

\frame{
\frametitle{Perkolation des \ltopwes}

Bei intersektiven Adjunkten ist \ltop vom Kopf und Adjunkt gleich. 

Bei nicht-intersektiven hat Adjunkt Skopus über Kopf.

\ltopw kann bzw.\ muss also vom Adjunkt kommen.

\ea
\type{head-adjunct-phrase} \impl
  \ms{ cont|ltop & \ibox{1}\\
       non-head-dtrs & \sliste{ [cont|ltop \ibox{1} ] }\\
     }
\z

}


\subsubsection{\emph{scheinbar einfache Beispiele}}

\frame{
\frametitle{\emph{scheinbar einfache Beispiele}}

\citet{Kasper97a}:
\ea
\label{ex-ein-scheinbar-einfaches-Beispiel}
ein scheinbar einfaches Beispiel
\z

Problem: \emph{scheinbar} bezieht sich nur auf \emph{einfaches}, nicht auf \emph{einfaches
  Beispiel}.

Wenn \emph{einfaches} sein Label mit dem des Nomens identifiziert, wäre das falsch und entspräche:
\ea
ein scheinbares einfaches Beispiel
\z

\pause

\eal
\ex \{ h1:exist\_q(x,h2,h3),h4:scheinbar(h5),\rotbf{h6}:einfach(x),\rotbf{h6}:beispiel(x) \},\\
    \{ h2 \qeq h4, h5 \qeq h6 \}   (falsch!)
\pause
\ex
\label{mrs-ein-scheinbar-einfaches-Beispiel} 
\{ h1:exist\_q(x,h2,h3),\gruenbf{h4}:scheinbar(h5),h6:einfach(x),\gruenbf{h4}:beispiel(x) \},\\
\{ h2 \qeq h4, h5 \qeq h6 \}  
\pause
\ex
\{ h1:exist\_q(x,h2,h3),\rotbf{h4}:scheinbar(h5),\rotbf{h4}:einfach(x),\rotbf{h4}:beispiel(x) \},\\
\{ h2 \qeq h4, h5 \qeq h4 \}  (schrecklich falsch!)
\zl

}

\frame{
\frametitle{Lexikoneintrag für \emph{kleinen}}

\ms{
cat & \ms{ head & \ms[adj]{
                  mod & \ms{ cat & \ms{ head & noun\\
                                        spr  & \nliste{ Det }\\
                                        comps & \eliste }\\
                             cont & \ms{ %ltop & \ibox{1}\\
                                         ind  & \ibox{2}\\
                                       }}}\\
           spr & \eliste\\
           comps & \eliste\\
         }\\
cont & \ms{ ltop & \ibox{1}\\
%            ind  & \ibox{2}\\
            rels & \liste{ \ms[klein]{
                           lbl  & \ibox{1}\\ 
                           arg1 & \ibox{2}\\
                           } }\\
            hcons & \eliste\\
          }\\
}

\ltopw nicht mit \ltopw innerhalb von \textsc{mod} geteilt!

}

\frame{
\frametitle{Erst, wenn wir wissen, was wir tun, tun wir etwas.}


\ea
\label{ex-scopal-ltop}
\avm{
[ non-head-dtrs < [ cat|head|scopal & minus ] > ]} \impl\\[2mm]
\hfill 
\avm{[ head-dtr|cont|ltop & \1\\
  non-head-dtrs < [ cont|ltop & \1 ] > ]
}
\z

\ea
scheinbar einfaches Beispiel
\z
Nicht im Lexikoneintrag von \emph{einfaches} wird Identifikation der \ltopwe vorgenommen, sondern in
dem Moment, wo \emph{scheinbar einfaches} mit \emph{Beispiel} kombiniert wird.

Da wird dann das Handel des skopenden Elements \emph{scheinbar} benutzt. 

\emph{scheinbar} ist nicht-intersektiv, aber es ist auch nicht der Kopf. 

Der Kopf ist \emph{einfaches}, weshalb der \textsc{scopal}"=Wert von \emph{scheinbar einfaches} $-$ ist.

\backslash ő/

}

\subsection{Das Semantikprinzip}

\frame{
\frametitle{Das Semantikprinzip}

Index und Ereignisvariable wird weitergegeben:
\ea
\type{headed-phrase} \impl\\
\ms{
cont|ind & \ibox{1}\\
head-dtr|cont|ind & \ibox{1}\\
}
\z

\pause

\ltop kommt vom Kopf, außer bei Kopf"=Adjunkt-Phrasen:
\eal
\label{Semantikprinzip-LTOP}
\ex \type{head-non-adjunct-phrase} \impl\\
\ms{
cont|ltop & \ibox{1}\\
head-dtr|cont|ltop & \ibox{1}\\
}
\ex
\label{Beschränkung-Semantik-Kopf-Adjunkt-Strukturen2} 
\type{head-adjunct-phrase} \impl\\
  \ms{ cont|ltop & \ibox{1}\\
       non-head-dtrs & \sliste{ [cont|ltop \ibox{1} ] }\\
     }

\zl 

}

\frame{
\frametitle{Identifikation von \ltop von Töchtern}

\ltopw von intersektiven Modifikatoren wird mit \ltopw der Kopftochter identifiziert:
\ea
\avm{
[\type{head-adjunct-phrase}\\
 non-head-dtrs < [ cat|head|scopal & minus ] > ]} \impl\\[2mm]
\hfill 
\avm{[ head-dtr|cont|ltop & \1\\
  non-head-dtrs < [ cont|ltop & \1 ] > ]
}
\z

\pause
Ansonsten regelt die nicht-intersektive Tochter das und fügt \hcons hinzu.

}


} % \end{teil1}

\subsection{Übungsaufgaben}

\frame{
\frametitle{Übungsaufgaben}

\begin{enumerate}
\item Wie kann man den semantischen Beitrag von \emph{lacht} repräsentieren?
\item Geben Sie eine Merkmalbeschreibung für \emph{er lacht} in (\mex{1}) an:
      \ea
      {}[dass] er lacht
      \z
\end{enumerate}

}
