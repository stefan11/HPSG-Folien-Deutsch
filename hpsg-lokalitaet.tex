%%%%%%%%%%%%%%%%%%%%%%%%%%%%%%%%%%%%%%%%%%%%%%%%%%%%%%%%%
%%   $RCSfile: lokalitaet-hpsg-folien.tex,v $
%%  $Revision: 1.2 $
%%      $Date: 2007/05/27 13:27:12 $
%%     Author: Stefan Mueller (CL Uni-Bremen)
%%    Purpose: 
%%   Language: LaTeX
%%%%%%%%%%%%%%%%%%%%%%%%%%%%%%%%%%%%%%%%%%%%%%%%%%%%%%%%%


\section{Lokalität}

\outline{
\begin{itemize}
%\item Ziele
\item Wozu Syntax? / Phrasenstrukturgrammatiken
\item Formalismus
\item Valenz und Grammatikregeln
\item Dominanzstrukturen und Prinzipien
\item Semantik
\item Adjunktion und Spezifikation
\item Das Lexikon: Typen und Lexikonregeln
\item Topologie des deutschen Satzes
\item Konstituentenreihenfolge
\item Nichtlokale Abhängigkeiten
\item Relativsätze
\item \blau{Lokalität}
%\item Komplexe Prädikate: Der Verbalkomplex
\end{itemize}
}

\frame{
\frametitle{Lokalität}
%
\begin{itemize}
\item Literatur: \citew[Kapitel~12.1]{MuellerLehrbuch3}
\end{itemize}

\vspace{1cm}

%% \rotbf{Achtung, wichtiger Hinweis: Diese Literaturangabe hier bedeutet,\\dass Sie die Literatur zum
%%   nächsten Mal lesen sollen!!!!}
}


\frame{
\frametitle{Lokalität der Selektion}

\begin{itemize}[<+->]
\item mit aktueller Merkmalsgeometrie Zugriff auf die phonologische Form und die interne Struktur von Komplementen und
      von Köpfen\\
      in Kopf-Adjunkt-Strukturen
\item Kopf kann sagen: ich möchte etwas, dessen Komplementtochter etwas mit {\sc phon}"=Wert {\em dem Mann\/} ist
\item Sowas soll ausgeschlossen werden. $\to$
      entsprechende Merkmalsgeometrie
\item Gruppierung aller Merkmale, die selegiert werden können,\\
      unter einem Pfad
\item Sowohl syntaktische als auch semantische Information kann selegiert werden.
\end{itemize}
}

\subsection{Die Die Datenstruktur}

\frame[shrink=25]{
\frametitle{Lokalität der Selektion: Die Datenstruktur}

\begin{tabular}{@{}l@{\hspace{3em}}l@{}}
bisherige Datenstruktur:                                  & \visible<2->{neue Datenstruktur:}\\
\scalebox{0.83}{\ms{ phon & list~of~phoneme strings\\
     loc  & \ms[loc]{  cat  & \ms[cat]{ head   & head \\
                                        subcat & list\\
                                      } \\
                       cont & cont\\
                    }\\
     nonloc & \ms[nonloc]{
              que & \type{list~of~npros} \\
              rel & \type{list~of~indices} \\
              slash & \type{list~of~local~structures} \\ %\\
              %extra & \ms[list~of~local~structures]{} \\
              }\\
     head-dtr & sign\\
     non-head-dtrs & list~of~signs\\ 
         }} &
\visible<2->{\scalebox{0.83}{\ms{ phon & list~of~phoneme strings\\
     syntax-semantics & \highlight{\ms[synsem]{
     loc  & \ms[loc]{  cat  & \ms[cat]{ head   & head \\
                                        subcat & list\\
                                      } \\
                       cont & cont\\
                    }\\
     nonloc & \ms[nonloc]{
              que & \type{list~of~npros} \\
              rel & \type{list~of~indices} \\
              slash & \type{list~of~local~structures} \\ %\\
              %extra & \ms[list~of~local~structures]{} \\
              }\\
     }} \\
     head-dtr & sign\\
     non-head-dtrs & list~of~signs\\ 
}}}\\
\end{tabular}
\begin{itemize}
\pause
\item {\sc synsem} steht für {\sc syntax-sematics}.
\pause
\item nur markierter Bereich kann selegiert werden $\to$ keine Töchter oder {\sc phon}
\pause
\item Elemente in {\sc subcat}"=Listen sind {\it synsem\/}-Objekte.
\end{itemize}
}

\subsection{Das angepasste Kopf-Argument-Schema}

\frame[label=schema-head-arg]{
\frametitle{Das angepaßte Kopf-Argument-Schema}

{\it head-argument-phrase\/} \impl\\
\ms{
      synsem & \onems{ loc$|$cat$|$subcat \ibox{1} $\oplus$ \ibox{3}\\
                     }\\
      head-dtr & \onems{ synsem$|$loc$|$cat$|$subcat \ibox{1} $\oplus$ \sliste{ \ibox{2} } $\oplus$ \ibox{3} \\
                       }\\
      non-head-dtrs & \sliste{\onems{ synsem \ibox{2}  \\ }}
}




}

