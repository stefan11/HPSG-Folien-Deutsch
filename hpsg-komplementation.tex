\section{Komplementation}

\iftoggle{teil1}{
\outline{

\begin{itemize}
\item Wozu Syntax? / Phrasenstrukturgrammatiken
\item Formalismus
\item Valenz und Grammatikregeln
\item \blaubf{Komplementation}
\item Semantik
\item Adjunktion und Spezifikation
\item Das Lexikon: Typen und Lexikonregeln
\item Topologie des deutschen Satzes
\item Konstituentenreihenfolge
\item Nichtlokale Abhängigkeiten
\item Relativsätze
\item Lokalität
%\item Komplexe Prädikate: Der Verbalkomplex
\end{itemize}
}
} %\end{teil1}

\iftoggle{teil2}{
\outline{
\begin{itemize}
\item Wiederholung
      \begin{itemize}
\item Wozu Syntax? / Phrasenstrukturgrammatiken
\item Formalismus
\item Valenz und Grammatikregeln
\item \blau{Komplementation}
\item Semantik
\item Topologie des deutschen Satzes
\item Konstituentenreihenfolge
\item Nichtlokale Abhängigkeiten
\end{itemize}
\item Kongruenz
\item Kasus
\item Der Verbalkomplex
\item Kohärenz, Inkohärenz, Anhebung und Kontrolle
\item Passiv
\item Partikelverben
\item Morphologie
\end{itemize}
}
} % \end{teil2}

\frame{
\frametitle{Komplementation}
%
\begin{itemize}
\item Literatur: \citew[Kapitel~4]{MuellerLehrbuch3}
\end{itemize}

\vspace{1cm}

\rotbf{Bitte zum nächsten Mal lesen}

\vspace{1cm}

Damit alles kompatibel zum Lehrbuch bleibt,\\
nehmen wir hier auch das \subcatm für die Valenz an.

\subcat = \spr + \comps

Zu neueren Versionen der HPSG, die \subcat in \spr und \comps unterteilen,\\ siehe
\citew{Sag97a,HPSGHandbook,MuellerGermanic}.

Deutsch: Argumente von finiten Verben sind alle auf \comps, so
dass die Verwendung von \subcat hier keinen Unterschied macht.

}


\subsection{Die Modellierung von Konsituentenstruktur mit Hilfe von Merkmalstrukturen}


\frame{
\frametitle{Repräsentation von Grammatikregeln}

\begin{itemize}
\item Merkmalstrukturen als einheitliches Beschreibungsinventar für 
      \begin{itemize}
      \item morphologische Regeln
      \item Lexikoneinträge
      \item syntaktische Regeln
      \end{itemize}
\pause
\item Trennung von unmittelbarer Dominanz (ID) und linearer Präzedenz (LP)
\pause
\item Dominanz in \textsc{dtr}"=Merkmalen (Kopftochter und Nicht-Kopftöchter)
\pause
\item Präzedenz implizit in \phon
\end{itemize}

}


\frame{
\frametitlefit{Teilstruktur in Merkmalstrukturrepräsentation -- \phonw{}e (I)}

\vfill

\hfill\begin{tabular}[t]{@{}l@{\hspace{5em}}l@{}}
\begin{forest}
sm edges
[NP
  [Det,baseline
    [das]]
  [N
    [Kind]]]
\end{forest} &%
\ms{ 
                                                  phon     & \phonliste{ dem Eichhörnchen }\\
                                                  head-dtr & \onems{ phon \phonliste{ Eichhörnchen }  \\
                                                                 }\\
                                                  non-head-dtrs & \liste{ \onems{ phon \phonliste{ dem } \\
                                                                              }}\\
                                                  }\end{tabular}\hfill\mbox{}

\begin{itemize}
\item Es gibt genau eine Kopftochter (\textsc{head-dtr}).\\
      Die Kopftochter enthält den Kopf bzw.\ ist der Kopf.\\
      Struktur mit den Töchtern \emph{das} und \emph{Bild von Maria} $\to$\\
        \emph{Bild von Maria} ist die Kopftochter, da \emph{Bild} der Kopf ist.
\pause
\item Es kann mehrere Nicht-Kopftöchter geben\\
      (bei Annahme von flachen Strukturen oder bei binär verzweigenden Strukturen ohne Kopf).
\end{itemize}
}

%% \frame{
%% \frametitle{Baum mit \textsc{dtrs}"=Werten (I)}

%% \vfill
%% \centerline{\epsfxsize=0.5\textwidth\mbox{\epsffile{dem-mann-dtrs.eps}}}
%% \vfill
%% }



\frame{

\small
\frametitle{Repräsentation von Grammatikregeln (II)}

\begin{itemize}
\item Dominanzregel:

\type{head-argument-phrase} \impl\\*
\onems{
      subcat \ibox{1} \\
      head-dtr$|$subcat \ibox{1} $\oplus$ \sliste{ \ibox{2} } \\
      non-head-dtrs \sliste{ \ibox{2} }\\
      } \\

Pfeil bedeutet Implikation
\item alternative Schreibweise, angelehnt an \xbar-Schema:
      \begin{tabular}[t]{@{}lll}
      H[SUBCAT \ibox{1}] & $\to$ & H[SUBCAT \ibox{1} $\oplus$ \sliste{ \ibox{2} } ]~~ \ibox{2}\\
      \end{tabular}

Pfeil bedeutet Ersetzung
\medskip
\pause
\item mögliche Instantiierungen:
      \begin{tabular}[t]{@{}lll}
      N[SUBCAT \ibox{1}] & $\to$ & Det N[SUBCAT \ibox{1} \eliste{} $\oplus$ \sliste{ Det } ]\\
      V[SUBCAT \ibox{1}] & $\to$ & V[SUBCAT \ibox{1} \eliste{} $\oplus$ \sliste{ NP } ]~~ NP\\
      V[SUBCAT \ibox{1}] & $\to$ & V[SUBCAT \ibox{1} \sliste{ NP } $\oplus$ \sliste{ NP } ]~~ NP\\
      \end{tabular}
\end{itemize}

}

\frame{
\frametitle{Ein Beispiel}

\centerline{%
\begin{forest}
sm edges
[{V[\subcat \sliste{} ]}
  [{\ibox{1} NP[\type{nom}]}
    [Aicke]]
  [{V[\subcat \sliste{ \ibox{1} }]}
    [{\ibox{2} NP[\type{acc}]} 
      [die Nuss, roof]]
    [{V[\subcat \sliste{ \ibox{1}, \ibox{2} }]}
      [{\ibox{3} NP[\type{dat}]}
        [dem Eichhörnchen,roof]]
      [{V[\subcat \sliste{ \ibox{1}, \ibox{2}, \ibox{3} }]}
        [gibt]]]]]
\end{forest}}

}

\iftoggle{teil1}{
\frame[shrink]{

\small
\frametitlefit{Teilstruktur in Merkmalstrukturrepräsentation -- \phonw{}e (I)}

\vfill
\centerline{\begin{forest}
sm edges
[V
  [{NP[\type{nom}]}
    [Aicke]]
  [V
    [{NP[\type{acc}]} 
      [die Nuss, roof]]
    [V
      [{NP[\type{dat}]}
        [dem Eichhörnchen,roof]]
      [V
        [gibt]]]]]
\end{forest}}

\vfill
\centerline{\(
\ms{
      phon     & \phonliste{ dem Eichhörnchen gibt }\\
      head-dtr & \onems{ phon \phonliste{ gibt }\\
                       } \\
      non-head-dtrs & \liste{ \ms{ 
                                                  phon     & \phonliste{ dem Eichhörnchen }\\
                                                  head-dtr & \onems{ phon \phonliste{ Eichhörnchen }  \\
                                                                 }\\
                                                  non-head-dtrs & \liste{ \onems{ phon \phonliste{ dem } \\
                                                                              }}\\
                                                  }
                       }\\
}
\)
}
\vfill
}

\mode<handout>{
%% \frame{

%% \small
%% \frametitle{\normalsize Baum mit \textsc{dtrs}"=Werten (II)}

%% \vfill
%% \centerline{\epsfxsize=0.6\textwidth\mbox{\epsffile{er-das-buch-dem-mann-gibt-simple-phon-dtr.eps}}}
%% \vfill
%% }

\frame{

\small
\frametitlefit{Teilstruktur in Merkmalstrukturrepräsentation -- \phonw{}e (II)}

\vfill
\resizebox{\linewidth}{!}{\(
\onems{
phon      \phonliste{ er die Nuss dem Eichhörnchen gibt }\\
head-dtr \onems{
phon     \phonliste{ die Nuss dem Eichhörnchen gibt }\\
head-dtr \onems{
      phon     \phonliste{ dem Eichhörnchen gibt }\\
      head-dtr \onems{ phon \phonliste{ gibt }\\
                       } \\
      non-head-dtrs \liste{ \onems{ 
                                                  phon     \phonliste{ dem Eichhörnchen }\\
                                                  head-dtr \onems{ phon \phonliste{ Eichhörnchen }  \\
                                                                 }\\
                                                  non-head-dtrs \liste{ \onems{ phon \phonliste{ dem } \\
                                                                              }}\\
                                                  }
                       }\\
}\\
non-head-dtrs \liste{ \onems{ 
                                                  phon     \phonliste{ die Nuss }\\
                                                  head-dtr \onems{ phon \phonliste{ Nuss }  \\
                                                                 }\\
                                                  non-head-dtrs \liste{ \onems{ phon \phonliste{ die } \\
                                                                              }}\\
                                                  }
                       }\\
} \\
non-head-dtrs \liste{ \onems{phon \phonliste{ er }\\
                       } \\
}\\
}
\)
}
\vfill
}
}%handout

\frame{


\frametitle{Teilstruktur in Merkmalstrukturrepräsentation}

\vfill

{\small
\centerline{
\onems[head-argument-phrase~]{
      phon    \phonliste{ dem Eichhörnchen gibt }\\
      subcat \ibox{1} \liste{ NP[\type{nom}], NP[\type{acc}]} \\
      head-dtr \onems{ phon \phonliste{ gibt }\\
                                     subcat \ibox{1}  $\oplus$ \liste{ \ibox{2} } \\
                       } \\
      non-head-dtrs \liste{ \ibox{2} \onems[head-argument-phrase~]{ 
                                                  phon  \phonliste{ dem Eichhörnchen }\\
                                                  p-o-s \type{noun}\\
                                                  subcat \liste{} \\
                                                  head-dtr \ldots \\
                                                  non-head-dtrs \ldots \\
                                                  }
                       }\\
}
}
\vfill
% \(
% \onems[head-argument-phrase~]{
%       phon    \phonliste{ dem Eichhörnchen gibt }\\
%       subcat \liste{ NP[\type{nom}], NP[\type{acc}]} \\
%       head-dtr \onems[word]{ phon \phonliste{ gibt }\\
%                                      subcat \liste{ NP[\type{nom}], NP[\type{acc}]}  $\oplus$ \liste{ \ibox{1} } \\
%                        } \\
%       non-head-dtrs \liste{ \ibox{1} \onems[head-argument-phrase~]{ 
%                                                   phon  \phonliste{ dem Eichhörnchen }\\
%                                                   head  \ms[noun]{ cas & dat\\
%                                                                  } \\
%                                                   subcat \liste{} \\
%                                                   head-dtr \ms{ phon & \phonliste{ Eichhörnchen }\\
%                                                                 head & \ms[noun]{ cas & dat\\
%                                                                                 } \\
%                                                                 subcat & \liste{ \ibox{2} } \\
%                                                               }\\
%                                                   non-head-dtrs \liste{ \ms{ phon & \phonliste{ dem }\\
%                                                                              head & \ms[det]{ cas & dat\\
%                                                                                             } \\
%                                                                              subcat & \liste{ } \\
%                                                                            } 
%                                                                       } \\
%                                                   }
%                        }\\
% }
% \)


% \medskip
% Selektion von \type{synsem}"=Objekten $\to$ Selektion ist lokal (\textsc{phon} und Töchter nicht selegierbar)
}
}

\mode<handout>{
\frame{

\small
\frametitle{Teilstruktur in Merkmalstrukturrepräsentation}


\vfill
\centerline{
\onems[head-argument-phrase~]{
      phon    \phonliste{ er die Nuss dem Eichhörnchen gibt }\\
      subcat \ibox{1} \liste{ } \\
      head-dtr \onems[head-argument-phrase]{ phon \phonliste{ die Nuss dem Eichhörnchen gibt }\\
                                       subcat \ibox{1}  $\oplus$ \liste{ \ibox{2} } \\
                       } \\
      non-head-dtrs \liste{ \ibox{2} \onems[word]{ 
                                       phon \phonliste{ er }\\
                                       p-o-s \type{noun}\\
                                       subcat  \liste{} \\
                                      }
                       }\\
}}
\vfill
}

}%handout
}{}%teil1

\subsection{Projektion von Kopfeigenschaften}

\frame{

\frametitle{Projektion von Eigenschaften des Kopfes}

\centerline{%
\begin{forest}
sm edges
[{V[\rnode{1fin}{\type{fin}}, \subcat \sliste{} ]}
  [{\ibox{1} NP[\type{nom}]}
    [Aicke]]
  [{V[\rnode{2fin}{\type{fin}}, \subcat \sliste{ \ibox{1} }]}
    [{\ibox{2} NP[\type{acc}]} 
      [die Nuss, roof]]
    [{V[\rnode{3fin}{\type{fin}}, \subcat \sliste{ \ibox{1}, \ibox{2} }]}
      [{\ibox{3} NP[\type{dat}]}
        [dem Eichhörnchen,roof]]
      [{V[\rnode{4fin}{\type{fin}}, \subcat \sliste{ \ibox{1}, \ibox{2}, \ibox{3} }]}
        [gibt]]]]]
\end{forest}}
\blau{\ncline[angleA=-90]{<->}{1fin}{2fin}
\ncline{<->}{2fin}{3fin}
\ncline{<->}{3fin}{4fin}}
% tl


\medskip
Kopf ist finites Verb


}


\frame{
\frametitle{Merkmalstrukturrepräsentation: der \textsc{head}"=Wert}

\begin{itemize}
\item mögliche Merkmalsgeometrie:\\
      \ms{ phon   & list~of~phoneme strings\\
           \blau<2->{p-o-s}  & \blau<2->{p-o-s} \\
           \blau<2->{vform} & \blau<2->{vform} \\ 
           subcat & list\\
         }\\
\pause
\item mehr Struktur, Bündelung der Information, die projiziert werden soll:\\
      \ms{ phon & list~of~phoneme strings\\
           head & \ms{
                  \blau<2->{p-o-s} & \blau<2->{p-o-s}\\
                  \blau<2->{vform} & \blau<2->{vform}\\
                  } \\
           subcat & list\\
         }\\
\end{itemize}
}

\frame{

\frametitle{Verschiedene Köpfe projizieren unterschiedliche Merkmale}

\begin{itemize}
\item \textsc{vform} ist nur für Verben sinnvoll
\item pränominale Adjektive und Nomina projizieren Kasus
\item Mögliche Struktur: Eine Struktur mit allen Merkmalen:\\
      \ms{
                  p-o-s & p-o-s\\
                  vform & vform\\
                  case  & case\\
         }

      Bei Verben hat \textsc{case} keinen Wert, bei Nomina \textsc{vform} keinen Wert
\pause
\item Besser: Verschiedene Typen von Merkmalstrukturen
      \begin{itemize}
      \item für Verben:

        \ms[verb]{
        vform & vform\\
        }
      \item für Nomina

        \ms[noun]{
        case & case\\
        }
        \end{itemize}
\end{itemize}
}


\subsection{Ein Lexikoneintrag mit Kopfmerkmalen}
\frame{
\frametitle{Ein Lexikoneintrag mit Kopfmerkmalen}


\begin{itemize}
\item Ein Lexikoneintrag besteht aus:

\mbox{\type{gibt}:}

\ms{ \visible<2->{phon}   & \visible<2->{\phonliste{ gibt }}\\
     \visible<3->{head}   & \visible<3->{\ms[verb]{ vform & fin \\}} \\
     \visible<4->{subcat} & \visible<4->{\liste{ NP[\type{nom}], NP[\type{acc}], NP[\type{dat}] }} \\
}
\pause
\begin{itemize}
\item phonologischer Information
\pause
\item Kopfinformation (part of speech, Verbform, \ldots)
\pause
\item Valenzinformation: einer Liste von Merkmalsbeschreibungen
\end{itemize}

\end{itemize}



}


\frame{
\frametitle{Kopfmerkmalsprinzip (Head Feature Principle)}


\begin{itemize}
\item In einer Struktur mit Kopf sind die Kopfmerkmale der Mutter token-identisch
mit den Kopfmerkmalen der Kopftochter.

\medskip
\type{headed-phrase} \impl
\ms{ 
head \ibox{1}\\
head-dtr$|$head \ibox{1}\\
} \\

\pause
%\item Kodierung der Prinzipien im Typsystem: \citet{Krieger94a} und \citet{Sag97a}
\item \type{head-argument-phrase} ist Untertyp von \type{headed-phrase}\\
      $\to$ Beschränkungen gelten auch
\pause
\item \type{head-argument-phrase} erbt Eigenschaften von \type{headed-phrase}.
\end{itemize}


}

\subsection{Typen}
\subsubsection{Ein nicht-linguistisches Beispiel für Mehrfachvererbung}

\frame{
\frametitlefit{Typen: Ein nicht-linguistisches Beispiel für Mehrfachvererbung}


\vfill
%\oneline{%
\hfill
\begin{tabular}{cccc}
\multicolumn{4}{c}{\mynode{ed}{\type{elektrisches Gerät}}}\\[6ex]
\mynode{p}{\type{druckendes Gerät}} & & \mynode{sc}{\type{scannendes Gerät}} & \mynode{other}{\rule[-0.5ex]{0cm}{2.5ex}\ldots}\\[6ex]
\mynode{printer}{\type{Drucker}}   & \mynode{copy}{\type{Kopierer}}  & \mynode{scanner}{\type{Scanner}}\\[6ex]
\mynode{l-p}{\type{Laserdrucker}}  & \mynode{other-p}{\rule[-0.5ex]{0cm}{2.5ex}\ldots}  & \mynode{negscan}{\type{Negativscanner}} & \mynode{other-sc}{\rule[-0.5ex]{0cm}{2.5ex}\ldots}\\
\end{tabular}
\begin{tikzpicture}[overlay,remember picture,shorten <=2pt,shorten >=2pt] 
\draw(ed.south)--(p.north)
     (ed.south)--(sc.north)
     (ed.south)--(other.north)
     (p.south)--(copy.north)
     (p.south)--(printer.north)
     (printer.south)--(l-p.north)
     (printer.south)--(other-p.north)
     (sc.south)--(copy.north)
     (sc.south)--(scanner.north)
     (scanner.south)--(negscan.north)
     (scanner.south)--(other-sc.north);
\end{tikzpicture}
%}
\hfill\hfill\mbox{}
\vfill

}

\frame{
\frametitle{Eigenschaften von Typhierarchien}


\begin{itemize}[<+->]
\item Subtypen erben Eigenschaften und Beschränkungen\\von ihre(n) Supertypen.

\item Generalisierungen können erfaßt werden:\\
      Allgemeine Beschränkungen werden an oberen Typen repräsentiert.

\item Speziellere Typen erben diese Information von ihren Obertypen.

\item Dadurch Repräsentation von Information ohne Redundanz möglich
\end{itemize}

}

\subsubsection{Linguistische Generalisierungen im Typsystem}
\frame{
\frametitle{Linguistische Generalisierungen im Typsystem}

\begin{itemize}
\item Typen bilden Hierarchie
\pause
\item oben steht der allgemeinste Typ
\pause
\item Information über Eigenschaften von Objekten eines bestimmten Typs werden beim Typ spezifiziert.
\pause
\item Untertypen ererben diese Eigenschaften
\pause
\item Beispiel: Lexikoneintrag in Meyers Lexikon. Verweise auf übergeordnete Konzepte, keine
      Wiederholung der bereits beim übergeordneten Konzept aufgeführten Information\\
\pause
\item Der obere Teil der Typhierarchie ist für alle Sprachen relevant (Universalgrammatik).
\pause
\item Spezifischere Typen können sprachklassen- oder sprachspezifisch sein.
\end{itemize}


}


\frame{
\frametitle{Typhierarchie für \type{sign}}

\vfill
\centerline{%
\begin{forest}
type hierarchy
[sign
  [word]
  [phrase
    [non-headed-phrase]
    [headed-phrase
      [\ldots]
      [head-argument-phrase]
      [\ldots]]]]
\end{forest}}
\vfill
\centerline{alle Untertypen von \type{headed-phrase} erben Beschränkung}
\vfill
}


\frame{
\frametitle{Kopf-Komplement-Schema + Kopfmerkmalsprinzip}


\ms[head-argument-phrase~]{
\blau{head}   & \blau{\ibox{1}} \\
subcat & \ibox{2} \\[2mm]
\blau{head-dtr}  & \ms{ \blau{head}   & \blau{\ibox{1}} \\
                 subcat & \ibox{2} $\oplus$ \sliste{ \ibox{3} } \\
               } \\
non-head-dtrs  & \sliste{ \ibox{3} } \\
}

Typ \type{head-argument-phrase} mit \blau{von \type{headed-phrase} ererbter Information}


}



\frame{

\frametitle{\normalsize Teilstruktur in Merkmalstrukturrepräsentation}



\centering
\scalebox{0.7}{%
\onems[head-argument-phrase~]{
      phon  \phonliste{ dem Eichhörnchen gibt }\\
      head  \ibox{1}\\
      subcat \ibox{2} \sliste{ NP[\type{nom}], NP[\type{acc}] } \\
      head-dtr \onems[word]{ phon \phonliste{ gibt }\\
                                     head \ibox{1} \ms[verb]{vform & fin \\
                                                            }\\
                                     subcat \ibox{2}  $\oplus$ \sliste{ \ibox{3} } \\
                       } \\
      non-head-dtrs \liste{ \ibox{3} \onems[head-argument-phrase~]{ 
                                        phon \phonliste{ dem Eichhörnchen }\\
                                        head \ms[noun]{ cas & dat\\
                                                      } \\
                                        subcat   \sliste{} \\
                                        head-dtr \ldots\\
                                        non-head-dtrs \ldots\\
                                     }
                       }\\
}}
}


\subsection{Übungsaufgaben}

\frame{
\frametitle{Übungsaufgaben}


\begin{enumerate}
\item Zeichnen Sie einen Syntaxbaum für (\mex{1}):
\ea
dass die Frau das spannende Buch liest
\z
Markieren Sie die Kanten im Baum mit Ad für Adjunkt, Ar für Argument und
H für Kopf.
\item Geben Sie die vollständige Merkmalstruktur für (\mex{1}) an:
      \ea
      Schläft das Kind?
      \z
\end{enumerate}

}
