\section{Dominanzstrukturen und Prinzipien}

\iftoggle{teil1}{
\outline{

\begin{itemize}
\item Wozu Syntax? / Phrasenstrukturgrammatiken
\item Formalismus
\item Valenz und Grammatikregeln
\item \blaubf{Dominanzstrukturen und Prinzipien}
\item Semantik
\item Adjunktion und Spezifikation
\item Das Lexikon: Typen und Lexikonregeln
\item Topologie des deutschen Satzes
\item Konstituentenreihenfolge
\item Nichtlokale Abhängigkeiten
\item Relativsätze
\item Lokalität
%\item Komplexe Prädikate: Der Verbalkomplex
\end{itemize}
}
} %\end{teil1}

\iftoggle{teil2}{
\outline{
\begin{itemize}
\item Wiederholung
      \begin{itemize}
\item Wozu Syntax? / Phrasenstrukturgrammatiken
\item Formalismus
\item Valenz und Grammatikregeln
\item \blau{Dominanzstrukturen und Prinzipien}
\item Semantik
\item Topologie des deutschen Satzes
\item Konstituentenreihenfolge
\item Nichtlokale Abhängigkeiten
\end{itemize}
\item Kongruenz
\item Kasus
\item Der Verbalkomplex
\item Kohärenz, Inkohärenz, Anhebung und Kontrolle
\item Passiv
\item Partikelverben
\item Morphologie
\end{itemize}
}
} % \end{teil2}

\frame{
\frametitle{Dominanzstrukturen und Prinzipien}
%
\begin{itemize}
\item Literatur: \citew[Kapitel~4]{MuellerLehrbuch4}
\end{itemize}

\vspace{1cm}

\rotbf{Bitte zum nächsten Mal lesen}

}


\subsection{Die Modellierung von Konsituentenstruktur mit Hilfe von Merkmalstrukturen}


\frame{
\frametitle{Repräsentation von Grammatikregeln}

\begin{itemize}
\item Merkmalstrukturen als einheitliches Beschreibungsinventar für 
      \begin{itemize}
      \item morphologische Regeln
      \item Lexikoneinträge
      \item syntaktische Regeln
      \end{itemize}
\pause
\item Trennung von unmittelbarer Dominanz (ID) und linearer Präzedenz (LP)
\pause
\item Dominanz in \textsc{dtr}"=Merkmalen (Kopftochter und Nicht-Kopftöchter)
\textsc{headdtr} und \textsc{non-head-dtr}
\pause
\item Präzedenz in \dtrs und sich daraus ergebend in \phon
\end{itemize}

}


\frame{
\frametitlefit{Teilstruktur in Merkmalstrukturrepräsentation -- \phonw{}e (I)}

\vfill

\hfill\begin{tabular}[t]{@{}l@{\hspace{5em}}l@{}}
\begin{forest}
sm edges
[NP
  [Det,baseline
    [dem]]
  [N
    [Eichhörnchen]]]
\end{forest} &%
\ms{ 
                                                  phon     & \phonliste{ dem Eichhörnchen }\\
                                                  head-dtr & \onems{ phon \phonliste{ Eichhörnchen }  \\
                                                                 }\\
                                                  non-head-dtrs & \liste{ \onems{ phon \phonliste{ dem } \\
                                                                              }}\\
                                                  }\end{tabular}\hfill\mbox{}

\begin{itemize}
\item Es gibt genau eine Kopftochter (\textsc{head-dtr}).\\
      Die Kopftochter enthält den Kopf bzw.\ ist der Kopf.\\
      Struktur mit den Töchtern \emph{das} und \emph{Bild von Maria} $\to$\\
        \emph{Bild von Maria} ist die Kopftochter, da \emph{Bild} der Kopf ist.
\pause
\item Es kann mehrere Nicht-Kopftöchter geben\\
      (bei Annahme von flachen Strukturen).
\end{itemize}
}

\frame{
\frametitlefit{Mit entsprechend der sichtbaren Reihenfolge geordneten \dtrs}

\ea
\onems{ 
  phon      \phonliste{ dem Eichhörnchen }\\[1mm]
  head-dtr  \ibox{1} \\
  non-head-dtrs  \nliste{ \ibox{2} }\\
  dtrs  \liste{ \ibox{2} \onems{ phon \phonliste{ dem } }, \ibox{1} \onems{ phon \phonliste{ Eichhörnchen } } }\\
}
\z

}

\frame{
\frametitle{Regelschemata als Merkmalbeschreibungen}

\ea
\begin{tabular}[t]{@{}l@{~}lll}
a. & H[\spr \ibox{1}] & $\to$ & \ibox{2} H[\spr \ibox{1}  $\oplus$ \sliste{ \ibox{2} }  ]\\
b. & H[\comps \ibox{1}] & $\to$ & H[\comps \ibox{1} $\oplus$ \sliste{ \ibox{2} } ] \ibox{2}\\
\end{tabular}
\z

\begin{schema}[Kopf-Spezifikator-Schema (vorläufige Version)]
\label{schema-head-specifier-prel}
\type{head-specifier-phrase}\istype{head"=specifier"=phrase} \impl\\
\onems{
      spr \ibox{1} \\
      head-dtr$|$spr \ibox{1} $\oplus$ \sliste{ \ibox{2} } \\
      non-head-dtrs \sliste{ \ibox{2} }\\
      }
\end{schema}
Pfeil bedeutet Implikation.

\pause
\begin{schema}[Kopf-Komplement-Schema (vorläufige Version)]
\label{schema-bin-prel}
\type{head-complement-phrase}\istype{head"=complement"=phrase} \impl\\
\onems{
      comps \ibox{1} \\
      head-dtr$|$comps \ibox{1} $\oplus$ \sliste{ \ibox{2} } \\
      non-head-dtrs \sliste{ \ibox{2} }\\
      }
\end{schema}

}


\frame{
\frametitle{Ein Beispiel}


\ea
\begin{tabular}[t]{@{}l@{~}lll}
a. & N[\spr \ibox{1}] & $\to$ & Det N[\spr \ibox{1} $\oplus$ \nliste{ Det } ]\\
b. & V[\comps \ibox{1}] & $\to$ & V[\comps \ibox{1} $\oplus$ \nliste{ NP } ]~~ NP\\
\end{tabular}
\z

\pause

\centerline{%
\begin{forest}
sm edges
[V
  [NP
    [Aicke]]
  [V
    [NP
      [Det [dem]]
      [N [Eichhörnchen]]]
    [V
      [NP
        [Det [die]]
        [N [Nuss]]]
      [V
        [gibt]]]]]
\end{forest}}
}


\iftoggle{teil1}{
\frame[shrink]{

\small
\frametitlefit{Teilstruktur in Merkmalstrukturrepräsentation -- \phonw{}e}

\vfill
\centerline{\begin{forest}
sm edges
[V
  [{NP[\type{nom}]}
    [Aicke]]
  [V
    [{NP[\type{dat}]}
        [dem Eichhörnchen,roof]]
    [V
      [{NP[\type{acc}]} 
        [die Nuss, roof]]
      [V
        [gibt]]]]]
\end{forest}}

\vfill
\centerline{
\ms{
      phon     & \phonliste{ die Nuss gibt }\\
      head-dtr & \onems{ phon \phonliste{ gibt }\\
                       } \\
      non-head-dtrs & \liste{ \ms{ 
                                                  phon     & \phonliste{ die Nuss }\\
                                                  head-dtr & \onems{ phon \phonliste{ Nuss }  \\
                                                                 }\\
                                                  non-head-dtrs & \liste{ \onems{ phon \phonliste{ die } \\
                                                                              }}\\
                                                  }
                       }\\
}
}
\vfill
}


\frame{

\small
\frametitlefit{Teilstruktur in Merkmalstrukturrepräsentation -- \phonw{}e (II)}

\vfill
%\resizebox{\linewidth}{!}{
\label{ms-die-Nuss-gibt}
\onems[head-complement-phrase~]{
      phon    \phonliste{ die Nuss gibt }\\
%      spr \liste{}\\
      comps \ibox{1} \liste{ NP[\type{nom}], NP[\type{dat}]} \\
      head-dtr \onems{ phon \phonliste{ gibt }\\
%                                     spr \liste{}\\
                                     comps \ibox{1}  $\oplus$ \liste{ \ibox{2} } \\
                       } \\
      non-head-dtrs \liste{ \ibox{2} \onems[head-specifier-phrase~]{ 
                                                  phon  \phonliste{ die Nuss }\\
                                                  p-o-s \type{noun}\\
                                                  % acc
%                                                  spr \liste{}\\
%                                                  comps \liste{} \\
                                                  head-dtr \ldots \\
                                                  non-head-dtrs \ldots \\
                                                  }
                       }\\
}
}
\vfill
%}
}{}%teil1

\subsection{Projektion von Kopfeigenschaften}

\frame{

\frametitle{Projektion von Eigenschaften des Kopfes}

\settowidth{\offset}{V[\type{fi}}
\settowidth{\offsetup}{V[\type{fin}}
\centerline{
\scalebox{.9}{%
\begin{forest}
sm edges, for tree={l+=\baselineskip}
[V{[\type{fin}, \comps \eliste]}, name=fin1
	[\ibox{1} NP{[\type{nom}]}
		[Aicke]]
	[V{[\type{fin}, \comps \sliste{ \ibox{1} }]}, name=fin2
		[\ibox{2} NP{[\textit{dat}]}
			[dem Eichhörnchen,roof]]
		[V{[\type{fin}, \comps \sliste{ \ibox{1}, \ibox{2} }]}, name=fin3
			[\ibox{3} NP{[\textit{acc}]}
				[die Nuss,roof]]
			[V{[\type{fin}, \comps \sliste{ \ibox{1}, \ibox{2}, \ibox{3} }]}, name=fin4
				[gibt]]]]]	
tikz={\draw[<->] ($(fin1.south west)+(\offsetup,0)$) to ($(fin2.north west)+(\offset,0)$);
      \draw[<->] ($(fin2.south west)+(\offsetup,0)$) to ($(fin3.north west)+(\offset,0)$);
      \draw[<->] ($(fin3.south west)+(\offsetup,0)$) to ($(fin4.north west)+(\offset,0)$);}
\end{forest}
}}


\medskip
Kopf ist finites Verb


}


\frame{
\frametitle{Merkmalstrukturrepräsentation: der \textsc{head}"=Wert}

\begin{itemize}
\item mögliche Merkmalsgeometrie:\\
      \ms{ phon   & list~of~phoneme strings\\
           \blau<2->{p-o-s}  & \blau<2->{p-o-s} \\
           \blau<2->{vform} & \blau<2->{vform} \\ 
           spr    & list~of~signs\\
           comps  & list~of~signs\\
           arg-st & list~of~signs\\
         }\\
\pause
\item mehr Struktur, Bündelung der Information, die projiziert werden soll:\\
      \ms{ phon & list~of~phoneme strings\\
           head & \ms{
                  \blau<2->{p-o-s} & \blau<2->{p-o-s}\\
                  \blau<2->{vform} & \blau<2->{vform}\\
                  } \\
           spr    & list~of~signs\\
           comps  & list~of~signs\\
           arg-st & list~of~signs\\
         }\\
\end{itemize}
}

\frame{

\frametitle{Verschiedene Köpfe projizieren unterschiedliche Merkmale}

\begin{itemize}
\item \textsc{vform} ist nur für Verben sinnvoll
\item pränominale Adjektive und Nomina projizieren Kasus
\item Mögliche Struktur: eine Struktur mit allen Merkmalen:\\
      \ms{
                  p-o-s & p-o-s\\
                  vform & vform\\
                  case  & case\\
         }

      Bei Verben hat \textsc{case} keinen Wert, bei Nomina \textsc{vform} keinen Wert
\pause
\item Besser: verschiedene Typen von Merkmalstrukturen
      \begin{itemize}
      \item für Verben:

        \ms[verb]{
        vform & vform\\
        }
      \item für Nomina

        \ms[noun]{
        case & case\\
        }
        \end{itemize}
\end{itemize}
}


\subsection{Ein Lexikoneintrag mit Kopfmerkmalen}
\frame{
\frametitle{Ein Lexikoneintrag mit Kopfmerkmalen}


\begin{itemize}
\item Ein Lexikoneintrag besteht aus:

\mbox{\type{gibt}:}

\ms{ \visible<2->{phon}   & \visible<2->{\phonliste{ gibt }}\\
     \visible<3->{head}   & \visible<3->{\ms[verb]{ vform & fin \\}} \\
     \visible<4->{arg-st} & \visible<4->{\liste{ NP[\type{nom}], NP[\type{dat}], NP[\type{acc}] }} \\
}
\pause
\begin{itemize}
\item phonologischer Information
\pause
\item Kopfinformation (part of speech, Verbform, \ldots)
\pause
\item Argumentstruktur: einer Liste von Merkmalsbeschreibungen
\pause
\item Wegen Argumentrealisierungsprinzip wird \argst auf \spr und \comps gemappt.
\pause
\item \spr ist für alle finiten Verben die leere Liste. Also: \comps = \argst.
\end{itemize}

\end{itemize}



}


\frame{
\frametitle{Kopfmerkmalsprinzip (Head Feature Principle)}


\begin{itemize}
\item In einer Struktur mit Kopf sind die Kopfmerkmale der Mutter token-identisch
mit den Kopfmerkmalen der Kopftochter.

\medskip
\type{headed-phrase} \impl
\ms{ 
head \ibox{1}\\
head-dtr$|$head \ibox{1}\\
} \\

\pause
%\item Kodierung der Prinzipien im Typsystem: \citet{Krieger94a} und \citet{Sag97a}
\item \type{head-complement-phrase} ist Untertyp von \type{headed-phrase}\\
      $\to$ Beschränkungen gelten auch
\pause
\item \type{head-complement-phrase} erbt Eigenschaften von \type{headed-phrase}.
\end{itemize}


}

\subsection{Typen}
\subsubsection{Ein nicht-linguistisches Beispiel für Mehrfachvererbung}

\frame{
\frametitlefit{Typen: Ein nicht-linguistisches Beispiel für Mehrfachvererbung}


\vfill
%\oneline{%
\centerline{%
\begin{forest} 
type hierarchy
[elektrisches Gerät, calign=midpoint, calign children={1}{2},
  [druckendes Gerät,calign=first
    [Drucker,calign=first
      [Laser-Drucker]
      [\ldots]]
    [Kopierer]]
  [scannendes Gerät,calign=last
    [,identify=!r12]
    [Scanner,calign=first
      [Negativ-Scanner]
      [\ldots]]]
  [\ldots]]
\end{forest}}
\vfill

}

\frame{
\frametitle{Eigenschaften von Typhierarchien}


\begin{itemize}[<+->]
\item Subtypen erben Eigenschaften und Beschränkungen\\von ihre(n) Supertypen.

\item Generalisierungen können erfasst werden:\\
      Allgemeine Beschränkungen werden an oberen Typen repräsentiert.

\item Speziellere Typen erben diese Information von ihren Obertypen.

\item Dadurch Repräsentation von Information ohne Redundanz möglich
\end{itemize}

}

\subsubsection{Linguistische Generalisierungen im Typsystem}
\frame{
\frametitle{Linguistische Generalisierungen im Typsystem}

\begin{itemize}
\item Typen bilden Hierarchie
\pause
\item oben steht der allgemeinste Typ
\pause
\item Information über Eigenschaften von Objekten eines bestimmten Typs werden beim Typ spezifiziert.
\pause
\item Untertypen ererben diese Eigenschaften
\pause
\item Beispiel: Lexikoneintrag in Meyers Lexikon. Verweise auf übergeordnete Konzepte, keine
      Wiederholung der bereits beim übergeordneten Konzept aufgeführten Information\\
\pause
\item Der obere Teil der Typhierarchie ist für alle Sprachen relevant (Universalgrammatik oder
  CoreGram; \citealt{MuellerCoreGram}).
\pause
\item Spezifischere Typen können sprachklassen- oder sprachspezifisch sein.
\end{itemize}


}


\frame{
\frametitle{Typhierarchie für \type{sign}}

\vfill
\centerline{%
\begin{forest}
type hierarchy
[sign,
   calign=fixed angles,
   calign angle=60
  [word]
  [phrase
    [\ldots]
    [headed-phrase
      [head-specifier-phrase]
      [head-complement-phrase]
      [head-adjunct-phrase]]]]
\end{forest}}
\vfill
\centerline{alle Untertypen von \type{headed-phrase} erben Beschränkung}
\vfill
}


\frame{
\frametitle{Kopf-Komplement-Schema + Kopfmerkmalsprinzip}


\ms[head-complement-phrase~]{
\blau{head}   & \blau{\ibox{1}} \\
comps & \ibox{2} \\[2mm]
\blau{head-dtr}  & \ms{ \blau{head}   & \blau{\ibox{1}} \\
                 comps & \ibox{2} $\oplus$ \sliste{ \ibox{3} } \\
               } \\
non-head-dtrs  & \sliste{ \ibox{3} } \\
}

Typ \type{head-complement-phrase} mit \blau{von \type{headed-phrase} ererbter Information}


}



\frame{

\frametitle{\normalsize Teilstruktur in Merkmalstrukturrepräsentation}



\centering
\scalebox{0.7}{%
\onems[head-complement-phrase~]{
      phon  \phonliste{ die Nuss gibt }\\
      head  \blau{\ibox{1}}\\
      comps \ibox{2} \sliste{ NP[\nom], NP[\dat]} \\
      head-dtr \onems[word]{ phon \phonliste{ gibt }\\
                                     head \blau{\ibox{1} \ms[verb]{vform & fin \\
                                                            }}\\
                                     comps \ibox{2}  $\oplus$ \sliste{ \ibox{3} } \\
                       } \\
      non-head-dtrs \liste{ \ibox{3} \onems[head-specifier-phrase~]{ 
                                        phon \phonliste{ die Nuss }\\
                                        head \ms[noun]{ cas & acc\\
                                                      } \\
%                                        comps   \eliste \\
                                        head-dtr \ldots\\
                                        non-head-dtrs \ldots\\
                                     }
                       }\\
}}
}

\frame{
\frametitle{\textsc{mod} als Kopfmerkmal}

\ea
\label{le-treue-head-mod}
\emph{treues}:\\
\ms{ 
   phon & \phonliste{ treues }\\
   head & \ms[adj]{ %prd & $-$ \\
                        mod &  \nbar\\
                      } \\
   spr & \eliste\\
   comps & \liste{ NP[\type{dat}] } \\
}
\z
\begin{itemize}
\item Bei \emph{dem König treues} wird \headw zur Mutter nach oben gegeben.
\item Deshalb \modw von \emph{dem König treues} mit dem von \emph{treues} identisch.
\end{itemize}

}

\frame{
\frametitle{Kopf-Adjunkt-Schema}


\begin{schema}[Kopf-Adjunkt-Schema (vorläufige Version)]
\label{ha-schema-prel}
\textit{head"=adjunct"=phrase} \impl\\
\is{Schema!Kopf"=Adjunkt"=}
\ms{ 
head"=dtr      & \ibox{1} \\[2mm]
non-head"=dtrs & \liste{ \ms{ head$|$mod & \ibox{1} \\
                              spr & \eliste\\
                              comps   & \eliste \\
                           }}\\
}
\end{schema}

\begin{itemize}
\item \modw von Nicht-Kopftochter mit Kopftochter identifiziert.
\pause
\item \sprl und die \compsl der Nicht"=Kopf"|tochter leere Liste $\to$ nur vollständig gesättigte Adjunkte 
zugelassen
\ea[*]{
die Wurst in
}
\z
\pause
\item Keine Spezifikatoren oder Komplemente abgebunden. \\
Müssen zur Mutter weitergereicht werden.
\end{itemize}
}

\subsection{Das Valenzprinzip}
\label{Abschnitt-Valenzprinzip}

\frame{
\frametitle{Das Valenzprinzip}

\ea
\label{type-head-non-specifier-phrase}
\type{head-non-specifier-phrase}\istype{head"=non"=specifier"=phrase} \impl\\
\onems{
      spr \ibox{1} \\
      head-dtr$|$spr \ibox{1} \\
      }
\z
Der \sprw der Mutter wird mit dem der Kopftochter identifiziert.

\pause

\ea
\label{type-head-non-complement-phrase}
\type{head-non-complement-phrase}\istype{head"=non"=complement"=phrase} \impl\\
\onems{
      comps \ibox{1} \\
      head-dtr$|$comps \ibox{1} \\
      }
\z

}

\frame{
\frametitle{Typhierachie mit negierten Typen}


\centerline{%
\scalebox{.8}{%
\begin{forest}
type hierarchy
[sign,
 for tree={
   calign=fixed angles,
   calign angle=60
 } 
  [word]
  [phrase
    [\ldots]
    [headed-phrase
      [head-non-specifier-phrase,name=non-specifier
        [head-complement-phrase,name=complement]]
      [head-non-complement-phrase,name=non-complement
        [head-adjunct-phrase, no edge,name=head-adjunct-phrase]
        [head-specifier-phrase,name=specifier]]]]]
\draw (non-specifier.south) to (head-adjunct-phrase.north);
\draw (non-complement.south) to (head-adjunct-phrase.north);
\end{forest}}}



\begin{prinzip-break}[Valenzprinzip]
In Strukturen mit Kopf entspricht jedes Valenzmerkmal des Mutterknotens dem Wert des Valenzmerkmals der Kopf"|tochter
minus den als Nicht"=Kopf"|töchter realisierten Argumenten.
\end{prinzip-break}

}

\frame{
\frametitle{Klammerung in Nominalphrasen}

\vfill
\hfill
\begin{forest}
sm edges
[{N[\spr \eliste, \comps \eliste]}
  [Det [das]]
  [{N[\spr \nliste{ Det }, \comps \eliste]} 
    [N\feattab{\spr \nliste{ Det },\\ \comps \nliste{ NP }} [Bild]]
    [NP
      [des Gleimtunnels, roof]]]]
\end{forest}
\hfill
\begin{forest}
sm edges
[{N[\spr \eliste, \comps \eliste]}, s sep+=3ex
  [{N[\spr \eliste, \comps \nliste{ NP }]} 
    [Det [das]]
    [N\feattab{\spr \nliste{ Det },\\ \comps \nliste{ NP }} [Bild]]]
  [NP
    [des Gleimtunnels, roof]]]
\end{forest}\hfill\mbox{}

\pause
\ea
alle [[großen Seeelefanten] und [grauen Eichhörnchen]]\\
\z
\begin{itemize}
\item Der Quantor \emph{alle} bezieht sich auf \emph{großen Seeelefanten und grauen Eichhörnchen}. 
\item Würde man erst \emph{alle} mit \emph{große Seeelefanten} verbinden,\\
 hätte man mit \emph{grauen Eichhörnchen} (Dativ) einen unintegrierbaren Rest. 
\end{itemize}

}

\frame{
\frametitle{Das Kopf-Spezifikator-Schema}

Auch in SVO-Sprachen wird erst VP gebildet und dann das Subjekt dazugenommen.

\begin{schema}[Kopf-Spezifikator-Schema (vorläufige Version)]
\label{schema-head-specifier-prel-comps-leer}
\type{head-specifier-phrase}\istype{head"=specifier"=phrase} \impl\\
\onems{
      spr \ibox{1} \\
      head-dtr \ms{ spr   & \ibox{1} $\oplus$ \sliste{ \ibox{2} } \\
                    comps & \eliste\\}\\
      non-head-dtrs \sliste{ \ibox{2} }\\
      }
\end{schema}

Also: \compsl der Kopftochter muss leer sein.

}

\frame{
\frametitle{Startbedingung}

\begin{itemize}
\item Struktur für
\ea
Aicke dem Eichhörnchen die Nuss gibt
\z
\ea
\ms{
head  & \ms[verb]{vform & fin \\
              }\\
spr   & \eliste\\
comps & \eliste\\
}
\z

\end{itemize}

}

\subsection{Übungsaufgaben}

\frame{
\frametitle{Übungsaufgaben}


\begin{enumerate}
\item Zeichnen Sie einen Syntaxbaum für (\mex{1}):
\ea
dass die Frau das spannende Buch liest
\z
Markieren Sie die Kanten im Baum mit Ad für Adjunkt, Ar für Argument und
H für Kopf.
\item Geben Sie die vollständige Merkmalstruktur für (\mex{1}) an:
      \ea
      Schläft das Kind?
      \z
\end{enumerate}

}
